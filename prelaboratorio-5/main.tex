\documentclass{article}
\usepackage{graphicx} % Para incluir imágenes
\usepackage[spanish]{babel}
\usepackage{makeidx}
\usepackage{geometry}
\usepackage{amsmath}


\geometry{a4paper, total={170mm,257mm}, left=20mm, top=20mm}


\begin{document}
\begin{titlepage}
    \centering
    \Large
    Universidad Central de Venezuela\\
    Facultad de Ingeniería\\
    Escuela de Ingeniería Eléctrica
    \vspace*{8cm}

    \Huge
    \textbf{Prelaboratorio N° 11: } 

    \textbf{Multivibradores}
    \vfill


    \Large

    Emerson Warhman \\
    C.I. 25.795.480 \\
    \today

\end{titlepage}


En primer lugar identificamos la etapa del amplificador de diferencial en el diagrama del amplificador multietapas, la cual es la que contiene a los transistores $Q1$ y$Q2$.

Ahora procedemos a calcular los puntos estáticos de operación, para ello tomamos los capacitores como circuitos abiertos, ya que estamos trabajando en DC y empezamos a calcular las corrientes en los transistores.

Asumiremos que la corriente que pasa por $R5$ es 0.

Para calcular la corriente de base se usará el teorema de thevenim para sustituir $R_1$ y $R_2$ por una fuente y una resistencia que pasa por la base de $Q_1$.

Para calcular el valor de la resistencia equivalente de thevenim:
$$R_{th} = R_1 // R_2$$

cómo $R_1 = R_2$
$$Rth = \frac{R_1}{2} = 50k \Omega$$

Ahora, calculamos el valor de la fuente de thevenim aplicando un divisor de voltaje:

$$V_{th} = \frac{R_2 ( V_{cc} - V_{EE}}{R_1 + R_2}$$

$$V_{th} = 10 V$$

Ahora aplicando LVK en la malla B-E ($Q_4$):
\begin{equation*}
V_{th} - R_{th}I_b - V_{be} - R_7(I_e) = 0
\end{equation*}

despejando $I_b$, tenemos

\begin{equation*}
    I_b = \frac{V_{th} - V_{be}}{R_{th} + (\beta + 1) R_7}
\end{equation*}

usando un $\beta = 230$ entonces

$$I_{b} = 2,65\mu A$$

Ahora, para calcular $I_c$:

$$I_{c} = \beta I_b = 0,62mA$$

Y por último, para calcular $V_{ce}$ aplicamos LVK:

$$Vcc - I_cR_3 - I_eR_4 - V_{ce} - V_{ee} = 0$$

Despejando $V_{ce}$ y aproximando $I_c \approx I_e$, tenemos 

$$V_{ce} = V_{cc}  - V_{ee} - I_c (R_4 + R_3)$$

$$V_{ce} = 7,79 V$$

Como el circuito es simétrico, los voltajes y corrientes $I_b$, $I_e$, $I_c$, $V_{ce}$ y $V_{be}$ son iguales.


Ahora, para la parte dinámica, calculamos los parámetros del transistor:

$$V_t = 26 mV$$

$$gm = \frac{I_c}{V_t} = \frac{0,62 mA}{26 mV}$$

$$gm = 23,85 x 10 ^{-3}$$

$$R_\pi = \frac{\beta}{gm} = \frac{230}{23,85x10^{-3}}$$

$$ R_\pi = 9,6k \Omega$$

Tomando el amplificador en su modo común, en primer lugar calculamos la impedancia de entrada, analizando el circuito obtenemos la expresión:

$$ Z_c = R_1 || R_2 || (R_\pi + (1 + gmR_\pi)R_4$$

sustituyendo los valores tenemos:

$$ Z_c = 49K\Omega$$

Ahora la impedancia de salida es:

$$ Z_o = gmR_\pi R_3$$

$$ Z_0 = 1 M\Omega$$

Ahora la ganancia es:

$$A_c = - \frac{ gmR_\pi R_3}{R_\pi + (1 + gmR_\pi)R_4}$$

$$A_c = 0,31$$

Ahora analizamos en modo diferencial:

$$Z_d = R_1 || R_2 || (2R_\pi + (1+gmR_\pi )R_5)$$

$$Z_d = 43,99 k\Omega$$

$Z_o$ es la misma que en modo común:

$$Z_0 = 4,9 K\Omega$$

Por último calculamos la ganancia en modo diferencial:

$$A_d = -\frac{gmR_\pi R_3}{2R_\pi + (1 + gmR_\pi) R_5}$$

$$A_d = -2,96 $$

Procedemos a analizar la etapa elevadora (Etapa con $Q_3$)

aplicamos thevenim de manera que:

$$R_{th}= R15 || R10$$

$R_{th}=29 k\Omega$

$$V_{th} = \frac{R15}{R15 + R10} (V_{cc} - V_{ee})$$

$$V_{th} = 2.61 V $$

Aplicando LVK en la malla del emisor:

$$V_{th} - V_{be} = R11 \beta I_{b} + R_{th}I_b$$

$$ Ib = \frac{V_{th} - V_{be}}{R11\beta + R_{th}}$$

usando un $\beta = 230$

$$ I_b = 10.03 \mu A$$

$$Ic = \beta I_b$$

$$I_c = 2.37 mA$$

Aplicando LVK:

$$V_{ce} = V_{cc} - V_{ee} - I_eR11 - I_cR16$$

$$ V_{ce} = 2.27 V$$

Haciendo el analisis AC, tenemos 
$$gm = \frac{I_{c}}{V_t} = 0.09$$

$$R_{\pi} = \frac{\beta}{gm} = 2523$$

La impedancia de entrada es:

$$Z_i = R15 || R10 || r_\pi $$
$$z_i = 2.31 k \Omega $$

$$Z_o = \frac{r_\pi + R15 || R10}{1 + gmr_\pi} $$

$$Z_0 = 134$$

Ganancia:

$$A = \frac{gmr_\pi R16}{r_\pi} = 49.61$$

Ahora la etapa del amplificador de potencia:

Obtenemos los puntos estáticos de operación, para ello tomamos los capacitores como circuitos abiertos, ya que estamos trabajando en DC y empezamos a calcular las corrientes en el transistor $Q4$.

Asumiremos que las corrientes de base $I_{bQ5}$ e $I_{bQ6}$ son muy pequeñas en comparación con la corriente $I_{R17}$ por tanto la tomaremos como despreciables.

Ahora aplicando LCK en el multiplicador de voltaje ($Q_4$):
\begin{equation}
I_{RV1} + I_{cQ4} = I_{R17}    
\end{equation}

Si ahora asumimos $I_bQ4$ despreciable:

\begin{equation}
    I_{RV1} = \frac{V_{BEQ4}}{XR_{V1}}
\end{equation}

Aplicando LVK tenemos:

\begin{equation}
    V_{CEQ4} = I_{RV1} * R_v1
\end{equation}

Usando (2) y (3):

$$ V_{CEQ4} = \frac{V_{BEQ4}}{X*R_{V1}} * R_{V!}$$

\begin{equation}
    V_{CEQ4} = \frac{V_{BEQ4}}{X}
\end{equation}

Debido a que el amplificador es de clase AB el voltaje $V_{CEQ4}$ tiene que ser dos veces el voltaje base emisor $V_{be}$ para los transistores $Q5$ y $Q6$ estén lo más cerca posible de la zona activa y se pueda reducir el efecto crossover de la salida.

Aplicando LVK entre las dos referencias tenemos:

$$10V - R_{17}*I_{17} - 2V_{beQ4} - R_{12}*I_{17} + 10V = 0$$

despejando $I_{17}$:

\begin{equation}
    I_{17} = \frac{20 - 2V_{beQ4}}{R_{17} + R_{12}}
\end{equation}

Usando (1), (2) y (5) tenemos:

\begin{equation}
    I_{cQ4} = \frac{20 - 2V_{beQ4}}{R_{17} + R_{12}} - \frac{2V_{beQ4}}{R_{V1}}
\end{equation}

Usando los datos y la ecuacion (6):

$V_{beQ4} = 0.62 V$ 

$R_{17} = R_{12} = 22k\omega$ 

$R_{V1} = 10k\omega$ 

$$    I_{cQ4} = \frac{20 - 2 * 0.62 V}{22k\omega + 22k\omega} - \frac{2 * 0.62 V}{10 k\omega} $$

$$ I_{cQ4} = 302.36 \upsilon A$$

Tomando $hfe_{Q4} = 230$

\begin{equation}
    I_{bQ4} = I_{cQ4} / hfe
\end{equation}

$$ I_{bQ4} = 1.31 \upsilon A$$

$$ V_{ceQ4} = 2 * 0.62 V = 1.24 V$$

Ahora, volviendo a despreciar las corriente de base y aplicando LVK en la malla con los transistores

$$V_{ceQ4} - V_{beQ5} - I_{eQ5}* (R_{13} + R_{14}) - V_{beQ6} = 0$$

despejando $I_{eQ5}$ 

\begin{equation}
    I_{eQ5} =\frac{ V_{ceQ4} - V_{beQ5} - V_{beQ6} }{R_{13} + R_{14}}
\end{equation}

Tomando $V_{beQ5} = V_{beQ4} = 0.62 V$ y $V_{beQ6} = 0.55 V$, entonces: 

$$I_{eQ5} = \frac{ 1.24 V - 0.62 V - 0.55 V }{10 \omega + 10 \omega }$$

$$ I_{eQ5} = I_{eQ6} \approx I_{cQ5} \approx I_{cQ6} = 350\upsilon A$$

Basado en las I de emisor ahora calulamos las corrientes de base, asumiendo que $hfe_{Q5} = 230 $ y $hfe_{Q6} = 150$

$$I_{bQ5} = I_{cQ5} / hfe = 350\upsilon A / 230 = 1.52 \upsilon A $$

$$I_{bQ6} = I_{cQ6} / hfe = 350\upsilon A / 150 = 2.33 \upsilon A $$

Se asume que $V_{ceQ5} = V_{ceQ6} $, por tanto:

$$ 10 - 2 V_{ceQ5} - (R_{14} + R_{13}) * I_{eQ5} + 10 = 0$$

despejando $V_{ceQ5}$ tenemos:

\begin{equation}
    V_{ceQ5} = V_{ceQ6} = \frac{ 20 - (R_{14} + R_{13}) * I_{eQ5} }{2}
\end{equation}

por tanto 

$$ V_{ceQ5} = V_{ceQ6} = \frac{ 20 V - (10 + 10) \omega * 350\upsilon A }{2} = 9.99 V$$

Respuesta en frecuencia:

Condensadores de baja frecuencia:
    
Los capacitores de baja serán C1, C2, C3, C5, C6 y C7. para obtener las frecuencias de corte usaremos:

$$ \omega_{ci} = \frac{1}{C_i R_{eqCi}}$$

Para $C_1$ tenemos:

$$ \omega_{C1} = \frac{1}{C_1 * R_{eqC1}} =  \frac{1}{42K \times 10^{-6}} = 23.34 rad / s$$ 

Para $C_2$ tenemos:

$$ \omega_{C2} = \frac{1}{C_2 * R_{eqC2}} =  \frac{1}{10K \times 10^{-6}} = 140.34 rad / s$$ 

Para $C_3$ tenemos:

$$ \omega_{C3} = \frac{1}{C_3 * R_{eqC3}} =  \frac{1}{10K \times 10^{-6}} = 23.34 rad / s$$ 

Para $C_5$ tenemos:

$$ \omega_{C5} = \frac{1}{C_5 * R_{eqC5}} =  \frac{1}{10K \times 10^{-6}} = 0.84 rad / s$$ 

Para $C_6$ tenemos:

$$ \omega_{C6} = \frac{1}{C_6 * R_{eqC6}} =  \frac{1}{10K \times 10^{-6}} = 442.6 rad / s$$ 

Para $C_7$ tenemos:

$$ \omega_{C7} = \frac{1}{C_7 * R_{eqC7}} =  \frac{1}{10K \times 10^{-6}} = 65.34 rad / s$$

sabiendo que: 

$$ f = \frac{\omega}{2\pi} = \frac{1}{T} $$

\begin{tabular}{|c|c|c|}
\hline
Capacitor & Velocidad angular & Frecuencia\\
\hline
C1 & 23.34 rad / s & 3.401 Hz\\
\hline
C2 & 140.34 rad / s & 21.002 Hz\\
\hline
C3 & 23.34 rad / s & 3.561 Hz\\
\hline
C5 & 0.84 rad / s & 0.14 Hz\\
\hline
C6 & 442.6 rad / s & 70.414 Hz\\
\hline
C7 & 65.34 rad / s & 9.013 Hz\\
\hline
\end{tabular}


La frecuencia de corte inferior es la mayor frecuencia entre todos los valores obtenidos, por lo tanto

$$ f_{L} = 70.414 Hz$$

\end{document}
