\begin{table}[ht]
\centering
\begin{tabular}{|c|c|c|c|}
\hline
\(A\) & \(\Delta A\) & Valor Teórico & \% Error \\ \hline
2.95 & 0.25 & 3.00 & 1.67 \\ \hline
\end{tabular}
\caption{Mediciones de ganancia del oscilador y su error porcentual.}
\label{tab:mediciones-ganancia-error}
\end{table}

Al observar las tablas \ref{tab:mediciones-ganancia-error} y \ref{tab:mediciones-periodo-error} tenemos que el error de la ganancia es de 1.67 \% y el error máximo del periodo es de 9.0\% los cuales son bastantes precisos.

\begin{table}[ht]
\centering
\begin{tabular}{|c|c|c|c|c|}
\hline
Descripción & \(T\) [$\mu s$] & \(\Delta T\) [$\mu s$] & Valor Teórico [$\mu s$] & \% Error \\ \hline
Excursión máxima & 220 & 10 & 200 & 10 \\ \hline
Excursión mínima & 200 & 10 & 200 & 0 \\ \hline
Saturada & 200 & 10 & 200 & 0 \\ \hline
\end{tabular}
\caption{Mediciones de período del oscilador con control de amplitud y su error porcentual.}
\label{tab:mediciones-periodo-error}
\end{table}

\begin{table}[ht]
\centering
\begin{tabular}{|c|c|c|c|c|}
\hline
Descripción & \(x\) & \(\Delta x\) & Valor Teórico & \% Error \\ \hline
Excursión mínima & 0.52 & 0.03 & 0.50 & 4.00 \\ \hline
Excursión máxima & 0.63 & 0.03 & 0.66 & 4.55 \\ \hline
\end{tabular}
\caption{Mediciones de excursión y su error porcentual.}
\label{tab:mediciones-excursion-error}
\end{table}

La tabla \ref{tab:mediciones-excursion-error} muestra que los errores son de 4 \% para la excursión mínima y 4.55 \% para la excursión máxima, los cuales son bastante precisos.


En el caso del oscilador de puente de Wien no hubo una alteración notable del resultado deseado debido al Slewrate.
