Este informe experimental analizó tres configuraciones no lineales fundamentales de amplificadores operacionales:

\subsection{1. Oscilador de Puente de Wien}
\begin{itemize}
\item Genera señales sinusoidales mediante realimentación positiva
\item Frecuencia de oscilación: $f_o = \frac{1}{2\pi RC}$
\item Se verificó la condición de oscilación ($\beta A_v \geq 1$) y se midió distorsión armónica
\item Aplicación: Generación de señales de prueba en sistemas de audio
\end{itemize}

\subsection{2. Multivibradores}
\begin{itemize}
\item \textbf{Astable}: Generador de onda cuadrada autónomo
  \begin{itemize}
  \item Período medido: $T = 2R_3C_1\ln(1+\frac{2R_1}{R_2})$
  \item Error experimental vs teórico: 4.55\%
  \end{itemize}
  
\item \textbf{Monoestable}: Produce pulsos de ancho controlado
  \begin{itemize}
  \item Ancho de pulso: $t_w = R_3C_1\ln(3)$
  \item Se comprobó la dependencia con $R_3C_1$ (error: 4\%)
  \end{itemize}
\end{itemize}

\subsection{3. Generador de Funciones}
\begin{itemize}
\item Circuito integrado para producir señales triangulares/cuadradas
\item Se caracterizó la relación frecuencia-capacitancia ($f \propto 1/C$)
\item No linealidades observadas en amplitudes altas
\end{itemize}
