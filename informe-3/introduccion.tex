Los amplificadores operacionales (op-amps) son dispositivos versátiles que encuentran aplicaciones tanto en circuitos lineales como no lineales. Mientras que en configuraciones lineales se utilizan principalmente para operaciones de amplificación y filtrado, sus aplicaciones no lineales abren un panorama de posibilidades en generación y conformación de señales. Este informe explora tres configuraciones fundamentales de op-amps en régimen no lineal: el oscilador de puente de Wien, los multivibradores (astable y monoestable) y los generadores de funciones.

El \textbf{oscilador de puente de Wien} representa una aplicación clásica en generación de señales sinusoidales. Su diseño aprovecha el balance entre una red de realimentación positiva (para mantener las oscilaciones) y un mecanismo de control de amplitud (usualmente mediante diodos o JFETs) que garantiza estabilidad en la señal de salida. La frecuencia de oscilación está determinada por la relación $f_o = \frac{1}{2\pi RC}$, demostrando cómo componentes pasivos simples pueden definir características fundamentales de la señal generada.

Los \textbf{multivibradores} ilustran la capacidad de los op-amps para trabajar en conmutación:
\begin{itemize}
    \item El \textbf{astable} opera como oscilador libre, generando una onda cuadrada continua cuya frecuencia depende de los elementos RC en el circuito
    \item El \textbf{monoestable} produce un pulso de duración fija ante un disparo externo, siendo útil en aplicaciones de temporización
\end{itemize}

Los \textbf{generadores de funciones} amplían estas capacidades para producir formas de onda complejas (triangulares, diente de sierra, etc.), a menudo mediante la integración controlada de señales cuadradas. Estos circuitos encuentran aplicaciones prácticas en sistemas de comunicaciones, instrumentación médica y equipos de prueba electrónicos.

Este estudio experimental permitirá verificar los principios teóricos de operación, analizar las relaciones entre los componentes y los parámetros de salida, y evaluar las limitaciones prácticas de estos circuitos. Particular atención se dedicará al análisis de distorsión armónica en el oscilador Wien y a la precisión temporal en los multivibradores, aspectos críticos para aplicaciones reales.

