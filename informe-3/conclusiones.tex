\section{Conclusiones}

A lo largo de este trabajo de laboratorio, se estudiaron diferentes aspectos de los amplificadores operacionales y sus aplicaciones no lineales, llegando a las siguientes conclusiones:

\begin{itemize}
    \item La configuración de oscilador de puente de Wien es practicamente la esperada, el error máximo fue del 4\%. Para esta configuración no hubo alteración debido al Slewrate
    
    \item Para las configuraciones tanto de multivibradores y de generador de funciones se obtuvieron errores algo más considerables (error mínimo 5\% y máximo 36\%), dichos errores tuvieron que ver en gran medida al efecto del Slewrate del amplificador. El Slewrate también alteró la forma esperada de las señales por lo es necesario tomar en cuenta de este parámetros para el diseño de multivibradores y para generadores de funciones.

\end{itemize}

Estas observaciones demuestran la importancia de considerar las no idealidades y limitaciones prácticas al trabajar con circuitos amplificadores operacionales, así como la necesidad de seleccionar cuidadosamente los componentes y condiciones de operación para obtener los resultados deseados.