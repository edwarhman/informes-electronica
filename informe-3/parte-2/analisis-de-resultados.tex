\subsubsection{Multivibrador astable}

\begin{table}[ht]
\centering
\begin{tabular}{|c|c|c|c|}
\hline
\(V_C\) [V] & \(\Delta V_C\) [V] & Valor teórico [V] & Error \% \\ \hline
2.60 & 0.20 & 2.00 & 30.00 \\ \hline
\end{tabular}
\caption{Comparación de mediciones y valores teóricos de la amplitud $V_c$ del multivibrador astable.}
\label{tab:comparacion-multivibrador-astable-amplitud}
\end{table}

\begin{table}[ht]
\centering
\begin{tabular}{|c|c|c|c|c|}
\hline
Descripción & \(T\) [\(\mu\)s] & \(\Delta T\) [\(\mu\)s] & Valor teórico [\(\mu\)s] & Error \% \\ \hline
Sin tomar en cuenta el SR &230 & 10 & 200 & 15.00 \\ \hline
Tomando en cuenta el SR & 190 & 10 & 200 & 5.00 \\ \hline
\end{tabular}
\caption{Comparación de mediciones y valores teóricos de tiempo del multivibrador astable.}
\label{tab:comparacion-tiempo-multivibrador-astable}
\end{table}

Del cuadro \ref{tab:comparacion-multivibrador-astable-amplitud} se observa un error del 30\% para el valor de la amplitud de la onda $Vc$, lo cual es algo elevado. Este error puede ser ocasionado porque durante el tiempo de respuesta debido al Slewrate el circuito deja de comportarse adecuadamente y el condensador se sigue cargando o descargando por encima del valor esperado, esto hace que la amplitud medida sea algo mayor a la esperada.

El cuadro \ref{tab:comparacion-tiempo-multivibrador-astable} muestra un error del 15\% en el valor del periodo con respecto al valor teórico esperado cuando no se toma en cuenta el tiempo de retardo del Slewrate, cuando tomamos en cuenta el retardo del Slewrate el error es de tan solo 5 \%, lo cual es aceptable.


\subsubsection{Multivibrador monoestable}

\begin{table}[ht]
\centering
\begin{tabular}{|c|c|c|c|}
\hline
\(T\) [ms] & \(\Delta T\) [ms] & Valor teórico [ms] & Error \% \\ \hline
13.60 & 0.40 & 10.00 & 36.00 \\ \hline
\end{tabular}
\caption{Comparación de mediciones y valores teóricos del multivibrador monoestable.}
\label{tab:comparacion-multivibrador-monostable}
\end{table}

Viendo el cuadro \ref{tab:comparacion-multivibrador-monostable} obtenemos que el error del periodo de la onda $V_c$ es de $36.00$ es bastante elevado, sin embargo es poco probable a que este error sea debido al Slewrate ya que para esta configuración  $T_{sr}$ tiene un valor aproximado de $20\mu s$ lo cual es insignificante comparándolo con el periodo $T=10ms$. Esto también se hace evidente debido al hecho de que la forma de onda de $V_t$ es exponencial, y de que la onda $V_o$ tiene forma cuadrada. No hubo distorsión debido al Slewrate. El error observado puede ser debido a que el valor comercial de una de las resistencias $150k$ es algo distinto al valor teórico ($160k$), además de los errores debidos a los elementos como las resistencias y el condensador.

\subsubsection{Ganancias topologías clásicas}

A partir de los resultados obtenidos, se presenta el siguiente análisis de las ganancias medidas para cada topología:

\begin{table}[ht]
\centering
\begin{tabular}{|c|c|c|c|c|}
\hline
Topología & Ganancia & $\Delta$ Ganancia & Ganancia teórica & Error (\%) \\ \hline
Inversor & -2.00 & 0.45 & -2 & 0 \\ \hline
No Inversor & 3.00 & 0.63 & 3 & 0 \\ \hline
Restador & 4.00 & 1.77 & 2 & 100 \\ \hline
\end{tabular}
\caption{Comparación de ganancias medidas vs teóricas}
\label{tab:comparacion-ganancias}
\end{table}

Del análisis de los resultados se puede observar que:

\begin{itemize}
    \item Para el amplificador inversor, se obtuvo una ganancia de $-2.00 \pm 0.45$, lo cual coincide exactamente con la ganancia teórica esperada de $-2$, resultando en un error del 0\%.
    
    \item En el caso del amplificador no inversor, se midió una ganancia de $3.00 \pm 0.63$, que también coincide perfectamente con el valor teórico de $3$, presentando un error del 0\%.
    
    \item Para el amplificador restador, se obtuvo una ganancia de $4.00 \pm 1.77$. Este valor difiere significativamente del valor teórico esperado de $2$, presentando un error del 100\%. Esta discrepancia se debe a un error cometido al momento de tomar la medición, ya que se consideró que el voltaje de entrada era $V_1$ cuando realmente es la resta $V_2 - V_1$, esto llevó a una mala interpretación de los valores observados en el osciloscopio.
\end{itemize}

Los amplificadores inversor y no inversor mostraron un comportamiento muy cercano al ideal, mientras que el restador presentó desviaciones significativas debido a un error humano.

\subsubsection{Efecto del integrador no inversor}

La ilustración \ref{ilus:entrada-salida-integrador-no-inversor} muestra que al pasar una señal cuadrada al amplificador integrador no inversor, la señal de salida es una señal triangular cuya pendiente positiva coincide con el semiciclo positivo de la señal cuadrada, mientras que la pendiente negativa coincide con el semicliclo negativo de la señal cuadrada, este es el comportamiento que se esperaba observar.

\subsubsection{Convertidor de tensión a corriente}

A continuación se presentan los resultados de la corriente medida para cada valor de resistencia, junto con el error porcentual respecto al valor teórico.

\begin{table}[h!]
\centering
\begin{tabular}{|c|c|c|c|}
\hline
I (mA) & \(\Delta I\) (mA) & I Teórica [mA] & Error (\%) \\ \hline
50.00 & 10.30 & 52 & 3.85 \\ \hline
45.50 & 9.370 & 52 & 12.59 \\ \hline
45.50 & 5.080 & 52 & 12.59 \\ \hline
51.30 & 5.730 & 52 & 1.38 \\ \hline
48.10 & 4.420 & 52 & 7.41 \\ \hline
\end{tabular}
\caption{Porcentaje de error del convertidor tensión-corriente.}
\label{tab:analisis-resultados-convertidor-tension-corriente}
\end{table}

\begin{table}[h!]
\centering
\begin{tabular}{|c|c|c|c|}
\hline
I (mA) & \(\Delta I\) (mA) & I Teórica [mA] & Error (\%) \\ \hline
48.07 & 6.982 & 52 & 7.56 \\ \hline
\end{tabular}
\caption{Porcentaje de error promedio del convertidor tensión-corriente.}
\label{tab:analisis-resultados-convertidor-tension-corriente-promedio}
\end{table}

De los cuadros \ref{tab:analisis-resultados-convertidor-tension-corriente} y \ref{tab:analisis-resultados-convertidor-tension-corriente-promedio} se observa que la corriente medida se mantiene relativamente constante alrededor de los 48 mA, con una desviación promedio del 7.56\% respecto al valor teórico de 52 mA. y una desviación máxima del 12.59\%. Esto puede ser debido a que $R_1$ (100k) no era significativamente grande en comparación con $R_2$ (40k). Las incertidumbres en las mediciones son significativas, especialmente en las primeras mediciones, Esto puede ser debido a que para las primeras mediciones los valores medidos de voltaje fueron bajos, o que las primeras resistencias fueron de un valor bajo, siendo su incertidumbre baja y al ser una división la incertidumbre de la corriente fue alta.

El valor teórico de la corriente se obtuvo de la formula \ref{eq:mt-io-fuente-corriente} con $R_1 = 100k\Omega$ y $V_i = 5.2V$.