\documentclass[12pt, a4paper, Spanish]{article}
\usepackage[utf8]{inputenc}
\usepackage{graphicx, enumitem, chngcntr, csquotes, amsmath, mathptmx, amssymb,  float, amssymb, amsfonts, fullpage, caption}% makeidx, enumerate} %despues de float, enumerate
\usepackage{hyperref} %[hidelinks]
\hypersetup{
    colorlinks=true,
    linkcolor=red,     
    urlcolor=cyan,
    citecolor=black}
%\counterwithin{figure}{section}
\renewcommand{\thesection}{\arabic{section}} 
\renewcommand{\thesubsection}{\thesection.\arabic{subsection}}
\renewcommand{\baselinestretch}{1.5} %interlineado
\usepackage[spanish, es-tabla]{babel} %es-tabla añadido
\pagenumbering{arabic}
%\usepackage[left=25mm, right=25mm, top=25mm, bottom=25mm]{geometry}
\usepackage[backend=biber,citestyle=numeric]{biblatex}
%\bibliography{referencias}
%\usepackage[margin=25mm]{geometry}
%\usepackage{hyperref}%para link referenciales
\graphicspath{{imagenes}} %permite importar imagenes dentro de la carpeta imagenes
\usepackage{float}
\usepackage{pdfpages}
\usepackage{bm} % Para usar \bm{}

\usepackage{caption}
\usepackage{newfloat}
\DeclareFloatingEnvironment[
    name=Ilustración,
    listname={Lista de ilustraciones},
    fileext=loi
]{ilustracion}


\begin{document}

    %---PORTADA---

\begin{titlepage}
    \centering
    {\scshape\Large Universidad Central de Venezuela \par}
    {\scshape\Large Facultad de ingeniería \par}
    {\scshape\Large Escuela de ingeniería Eléctrica \par}
    {\scshape\Large Departamento de Electrónica, Computación y Control \par}

    \vspace{6cm}
    {\Large\bfseries Laboratorio N°1: Amplificadores Discretos\par}
    \vspace{6cm}

    \vfill
    \begin{flushright}
        Estudiante:     \par
        Fajardo Carla   \par
        C.I.:27571576   \par
        \vspace{1cm}  
    \end{flushright}
    \vfill
    {\large \today \par}
\end{titlepage}

%\tableofcontents

\newpage

    %---INDICE---

\tableofcontents
\newpage

    %---RESUMEN---
    
\section{Resumen}

En este laboratorio se estudió el amplificador discreto, compuesto por transistores, resistencias y condensadores, analizando sus tres etapas clave: diferencial (encargada de amplificar la diferencia entre señales de entrada), impulsora (que provee la corriente necesaria para excitar la etapa final) y de potencia (que entrega la señal amplificada a la carga). El propósito fue comparar los valores teóricos y experimentales de los puntos de operación, ganancia e impedancias (en modos diferencial y común), así como evaluar su respuesta en frecuencia (ancho de banda y ganancia) y el efecto de la realimentación negativa (estabilización de la ganancia) y positiva (aumento de ganancia con riesgo de inestabilidad). Los resultados demostraron la importancia del diseño individual de cada etapa y su interacción en el sistema, validando los principios teóricos de un amplificador discreto y realimentación. 


    %---INTRODUCCION---

\section{Introducción}

Los amplificadores discretos, implementados con componentes individuales, constituyen un elemento fundamental en electrónica para comprender los principios básicos de amplificación de señales. Este estudio experimental caracteriza el comportamiento de un amplificador discreto, analizando tanto el funcionamiento independiente como acoplado de sus tres etapas principales: la etapa diferencial (encargada del procesamiento inicial de la señal), la etapa impulsora (que condiciona la señal para su amplificación final) y la etapa de potencia (responsable de entregar la energía a la carga).

A través de mediciones precisas de ganancia, impedancias y respuesta en frecuencia, la investigación establece una conexión entre los fundamentos teóricos y su manifestación práctica, con especial énfasis en los efectos de la realimentación negativa (para estabilidad) y positiva (para incremento de ganancia).

Los resultados obtenidos no solo validan los modelos teóricos, sino que además revelan las relaciones críticas entre las diferentes etapas del amplificador. Estas conclusiones proporcionan directrices valiosas para el diseño y optimización de circuitos amplificadores, remarcando la importancia de considerar tanto el comportamiento individual de cada componente como su interacción dentro del sistema completo. 
\newpage

    %---MARCO TEORICO---

\section{Marco Teórico}

\subsection{Constitución interna de un BJT}

El transistor es un dispositivo semiconductor de tres terminales (emisor, base y colector), equivalente a dos diodos PN unidos en sentido opuesto. Existen dos tipos principales según la configuración de sus uniones: NPN y PNP.

Funcionamiento básico:\par

Polarización: la unión base-emisor se polariza en directa, mientras que la unión base-colector se polariza en inversa. Esto permite que circule una corriente inversa por la unión base-colector.

\begin{ilustracion}[H]
    \centering
    \includegraphics[width=0.5\linewidth]{IMAGENES - Marco_Teorico/M_T_1.PNG}
    \caption{Transistores BJT}
    \label{fig:enter-label}
\end{ilustracion}

\subsection{El transistor NPN}

El transistor bipolar NPN es un dispositivo semiconductor compuesto por tres regiones dopadas: una capa delgada de material tipo P (base) intercalada entre dos capas de material tipo N (emisor y colector), fabricadas en un mismo cristal de silicio o germanio. Estas zonas corresponden a los terminales del transistor: emisor (E), base (B) y colector (C).

\subsubsection{Flujo de portadores}

El emisor (altamente dopado) inyecta portadores de carga (electrones) hacia la base.

La base (delgada y poco dopada, $\approx$ 100 veces menos que el emisor/colector) controla el flujo de estos portadores.

El colector (de mayor espesor, $\approx$ 2 veces el del emisor) recoge los portadores que no son recombinados en la base.

\subsubsection{Uniones PN}

\begin{itemize}
    \item Unión emisor-base (P-N): Polarizada en directa para permitir la inyección de electrones.

    \item Unión base-colector (N-P): Polarizada en inversa para facilitar la captura de electrones por el colector. 
\end{itemize}

\begin{ilustracion}[H]
    \centering
    \includegraphics[width=0.5\linewidth]{IMAGENES - Marco_Teorico/M_T_2PNG.PNG}
    \caption{Transistores NPN}
    \label{fig:enter-label}
\end{ilustracion}

\subsection{Transistor Bipolar PNP}

El transistor bipolar PNP es un dispositivo semiconductor con tres regiones (P-N-P), donde una capa delgada N (base) se encuentra entre dos capas P (emisor y colector).

\subsubsection{Polarización:}

\begin{itemize}
    \item Emisor-Base: Polarizada en directa (tensión negativa en emisor respecto a base).

    \item Base-Colector: Polarizada en inversa (tensión negativa en base respecto a colector).
\end{itemize}

\subsubsection{Flujo de corriente:}

\begin{itemize}
    \item El emisor inyecta huecos hacia la base.

    \item La base (delgada y poco dopada) controla el flujo; los huecos no recombinados llegan al colector.
\end{itemize}


Comparación con NPN:

Mismo principio, pero con polaridades y portadores invertidos (huecos en lugar de electrones). \cite{1}

\subsection{Amplificadores - Amplificación}

Los amplificadores son circuitos que se utilizan para aumentar (amplificar) el valor de la señal de entrada, generalmente muy pequeña, y así obtener una señal a la salida con una amplitud mucho mayor a la señal original.

Los amplificadores pueden implementarse con:
\begin{itemize}
    \item Transistores bipolares
    \item Amplificadores operacionales
    \item Tubos o válvulas electrónicas
    \item FETs
\end{itemize} \cite{2}


\subsection{Clasificación de los amplificadores}
Los amplificadores se clasifican según las frecuencias con las que trabajan:

\begin{itemize}
    \item \textbf{Amplificadores de Audiofrecuencia (A.F.) o Baja Frecuencia (B.F.)}: Trabajan dentro de la banda audible
    \item \textbf{Amplificadores de Radiofrecuencia (R.F.)}: Trabajan con altas frecuencias
\end{itemize}

Dentro de estas categorías, se clasifican según su forma de trabajo:

\begin{enumerate}
    \item \textbf{Amplificadores de tensión}: Proporcionan mayor tensión en la salida que en la entrada
    \item \textbf{Amplificadores de potencia}: Proporcionan mayor tensión \textit{y} corriente (amplificación de potencia)
\end{enumerate}

\subsection{Clases de amplificadores de potencia}
Los amplificadores de potencia (para B.F. o R.F.) se clasifican según cuánta señal entregan:

\subsubsection{Amplificador Clase A}
\begin{itemize}
    \item Alta fidelidad de audio
    \item Mayor costo y tamaño
    \item Alta disipación de potencia
    \item Eficiencia típica: 20-30\%
\end{itemize}

\begin{ilustracion}[H]
    \centering
    \includegraphics[width=0.5\linewidth]{IMAGENES - Marco_Teorico/M_T_3.PNG}
    \caption{Amplificador Clase A}
    \label{fig:enter-label}
\end{ilustracion}

\subsubsection{Amplificador Clase B}
\begin{itemize}
    \item Reposo en corte (sin consumo en ausencia de señal)
    \item Eficiencia típica: 70\%
    \item Distorsión de cruce por cero
\end{itemize}

\begin{ilustracion}[H]
    \centering
    \includegraphics[width=0.5\linewidth]{IMAGENES - Marco_Teorico/M_T_4.PNG}
    \caption{Amplificador Clase B}
    \label{fig:enter-label}
\end{ilustracion}

\subsubsection{Amplificador Clase AB}
\begin{itemize}
    \item Combina características de Clase A y B
    \item Menor distorsión que Clase B
    \item Mayor eficiencia que Clase A (50-60\%)
\end{itemize}

\begin{ilustracion}[H]
    \centering
    \includegraphics[width=0.5\linewidth]{IMAGENES - Marco_Teorico/M_T_5.PNG}
    \caption{Amplificador Clase AB}
    \label{fig:enter-label}
\end{ilustracion}

\subsubsection{Amplificador Clase C}
\begin{itemize}
    \item Uso exclusivo en R.F.
    \item Alta eficiencia (70-85\%)
    \item Circuito resonante (tanque LC) en colector
    \item Ecuación de frecuencia de resonancia:
    \begin{equation}
        f_r = \frac{1}{2\pi\sqrt{LC}}
    \end{equation}
\end{itemize}

\begin{ilustracion}[H]
    \centering
    \includegraphics[width=0.5\linewidth]{IMAGENES - Marco_Teorico/M_T_6.PNG}
    \caption{Amplificador Clase C}
    \label{fig:enter-label}
\end{ilustracion}

\subsubsection{Amplificador Clase D}
\begin{itemize}
    \item Amplificadores digitales (conversión PWM)
    \item Eficiencia mayor del 90\%
    \item Baja disipación térmica
    \item Ecuación de modulación PWM:
    \begin{equation}
        D = \frac{t_{on}}{T}
    \end{equation}
    
    donde $D$ es el ciclo de trabajo
\end{itemize}
\cite{3}

\subsection{Amplificadores Diferenciales}

Los amplificadores diferenciales son circuitos electrónicos especializados en amplificar exclusivamente la diferencia entre dos señales de entrada ($V_1$ y $V_2$), mientras rechazan eficientemente cualquier señal común a ambas entradas. La relación fundamental que describe su operación está dada por:

\begin{equation}
V_{out} = A_d(V_1 - V_2)
\end{equation}

donde $A_d$ representa la ganancia diferencial del amplificador. Una de sus características más importantes es el \textit{Common-Mode Rejection Ratio} (CMRR), que se expresa como:

\begin{equation}
\text{CMRR} = 20\log_{10}\left(\frac{A_d}{A_{cm}}\right) \quad [\text{dB}]
\end{equation}

siendo $A_{cm}$ la ganancia en modo común. Estos circuitos típicamente emplean:

\begin{itemize}
\item Un par de transistores cuidadosamente apareados (BJT o MOSFET)
\item Una fuente de corriente constante para polarización
\item Resistencias de carga precisas
\end{itemize}

\subsubsection{Aplicaciones Principales}
Gracias a su capacidad para minimizar ruido e interferencias, estos amplificadores son componentes esenciales en:

\begin{itemize}
\item La etapa de entrada de amplificadores operacionales
\item Sistemas de conversión analógico-digital (ADC)
\item Equipos de audio profesional de alta fidelidad
\item Instrumentación médica y equipos de diagnóstico
\end{itemize}

Su diseño versátil y alto rendimiento los hace indispensables en aplicaciones donde se requiere precisión y rechazo de señales parásitas.
\cite{4}

\subsection{Respuesta en Frecuencia}

La respuesta en frecuencia es un concepto fundamental en la ingeniería electrónica y se refiere a la manera en que un sistema electrónico responde a diferentes frecuencias de señal. En particular, en el caso de un amplificador, la respuesta en frecuencia se refiere a cómo el amplificador afecta la amplitud y la fase de la señal de entrada a diferentes frecuencias.

La respuesta en frecuencia de un amplificador se puede representar mediante una curva de respuesta en frecuencia, que muestra la amplitud de la señal de salida en función de la frecuencia de la señal de entrada. Esta curva puede mostrar las frecuencias en las que el amplificador tiene una ganancia máxima, así como las frecuencias en las que la ganancia es menor.

Es importante comprender la respuesta en frecuencia de un amplificador porque puede afectar la calidad de la señal que se está amplificando. Si un amplificador tiene una respuesta en frecuencia deficiente, puede introducir distorsión en la señal de salida, lo que puede hacer que la señal sea difícil de interpretar o incluso inútil.

Por lo tanto, es importante diseñar amplificadores con una respuesta en frecuencia adecuada y medir la respuesta en frecuencia de un amplificador existente para asegurarse de que está funcionando correctamente.

En el caso de un amplificador, la respuesta en frecuencia puede afectar la calidad de la señal  amplificada y es importante diseñar y medir la respuesta en frecuencia de un amplificador para asegurarse de que está funcionando correctamente.\cite{5}

\subsection{Sistemas Realimentados}

En un sistema realimentado, se toma una muestra de la señal de salida y se aplica a la entrada a través de una red adecuada. En la ilustración \ref{fig:1} se muestra un amplificador cuya ganancia sin realimentación es $A \angle \alpha$ y una red (generalmente pasiva) con relación de transferencia $B \angle \beta$ que retorna parte de la señal de salida a la entrada. Ambos parámetros dependen de la frecuencia.

\begin{ilustracion}[H]
    \centering
    \includegraphics[width=0.5\linewidth]{IMAGENES - Marco_Teorico/M_T_8.PNG}
    \caption{Sistemas Realimentados}
    \label{fig:1}
\end{ilustracion}

\begin{equation}
s_0 = s_c \cdot A \angle \alpha = (s_1 + s_r) A \angle \alpha
\quad
\end{equation}

\subsubsection{Casos Notables de Realimentación}

\begin{enumerate}

\item \textbf{Realimentación Positiva}:

Cuando $\alpha + \beta = 0^\circ$, la señal realimentada está en fase con la señal de entrada. 

En este caso:

\begin{itemize}
    \item $|A_r| > |A|$
    \item Si $A \cdot B = 1$, la amplificación tiende a infinito y el circuito oscila
\end{itemize}

\item \textbf{Realimentación Negativa}:

Cuando $\alpha + \beta = 180^\circ$, la señal realimentada está en oposición de fase:

\begin{itemize}
\item $|A_r| < |A|$
\item La ganancia con realimentación viene dada por:
\begin{equation}
A_r = \frac{s_o}{s_i} = \frac{A}{1 + A \cdot B}
\quad \label{Ec.1}
\end{equation}
\end{itemize}
\end{enumerate}

\footnotetext[1]{La ecuación (\ref{Ec.1}) es válida cuando se cumple la condición de fase $\alpha + \beta = 180^\circ$} \cite{6}



\newpage

    %---OBJETIVOS---

\section{Objetivos}

    \subsection{Objetivos Generales}

        \begin{itemize}

            \item Analizar el comportamiento de un Amplificador de Potencia.
            \item Analizar el comportamiento de un Amplificador Diferencial.
            \item Analizar un Amplificador de múltiples etapas, acoplado capacitivamente.
            \item Analizar el comportamiento en función de la frecuencia de un amplificador multietapas. 
            \item Analizar el efecto de la realimentación en el comportamiento de un amplificador.
            
        \end{itemize}

    \subsection{Objetivos Específicos}

        \begin{itemize}

            %-----Parte01-----
            \item Caracterizar, en dinámico, etapas amplificadoras de potencia B y AB. 
            \item Calcular y medir la polarizacion de los Amplificadores de potencia B y AB.
            \item Determinar las impedancias de entrada y salida de las distintas etapas.
            \item Reconocer las distorsiones generadas por las etapas de simetría complementaria.
            %-----Parte02-----
            \item Modelar circuitalmente etapas amplificadoras diferenciales, en dinámico y en términos de cada modo, común y diferencial.
            \item Determinar la polarización de un Amplificador diferencial y verificarla experimentalmente.
            \item Distinguir entre una señal de entrada modo común y modo diferencial.
            %-----Parte03-----
            \item Analizar un amplificador de múltiples etapas, acoplado capacitivamente de acuerdo a: Ganancia del amplificador, Impedancia de entrada y salida. Tensión de alimentación.
            \item Construir el amplificador con los transistores existentes comercialmente y con las especificaciones y curvas suministradas por el fabricante.
            \item Caracterizar el amplificador multietapas con los valores obtenidos experimental y teóricamente.
            %-----parte04-----
            \item Definir el modelo de banda ancha para amplificadores.
            \item Analizar los efectos de las reactancias en la respuesta en frecuencia de un amplificador y discriminar aquellas que actúan en la región de bajas frecuencias de
            la que actúan en altas frecuencias.
            %-----parte05-----
            \item Reconocer que la realimentación negativa reduce la ganancia y a cambio ofrece:  

            \begin{itemize}
            
                \item Mayor independencia de la ganancia del valor de la ganancia del amplificador base.  
                \item Disminución de la impedancia de salida.  
                \item Aumento de la impedancia de entrada.  
                \item Aumento del ancho de banda.  
                \item Mejoramiento de la linealidad del amplificador.  

            \end{itemize}
            
        \end{itemize}
\newpage

    %-----EQUIPOS E INSTRUMENTOS-----

\section{Equipos e Instrumentos}

    \begin{table}[H]
        \centering
        \caption{Equipos e instrumentos}
         \label{tab:instrumentos}
        \begin{tabular}{|c|c|c|} 
        \hline
            Equipo                    &  Marca &    Modelo   \\ \hline
            Osciloscopio              &  UNI-T & UTD2102CEX+ \\
            Generador de Onda         &  UNI-T & UTG932E     \\
            Fuente de alimentacion DC &  UNI-T & UTP3305-II  \\
            Protoboard                &  Miyako& EIC-108     \\
             \hline
        \end{tabular}
    \end{table}

    %-----COMPONENTES Y MATERIALES-----

\section{Componentes y materiales}

    \begin{table}[H]
        \centering
        \caption{Componentes y materiales de la etapa de potencia}
         \label{tab:componentes-01}
        \begin{tabular}{|c|c|c|} \hline
            Referecia    &         Descripcion                 &    Especificaciones   \\ \hline
            R17,R12      &  Resistencia de película de carbón  &     22k$\Omega$,1/4 W         \\
            R13,R14      &  Resistencia de película de carbón  &     10 $\Omega$,1/4 W         \\
            Q4           &         BJT de tipo NPN             &     2N2222                    \\
            Q5           &         BJT de tipo NPN             &     BC238                     \\
            Q6           &         BJT de tipo PNP             &     BC238                     \\
            CS           &  Capacitor de Poliéster             &     470nF, 100V                \\
            U1           &  Amplificador operacional           &     MC1741CP1             \\\hline
        \end{tabular}
    \end{table}

    \begin{table}[H]
        \centering
        \caption{Componentes y materiales de la etapa diferencial}
         \label{tab:componentes-02}
        \begin{tabular}{|c|c|c|} \hline
            Referecia    &         Descripcion                 &    Especificaciones   \\ \hline
            R1,R2,R8,R9  &  Resistencia de película de carbón  &     100k$\Omega$,1/4 W         \\
            R3,R6        &  Resistencia de película de carbón  &     4,7K$\Omega$,1/4 W         \\
            R4,R7        &  Resistencia de película de carbón  &     15K$\Omega$,1/4 W         \\
            R5           &  Resistencia de película de carbón  &     1,5K$\Omega$,1/4 W         \\
            Q1           &         BJT de tipo NPN             &     2N2222                    \\
            Q2           &         BJT de tipo NPN             &     2N2222                    \\
            C1, C3       &  Capacitor Electrolítico            &     10$\mu$F                  \\
            \hline
        \end{tabular}
    \end{table} 
    
    \begin{table}[H]
        \centering
        \caption{Componentes y materiales de la etapa impulsora}
         \label{tab:componentes-02}
        \begin{tabular}{|c|c|c|} \hline
            Referecia    &         Descripcion                 &    Especificaciones   \\ \hline
            R10          &  Resistencia de película de carbón  &     220k$\Omega$,1/4 W         \\
            R11          &  Resistencia de película de carbón  &     680$\Omega$,1/4 W         \\
            R16          &  Resistencia de película de carbón  &     6,8K$\Omega$,1/4 W         \\
            R15          &  Resistencia de película de carbón  &     33K$\Omega$,1/4 W         \\
            Q3           &         BJT de tipo PNP             &     BC307                    \\
            C2           &  Capacitor Electrolítico            &     1$\mu$F                  \\
            C4           &  Capacitor de Poliéster             &     22pF                  \\
            C6           &  Capacitor Electrolítico            &     100$\mu$F                  \\
            \hline
        \end{tabular}
    \end{table} 
\newpage

    %-----METODOLOGIA-----

\section{Metodología}

\begin{figure}[H]
    \centering
    \includegraphics[width=1.0\linewidth]{IMAGENES - PRE1/amplificador base.png}
    \caption{Amplificador Base}
    \label{fig:enter-label}
\end{figure}

    \subsection{Trabajo de Preparación}

        \subsubsection{Amplificador de Potencia}
        \begin{enumerate}
    \item Identifique la etapa de Potencia (EP) en el amplificador base (anexo), si es necesario consulte a su instructor.

    \begin{figure}[H]
        \centering
        \includegraphics[width=0.5\linewidth]{IMAGENES - PRE1/EP.PNG}
        \caption{Etapa de Potencia del amplificador base}
        \label{fig:enter-label}
    \end{figure} 

    \item Determinar los puntos de reposo de todos los transistores en la EP.

    Para determinar los puntos de operación de los transistores en la etapa de potencia, se realizó un análisis en DC, sin considerar los condensadores C5 y C7. 

    Se comenzó por estudiar al transistor Q4. Este transistor para el caso de la etapa de potencia cumple la función de generar el voltaje necesario para que los transistores Q5 y Q6 conduzcan, asumiendo que la corriente que pasa por la base del transistor Q5 es muy pequeña, se puede decir por LCK:

     \begin{equation}
        I_{R17} = I_{RV1} + I_{CQ4}\label{ecuacion1}
    \end{equation}

    Donde:

    \begin{itemize}
        \item \( I_{CQ4} \): Corriente de colector en el transistor Q4.
        \item \( I_{RV1} \): Corriente en el potenciómetro RV1.
        \item \( I_{R17} \): Corriente en la resistencia R17.
    \end{itemize}

    Si se asume que la corriente en la base de Q4 se puede despreciar por ser muy pequeña, se tiene que la corriente que pasa por el potenciómetro es:

    \begin{equation}
        I_{RV1} = \frac{V_{BE}}{X(RV1)}\label{ecuacion2}
    \end{equation}

    Donde:

    \begin{itemize}
        \item $I_{RV1}$: Corriente en el potenciómetro.
        \item $V_{BEQ4}$: Tensión base-emisor del transistor Q4.
        \item $RV1$: Potenciómetro.
    \end{itemize}
    
    Por LVK se tiene que:

    \begin{equation}
        V_{CEQ4} = I_{RV1}(RV1)\label{ecuacion3}
    \end{equation}

    Sustituyendo la Ec. \ref{ecuacion2}  en la Ec. \ref{ecuacion3} se tiene que:  

    \begin{equation}
        V_{CE} = \frac{V_{BE}}{X(RV1)}(RV1)
    \end{equation}

    Simplicando:

    \begin{equation}
        V_{CEQ4} = \frac{V_{BE}}{X}\label{ecuacion4}
    \end{equation}

    Como X solo puede adquirir valores entre 0 y 1 para que los transistores Q5 y Q6 conduzcan es necesario que el voltaje Colector – Emisor sea 2 veces el voltaje Base – Emisor por lo que se puede decir que X debería ser 0,5.

    Para calcular la corriente que pasa por el transistor Q4 se despreciaron las corrientes de base de Q5 y Q6 y usando LVK se tiene:

    \begin{equation}
        20 - (R_{17} + R_{12})I_{R17} - 2V_{BE} = 0  
    \end{equation}

    Despejando la corriente que pasa por R17:

    \begin{equation}
        I_{R17} = \frac{20 - 2V_{BE}}{R_{17} + R_{12}}\label{ecuacion5}
    \end{equation}

    Reemplazando  la Ec.\ref{ecuacion5} y la Ec. \ref{ecuacion2} en la Ec. \ref{ecuacion1}  se despeja la corriente de colector de Q4:

    \begin{equation}
        V_{CQ4} = \frac{20 - 2V_{BE}}{R_{17} + R_{12}} - \frac{V_{BE}}{X}\label{ecuacion6}
    \end{equation}

    Usando:
    \begin{itemize}
        \item $X = 0,5$
        \item $V_{BE} = 0,62V.$
        \item $RV1 = 10K$$\Omega$$.$
        \item $R_{12} = R_{17} = 22K$$\Omega$$.$ 
    \end{itemize}
    
    Se tiene que:
    
    $$ I_{CQ4} = 302.36\mu\text{A} $$
    
    Para calcular el punto de operaciones de Q5 y Q6 se usó la LVK en la malla que tiene los 3 transistores sin tomar en cuenta las corrientes de base por ser muy pequeñas:
    
        \begin{equation}
            V_{BEQ5} + 10I_{EQ5} + 10I_{EQ6} - V_{EBQ6} + V_{ECQ4} = 0 \label{ecuacion7}
        \end{equation}
    
    Donde se tomó en consideración que: $ I_{EQ5} \approx I_{EQ6}; V_{CEQ4} = 2V_{BEQ4} $

    Despejando $I_{E}$ 
    
    \begin{equation}
        I_E = \frac{V_{CEQ4} - V_{BEQ5} - V_{BEQ6}}{R_{13} + R_{14}}
    \end{equation}

    Sabiendo que \( V_{BEQ5} = 0{,}62\,V \) y \( V_{BEQ6} = 0{,}614\,V \), se obtiene:

    \[
    \therefore I_E = 0{,}3\,mA
    \]
    
    Por lo tanto, la corriente de colector para los transistores Q5 y Q6 se obtendrá de la siguiente relación: 

    \begin{equation}
        I_C = I_E \left( \frac{\beta}{1 + \beta} \right)
        \label{eq:corriente_colector}
    \end{equation}

    Para Q5 se tiene un \(\beta_{Q5} = 210\).\par
    Para Q6 se tiene un \(\beta_{Q6} = 240\).
    

    \[I_{CQ5} = 298{,}57 \, \mu\mathrm{A}\]
    
    \[I_{CQ6} = 298{,}75 \, \mu\mathrm{A}\]
    
    Se asume que \(V_{CEQ5} \approx -V_{CEQ6} = V_{CE}\), aplicando LVK tenemos:
    
    \[2V_{CE} + I_C (R_{13} + R_{14}) - 20 = 0\]
    
    Despejando \(V_{CE}\):
    
    \[V_{CE} = \frac{20 - I_C (R_{13} + R_{14})}{2}\]
    
    \[\therefore V_{CE} = 10\,\mathrm{V}\]
    
    \begin{table}[H]
    \centering
    \caption{Puntos estáticos de operación del amplificador de potencia}
    \label{tab:operacion}
    \begin{tabular}{|c|c|c|}
    \hline
    \textbf{Transistor} &       \(\bm{I_C}\)          & \(\bm{V_{CE}}\)         \\ \hline
    Q4                  & \(302{,}36\,\mu\mathrm{A}\) & \(1{,}24\,\mathrm{V}\)  \\ \hline
    Q5                  & \(298{,}57\,\mu\mathrm{A}\) & \(9{,}99\,\mathrm{V}\)  \\ \hline
    Q6                  & \(298{,}75\,\mu\mathrm{A}\) & \(-9{,}99\,\mathrm{V}\) \\ \hline
    \end{tabular}
    \end{table}
        
    \item Determine el modelo dinámico a pequeña señal y a frecuencias medias de la EP, ganancia e impedancias de entrada y salida.

    \begin{figure}[H]
        \centering
        \includegraphics[width=0.5\linewidth]{IMAGENES - PRE1/Modelo hibrido.png}
        \caption{Modelo híbrido}
        \label{fig:enter-label}
    \end{figure}

    Las ecuaciones a usar para obtener los parámetros serán los siguientes:

    \begin{equation}
        g_{m} = \frac{I_{C}}{V_{T}} \label{ec_gm}    
    \end{equation}
    
    \begin{equation}
        r_{\pi} = \frac{\beta}{g_{m}} \label{ec_rpi}   
    \end{equation}
    
    Donde $V_T$ es el voltaje térmico que a temperatura ambiente se tomó de 25,86mV aproximadamente, $\beta$ será la ganancia de corriente DC del transistor y $ R_{o} = \infty$.

    \begin{equation}
        h_{fe} = \beta \qquad
    \end{equation}

    \begin{equation}
        h_{ie} = r_\pi
    \end{equation}

    Asi se obtuvo que:
    
    \begin{table}[H]
    \centering
    \caption{Parámetros híbridos de los transistores en la etapa de potencia}
    \label{tab:hibridos}
    \begin{tabular}{|c|c|c|c|}
    \hline
    \textbf{Transistor} & \(\bm{h_{fe}}\) &     \(\bm{g_m}\)         & \(\bm{r_\pi}\)               \\ \hline
            Q4          &       210       & \(11{,}69\,\mathrm{mS}\) & \(17{,}96\,\mathrm{k\Omega}\) \\ \hline
            Q5          &       210       & \(11{,}54\,\mathrm{mS}\) & \(18{,}19\,\mathrm{k\Omega}\) \\ \hline
            Q6          &       240       & \(11{,}55\,\mathrm{mS}\) & \(20{,}77\,\mathrm{k\Omega}\) \\ \hline
    \end{tabular}
    \end{table}

    \underline{\textbf{Ganancia}}

    \begin{equation}
        \frac{V_{out}}{V_{in}} = \frac{(R_L + R_{13})(1 + g_{m5}r_{\pi5})}
                                {r_{\pi5} + (R_L + R_{13})(1 + g_{m5}r_{\pi5})} 
    \end{equation}
       
    Al sustituir los valores se tiene que:
    
    $$\frac{V_{out}}{V_{in}} = 0.96 \approx 1 $$
    
    \underline{\textbf{Impedancia de entrada}}
    
    \begin{equation}
        Z_{in} = R_{17}|| R_{12}||[r_{\pi} + (1 + g_mr_{\pi})(R_L + R_{13})]   
    \end{equation}
    
    $$ Z_{in} = 10,75 k\Omega \approx 10 k\Omega$$
    
    \underline{\textbf{Impedancia de salida}}
    
    \begin{equation}
        Z_o = R_{13} + \frac{r_{\pi} + R_{17}||R_{12}}{(1 + g_mr_{\pi})}
    \end{equation}
    
    $$ Z_o \approx R_{13} \approx 10\Omega $$

    \item Realice la simulación de la etapa con el fin de verificar tanto los puntos de operación, como el modelo circuital a pequeña señal. Explique cualquier diferencia respecto a sus cálculos, si la hay.

    \begin{ilustracion}[H]
        \centering
        \includegraphics[width=0.7\linewidth]{IMAGENES - PRE1/S1_Puntos_de_Operacion.PNG}
        \caption{Caption}
        \label{fig:enter-label}
    \end{ilustracion}

    \begin{ilustracion}[H]
        \centering
        \includegraphics[width=0.7\linewidth]{IMAGENES - PRE1/S2_A.PNG}
        \caption{Ganancia en la etapa de potencia}
        \label{fig:enter-label}
    \end{ilustracion}

    \begin{ilustracion}[H]
        \centering
        \includegraphics[width=0.7\linewidth]{IMAGENES - PRE1/S3_Zin.PNG}
        \caption{Impedancia de entrada en la etapa de potencia}
        \label{fig:enter-label}
    \end{ilustracion}

    \begin{ilustracion}[H]
        \centering
        \includegraphics[width=0.7\linewidth]{IMAGENES - PRE1/S4_Zo_EP.PNG}
        \caption{Impedancia de salida en la etapa de potencia}
        \label{fig:enter-label}
    \end{ilustracion}
    
\end{enumerate}
\newpage

    %-----RESULTADOS-----

\section{Presentación de Resultados}

\subsection{Parte 1. Amplificador de Potencia}

Se midieron los puntos de operación de todos los elementos activos

\begin{table}[H]
    \centering
    \caption{Mediciones para hallar los puntos estáticos de operación}
    \label{tab:puntos-operacion}
    \begin{tabular}{|c|c|c|c|c|c|c|c|c|c|c|}
        \hline
        \textbf{Transistor} & \textbf{VC [V]} & \textbf{AVC [V]} & \textbf{VE [V]} & \textbf{AVB [V]} & \textbf{VE [V]} & \textbf{AVE [V]} & \textbf{VCE [V]} & \textbf{IC [A]} \\ \hline
        Q4                  & 7.00e-01        & 1.00e-01         & -1.00e-01       & 1.00e-02         & -6.00e-01       & 4.00e-02         & 1.30             & 0.000329        \\
        Q5                  & 10              & 2.00             & 4.00e-01        & 4.00e-02         & -1.00e-01       & 2.00e-02         & 10               & 0.000500        \\
        Q6                  & -10             & 2.00             & -6.00e-01       & 4.00e-02         & -1.10e-01       & 1.00e-02         & -9.89            & 0.000500        \\
        \hline
    \end{tabular}
\end{table}


\newpage

    %-----ANALISIS DE RESULTADOS-----

\input{Analisis de Resultados}
\newpage

    %-----CONCLUSIONES-----

\input{Conclusiones}
\newpage

    %-----REFERENCIAS-----

\begin{thebibliography}{200}

\bibitem{1} 

    Madrid, G. V., \& Izquierdo, M. A. Z. "TRANSISTORES DE UNION BIPOLAR (BJT)", [En línea]. Disponible: https://www.academia.edu/download/53490536/BJT.pdf. [Accedido: 25-05-2025].  

\bibitem{2} 

    Unicrom. (2023). Amplificadores \& Amplificación, [En línea]. Disponible: {https://unicrom.com/amplificadores-amplificacion/. [Accedido: 26-05-2025]. 

\bibitem{3}

    Amplificadores de Potencia, Amplificadores.info, [En línea]. Disponible: https://amplificadores.info/amplificadores-de-potencia. [Accedido: 4-06-2025].  

\bibitem{4}

    Amplificadores diferenciales, Electricity-Magnetism.org, [En línea]. Disponible: https://www.electricity-magnetism.org/es/amplificadores-diferenciales/. [Accedido: 15-05-2025].

\bibitem{5}

    "¿Qué es la respuesta en frecuencia de un amplificador?", PCHardwarePro.com, [En línea]. Disponible: https://www.pchardwarepro.com/que-es-la-respuesta-en-frecuencia-de-un-amplificador/. [Accedido: 1-06-2025].

\bibitem{6}

    F. Ubiría y P. Castro, "9. Realimentación", Universidad Tecnológica del Uruguay (UTU), 2024. [En línea]. Disponible: https://its.utu.edu.uy/wp-content/uploads/sites/33/2024/04/cap_9.pdf.
    [Accedido: 05-06-2025]

%\bibitem{libro10} 
\end{thebibliography}

    %-----APENDICE-----

\input{Anexos}
\newpage

    %---ANEXOS---

\input{Anexos}


\end{document}
