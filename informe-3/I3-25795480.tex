\documentclass{article}
\usepackage{cancel}
\usepackage{xr}
\usepackage{graphicx} % Para incluir imágenes
\usepackage[spanish]{babel}
\usepackage{makeidx}
\usepackage{geometry}
\usepackage{amsmath}
\usepackage{graphicx}
\usepackage{caption}
\usepackage{pdfpages}
\usepackage{hyperref}
\usepackage{placeins}
\usepackage{adjustbox}
\usepackage{multicol}
\usepackage{../utils/laboratorio}


\hypersetup{
    colorlinks,
    citecolor=black,
    filecolor=black,
    linkcolor=black,
    urlcolor=black
}

\geometry{a4paper, total={170mm,257mm}, left=20mm, top=20mm}
\renewcommand{\familydefault}{\sfdefault}


\graphicspath{
    {./parte-1/src/images/}
    {./parte-2/src/images/}
    {./parte-3/src/images/}
    {./parte-4/src/images/}
}


\begin{document}
\begin{titlepage}
    \centering
    \Large
    Universidad Central de Venezuela\\
    Facultad de Ingeniería\\
    Escuela de Ingeniería Eléctrica
    \vspace*{8cm}

    \Huge
    \textbf{Prelaboratorio Nº 4:} 

    \textbf{Respuesta en frecuencia}
    \vfill


    \Large

    Emerson Warhman \\
    C.I. 25.795.480 \\
    \today

\end{titlepage}
\tableofcontents
\newpage
\section{Resumen}
Este informe experimental analizó tres configuraciones no lineales fundamentales de amplificadores operacionales:

\subsection{1. Oscilador de Puente de Wien}
\begin{itemize}
\item Genera señales sinusoidales mediante realimentación positiva
\item Frecuencia de oscilación: $f_o = \frac{1}{2\pi RC}$
\item Se verificó la condición de oscilación ($\beta A_v \geq 1$) y se midió distorsión armónica
\item Aplicación: Generación de señales de prueba en sistemas de audio
\end{itemize}

\subsection{2. Multivibradores}
\begin{itemize}
\item \textbf{Astable}: Generador de onda cuadrada autónomo
  \begin{itemize}
  \item Período medido: $T = 2R_3C_1\ln(1+\frac{2R_1}{R_2})$
  \item Error experimental vs teórico: 4.55\%
  \end{itemize}
  
\item \textbf{Monoestable}: Produce pulsos de ancho controlado
  \begin{itemize}
  \item Ancho de pulso: $t_w = R_3C_1\ln(3)$
  \item Se comprobó la dependencia con $R_3C_1$ (error: 4\%)
  \end{itemize}
\end{itemize}

\subsection{3. Generador de Funciones}
\begin{itemize}
\item Circuito integrado para producir señales triangulares/cuadradas
\item Se caracterizó la relación frecuencia-capacitancia ($f \propto 1/C$)
\item No linealidades observadas en amplitudes altas
\end{itemize}


\section{Introducción}
Los amplificadores operacionales (op-amps) son dispositivos versátiles que encuentran aplicaciones tanto en circuitos lineales como no lineales. Mientras que en configuraciones lineales se utilizan principalmente para operaciones de amplificación y filtrado, sus aplicaciones no lineales abren un panorama de posibilidades en generación y conformación de señales. Este informe explora tres configuraciones fundamentales de op-amps en régimen no lineal: el oscilador de puente de Wien, los multivibradores (astable y monoestable) y los generadores de funciones.

El \textbf{oscilador de puente de Wien} representa una aplicación clásica en generación de señales sinusoidales. Su diseño aprovecha el balance entre una red de realimentación positiva (para mantener las oscilaciones) y un mecanismo de control de amplitud (usualmente mediante diodos o JFETs) que garantiza estabilidad en la señal de salida. La frecuencia de oscilación está determinada por la relación $f_o = \frac{1}{2\pi RC}$, demostrando cómo componentes pasivos simples pueden definir características fundamentales de la señal generada.

Los \textbf{multivibradores} ilustran la capacidad de los op-amps para trabajar en conmutación:
\begin{itemize}
    \item El \textbf{astable} opera como oscilador libre, generando una onda cuadrada continua cuya frecuencia depende de los elementos RC en el circuito
    \item El \textbf{monoestable} produce un pulso de duración fija ante un disparo externo, siendo útil en aplicaciones de temporización
\end{itemize}

Los \textbf{generadores de funciones} amplían estas capacidades para producir formas de onda complejas (triangulares, diente de sierra, etc.), a menudo mediante la integración controlada de señales cuadradas. Estos circuitos encuentran aplicaciones prácticas en sistemas de comunicaciones, instrumentación médica y equipos de prueba electrónicos.

Este estudio experimental permitirá verificar los principios teóricos de operación, analizar las relaciones entre los componentes y los parámetros de salida, y evaluar las limitaciones prácticas de estos circuitos. Particular atención se dedicará al análisis de distorsión armónica en el oscilador Wien y a la precisión temporal en los multivibradores, aspectos críticos para aplicaciones reales.



\section{Objetivos}
\textbf{Objetivo General}

\begin{itemize}
    \item Reconocer las ventajas del uso del concepto de amplificadores operacionales en el diseño e implementación de sistemas analógicos.
\end{itemize}

\textbf{Objetivos Específicos}

\begin{itemize}
    \item Reconocer los efectos que produce la aplicación de filtros pasa bajos, pasa banda y pasa altos en distintas señales.
\end{itemize}


\section {Marco teórico}
\subsection{Generador de funciones}

La onda exponencial generada en un circuito astable puede ser cambiada a un una onda triangular reemplazando el circuito RC con un integrador cómo se muestra en la ilustración . El integrador ocaciona que el capacitor se cargue y descargue de manera lineal, obteniendo de esta forma una onda triangular. \cite[pag. ~1366]{sedra-smith}

\begin{ilustracion}[ht]
    \centering
    \includegraphics[width=0.5\textwidth]{marco-teorico/generador-funciones.png}
    \caption{Generador de funciones}
\end{ilustracion}

Supongamos que en la salida $V_{SQ}$ del circuito tenemos valores máximos $V_{SQ+}$ y mínimos $V_{SQ-}$, Cuando el valor de la salida es $V_{SQ+}$ una corriente es $V_{SQ+}/R$ va a pasar a traves de la resistencia y del condensador, causando que en la salida del integrador decrezca linealmente con una pendiente $-V_{SQ+}/RC$, Esto va a ocurrir hasta que la salida del integrador alcance el límite inferior del circuito astable, punto en el cual es circuito astable cambiará de estado, volviendose la salida del astable igual a $V_{SQ-}$. En este momento la corriente a traves de R y C cambiará de dirección y su valor se volverá $-V_{SQ-}/R$, causando que la salida del integrador aumente linealmente con una pendiente $V_{SQ-}/RC$ hasta que alcance el límite superior del circuito astable, punto en el cual el circuito astable cambiará de estado, volviendose la salida del astable igual a $V_{SQ+}$, una vez alcanzado este punto el circuito cambiará de estado nuevamente, haciendo que el voltaje en su salida sea $V_{SQ+}$ y repitiendo el ciclo.

De lo dicho anteriormente se puede deducir una expresión para el periodo $T$ de la onda triangular y la onda cuadrada. Durante el intervalo $T_1$ tenemos

\begin{equation*}
    \frac{V_{TH} - V_{TL}}{T_1} = \frac{V_{SQ+}}{RC}
\end{equation*}

de donde podemos despejar $T_1$

\begin{equation}
    T_1 = \frac{V_{TH} - V_{TL}}{V_{SQ+}}RC
    \label{eq:t1}
\end{equation}

De manera similar, durante $T_2$ tenemos

\begin{equation*}
    \frac{V_{TH} - V_{TL}}{T_2} = \frac{-V_{SQ-}}{RC}
\end{equation*}

de donde podemos despejar $T_2$

\begin{equation}
    T_2 = \frac{V_{TH} - V_{TL}}{-V_{SQ-}}RC
\end{equation}
\subsection{Generador de funciones}

La onda exponencial generada en un circuito astable puede ser cambiada a un una onda triangular reemplazando el circuito RC con un integrador cómo se muestra en la ilustración . El integrador ocaciona que el capacitor se cargue y descargue de manera lineal, obteniendo de esta forma una onda triangular. \cite[pag. ~1366]{sedra-smith}

\begin{ilustracion}[ht]
    \centering
    \includegraphics[width=0.5\textwidth]{marco-teorico/generador-funciones.png}
    \caption{Generador de funciones}
\end{ilustracion}

Supongamos que en la salida $V_{SQ}$ del circuito tenemos valores máximos $V_{SQ+}$ y mínimos $V_{SQ-}$, Cuando el valor de la salida es $V_{SQ+}$ una corriente es $V_{SQ+}/R$ va a pasar a traves de la resistencia y del condensador, causando que en la salida del integrador decrezca linealmente con una pendiente $-V_{SQ+}/RC$, Esto va a ocurrir hasta que la salida del integrador alcance el límite inferior del circuito astable, punto en el cual es circuito astable cambiará de estado, volviendose la salida del astable igual a $V_{SQ-}$. En este momento la corriente a traves de R y C cambiará de dirección y su valor se volverá $-V_{SQ-}/R$, causando que la salida del integrador aumente linealmente con una pendiente $V_{SQ-}/RC$ hasta que alcance el límite superior del circuito astable, punto en el cual el circuito astable cambiará de estado, volviendose la salida del astable igual a $V_{SQ+}$, una vez alcanzado este punto el circuito cambiará de estado nuevamente, haciendo que el voltaje en su salida sea $V_{SQ+}$ y repitiendo el ciclo.

De lo dicho anteriormente se puede deducir una expresión para el periodo $T$ de la onda triangular y la onda cuadrada. Durante el intervalo $T_1$ tenemos

\begin{equation*}
    \frac{V_{TH} - V_{TL}}{T_1} = \frac{V_{SQ+}}{RC}
\end{equation*}

de donde podemos despejar $T_1$

\begin{equation}
    T_1 = \frac{V_{TH} - V_{TL}}{V_{SQ+}}RC
    \label{eq:t1}
\end{equation}

De manera similar, durante $T_2$ tenemos

\begin{equation*}
    \frac{V_{TH} - V_{TL}}{T_2} = \frac{-V_{SQ-}}{RC}
\end{equation*}

de donde podemos despejar $T_2$

\begin{equation}
    T_2 = \frac{V_{TH} - V_{TL}}{-V_{SQ-}}RC
\end{equation}
\subsection{Generador de funciones}

La onda exponencial generada en un circuito astable puede ser cambiada a un una onda triangular reemplazando el circuito RC con un integrador cómo se muestra en la ilustración . El integrador ocaciona que el capacitor se cargue y descargue de manera lineal, obteniendo de esta forma una onda triangular. \cite[pag. ~1366]{sedra-smith}

\begin{ilustracion}[ht]
    \centering
    \includegraphics[width=0.5\textwidth]{marco-teorico/generador-funciones.png}
    \caption{Generador de funciones}
\end{ilustracion}

Supongamos que en la salida $V_{SQ}$ del circuito tenemos valores máximos $V_{SQ+}$ y mínimos $V_{SQ-}$, Cuando el valor de la salida es $V_{SQ+}$ una corriente es $V_{SQ+}/R$ va a pasar a traves de la resistencia y del condensador, causando que en la salida del integrador decrezca linealmente con una pendiente $-V_{SQ+}/RC$, Esto va a ocurrir hasta que la salida del integrador alcance el límite inferior del circuito astable, punto en el cual es circuito astable cambiará de estado, volviendose la salida del astable igual a $V_{SQ-}$. En este momento la corriente a traves de R y C cambiará de dirección y su valor se volverá $-V_{SQ-}/R$, causando que la salida del integrador aumente linealmente con una pendiente $V_{SQ-}/RC$ hasta que alcance el límite superior del circuito astable, punto en el cual el circuito astable cambiará de estado, volviendose la salida del astable igual a $V_{SQ+}$, una vez alcanzado este punto el circuito cambiará de estado nuevamente, haciendo que el voltaje en su salida sea $V_{SQ+}$ y repitiendo el ciclo.

De lo dicho anteriormente se puede deducir una expresión para el periodo $T$ de la onda triangular y la onda cuadrada. Durante el intervalo $T_1$ tenemos

\begin{equation*}
    \frac{V_{TH} - V_{TL}}{T_1} = \frac{V_{SQ+}}{RC}
\end{equation*}

de donde podemos despejar $T_1$

\begin{equation}
    T_1 = \frac{V_{TH} - V_{TL}}{V_{SQ+}}RC
    \label{eq:t1}
\end{equation}

De manera similar, durante $T_2$ tenemos

\begin{equation*}
    \frac{V_{TH} - V_{TL}}{T_2} = \frac{-V_{SQ-}}{RC}
\end{equation*}

de donde podemos despejar $T_2$

\begin{equation}
    T_2 = \frac{V_{TH} - V_{TL}}{-V_{SQ-}}RC
\end{equation}
\subsection{Generador de funciones}

La onda exponencial generada en un circuito astable puede ser cambiada a un una onda triangular reemplazando el circuito RC con un integrador cómo se muestra en la ilustración . El integrador ocaciona que el capacitor se cargue y descargue de manera lineal, obteniendo de esta forma una onda triangular. \cite[pag. ~1366]{sedra-smith}

\begin{ilustracion}[ht]
    \centering
    \includegraphics[width=0.5\textwidth]{marco-teorico/generador-funciones.png}
    \caption{Generador de funciones}
\end{ilustracion}

Supongamos que en la salida $V_{SQ}$ del circuito tenemos valores máximos $V_{SQ+}$ y mínimos $V_{SQ-}$, Cuando el valor de la salida es $V_{SQ+}$ una corriente es $V_{SQ+}/R$ va a pasar a traves de la resistencia y del condensador, causando que en la salida del integrador decrezca linealmente con una pendiente $-V_{SQ+}/RC$, Esto va a ocurrir hasta que la salida del integrador alcance el límite inferior del circuito astable, punto en el cual es circuito astable cambiará de estado, volviendose la salida del astable igual a $V_{SQ-}$. En este momento la corriente a traves de R y C cambiará de dirección y su valor se volverá $-V_{SQ-}/R$, causando que la salida del integrador aumente linealmente con una pendiente $V_{SQ-}/RC$ hasta que alcance el límite superior del circuito astable, punto en el cual el circuito astable cambiará de estado, volviendose la salida del astable igual a $V_{SQ+}$, una vez alcanzado este punto el circuito cambiará de estado nuevamente, haciendo que el voltaje en su salida sea $V_{SQ+}$ y repitiendo el ciclo.

De lo dicho anteriormente se puede deducir una expresión para el periodo $T$ de la onda triangular y la onda cuadrada. Durante el intervalo $T_1$ tenemos

\begin{equation*}
    \frac{V_{TH} - V_{TL}}{T_1} = \frac{V_{SQ+}}{RC}
\end{equation*}

de donde podemos despejar $T_1$

\begin{equation}
    T_1 = \frac{V_{TH} - V_{TL}}{V_{SQ+}}RC
    \label{eq:t1}
\end{equation}

De manera similar, durante $T_2$ tenemos

\begin{equation*}
    \frac{V_{TH} - V_{TL}}{T_2} = \frac{-V_{SQ-}}{RC}
\end{equation*}

de donde podemos despejar $T_2$

\begin{equation}
    T_2 = \frac{V_{TH} - V_{TL}}{-V_{SQ-}}RC
\end{equation}

\section{Metodología}
\subsection{Aplicaciones de las topologías clásicas}
\subsubsection{Regulador con tensión de salida fija}

\begin{figure}[ht]
    \centering
    \includegraphics[width=0.8\textwidth]{regulador-tension-salida-fija.png}
    \caption{Regulador lineal con tensión de salida fija}
    \label{fig:regulador-lineal-tension-fija}
\end{figure}

\subsubsection*{Explicar la función de los condensadores $C_2$ y $C_3$ en la figura \ref{fig:regulador-lineal-tension-fija}}

El dispositivo siempre debe estar equipado con un capacitor de entrada para reducir los efectos de la inductancia parásita en los cables de entrada, especialmente si el regulador está ubicado lejos de la fuente no regulada, y un capacitor de salida para ayudar a mejorar la respuesta a los cambios repentinos en la corriente de carga. Para obtener los mejores resultados, use cables y trazos gruesos, mantenga los cables cortos y monte ambos capacitores lo más cerca posible del regulador. Dependiendo del caso, puede ser necesario un disipador de calor para mantener la temperatura interna dentro de niveles tolerables.

\subsubsection*{Explicar cómo conectar el puente de diodos si el transformador no tiene toma central (CT).}

Si el transformador tiene toma central, se deja el jumper $\jumper{1}$ abierto, si no se tiene toma central, se cierra el jumper $\jumper{1}$ y se asegura que en ese nodo haya una referencia.

\subsubsection*{Suponiendo una carga de $80mA$ determine la tensión de rizado pico-pico que se va a presentar en $C_1$}.

Se tiene que el voltaje de rizo pico pico $V_{rpp}$ viene dada por la siguiente ecuación:

\begin{equation}
    V_{rpp} = \frac{I_{cd}}{2 f C}
\end{equation}

teniendo en cuenta que $f= 60Hz$ y $C= 470\mu F$  nos queda.

\begin{align*}
    V_{rpp} &= \frac{80m A}{2 \cdot 60Hz \cdot 470\mu F} \\
    V_{rpp} &= 1.42 V
\end{align*}

\subsubsection*{Determinar la tensión mínima del secundario del transformador en función de la corriente de salida, de manera que el regulador puede mantener la regulación.}

Según los datos del datasheet, la tensión de entrada mínima para mantener la regulación tiene que  ser de $ V_{ir} = 7.5V$. Ahora, para calcular la tensión en el secundario se debe calcular la caida de tensión tomando en cuenta los diodos, el voltaje de riso y el voltaje mínimo del regulador:

\begin{equation}
    V_{s} = 2 V_{d} + V_{rpp} + V_{ir}
    \label{eq:tension-min-secundario}
\end{equation}

$$ V_{s} = 10.32 V $$
 
O si se expresa en rms:

$$ V_{srms} = \frac{V_s}{\sqrt{2}} $$

$$ V_{srms} = 7.30 V $$

\subsubsection*{Determinar la regulación de voltaje que se va a obtener al colocar unas cargas de 100mA}

Recordando que la regulación de voltaje viene dada por:

\begin{equation}
    reg = \frac{V_{cc} - V_{sc}}{V_{sc}} 100 %
\end{equation}

Una carga de $100mA$ viene dada por:

$$ R = \frac{V}{I} = \frac{5V}{100mA} = 50 \Omega $$

Por tanto la regulación de voltaje será:

\begin{equation}
    reg = \frac{5 -5}{5} 100 = 0\%
\end{equation}

\subsubsection{Fuente regulada ajustable}

\begin{figure}[ht]
    \centering
    \includegraphics[width=0.8\textwidth]{regulador-tension-salida-ajustable.png}
    \caption{Fuente regulada ajustable}
    \label{fig:regulador-tension-salida-ajustable}
\end{figure}

\subsubsection*{Determinar el rango de tensiones de salida en función del accionamiento <<x>>}

sabemos que:

$$ I = \frac{5 V}{R_1}$$

$$ V_2 = I . x R_{v1}$$

y de la ecuación

\begin{equation}
    V_o = V_1 + I(V_2)
\end{equation}
 
tenemos 

\begin{align*}
    V_o &= 5V + \frac{5 V}{R_1} (x R_{v1}) \\
    V_o &= 5 \WrapParenthesis{1 + \frac{x R_{v1}}{R_{1}}} \\
\end{align*}

los valores de $x$ son $ 0 \leq x \leq 1$ por lo tanto.

\begin{equation}
    5 \leq V_o \leq 5 + \frac{R_{v1}}{R_{1}}
\end{equation}

\subsubsection*{Asignar el valor de $R_1$ de modo que la fuente suministre tensiones de hasta al menos 15V}

Partiendo de la expresión del valor máximo de $V_o$ podemos obtener el valor de $R_1$ que cumple con la condición:

\begin{align*}
    15V &= 5(1 + \frac{R_{v1}}{R_{1}}) \\
    3V &= 1 + \frac{R_{v1}}{R_{1}} \\
    2V &= \frac{R_{v1}}{R_{1}} \\
    R_{1} &= \frac{R_{v1}}{2} \\
    R_{1} &=  \frac{10k}{2} \\
\end{align*}

\begin{equation}
    \boxed{R_{1} = 5k \Omega}
\end{equation}

\subsubsection*{Determinar la corriente de polarización que suministra el amplificador operacional}

La corriente de polarización es la corriente que pasa por la resistencia $R_1$, por lo tanto:

\begin{align*}
    I &= \frac{V}{R_1}  \\
    I &= \frac{5V}{5k \Omega} \\
    I &= 1mA
\end{align*}

\subsubsection*{Determinar la tensión minima de secundario del transformador en función de la corriente de salida, de manera que el regulador pueda mantener la regulación}

Para obtener la tensión mínima del secundario se vuelve a utilizar la ecuación \ref{eq:tension-min-secundario}, está vez sustituyendo $V_{ir}$ por 15V:

\begin{align*}
    V_s &= 15V + \frac{I_{dc}}{2\cdot 60 \cdot 470\cdot 10^{-6}} + 2 \cdot 0.7 V \\
    V_s &= 16.40 + 17.73  I_{dc} \\
\end{align*}

Su valor rms sería:

$$ V_{srms} = \frac{V_s}{\sqrt{2}} $$

$$ V_{srms} = (11.60 + 12.53 I_{dc} )V $$

\subsubsection{Fuente de corriente variable}

\begin{figure}[ht]
    \centering
    \includegraphics[width=0.8\textwidth]{fuente-corriente-variable.png}
    \caption{Fuente de corriente variable}
    \label{fig:fuente-corriente-variable}
\end{figure}

\subsubsection*{Determinar el rango de corrientes de salida en función del accionamiento <<x>>}

utilizando la relación:

\begin{equation}
    I_o = \frac{5V}{R_1 + xR_{v1}}
\end{equation}

\begin{equation}
    I_o = \frac{5V}{240 \Omega + x 1k \Omega}
\end{equation}

para rangos de $x$ entre $0$ y $1$ se tiene que:

\begin{equation*}
    4.03 mA \leq I_o \leq 20.83 mA 
\end{equation*}

\subsubsection{Simulaciones}

\subsubsection*{Regulador de tensión de salida fija}

\begin{ilustracion}[ht]
    \centering
    \includegraphics[width=0.8\textwidth]{simulaciones/montaje-regulador-salida-fija-sin-ct.png}
    \caption{Simulación de regulador de tensión fija sin center tap}
    \label{ilus:simulacion-regulador-tension-fija-sin-ct}
\end{ilustracion}

La ilustración \ref{ilus:simulacion-regulador-tension-fija-sin-ct} muestra la simulación de un regulador de tensión fija sin center tap, se observa que para una carga de $80mA$ el voltaje de riso es $1.05 Vpp$, el cual es semejante al valor que calculamos $1.42 V$.

Se puede apreciar mejor la forma de la onda de riso en la ilustración \ref{ilus:onda-de-riso-sin-ct} 

\begin{ilustracion}[ht]
    \centering
    \includegraphics[width=0.6\textwidth]{simulaciones/onda-de-riso-sin-ct.png}
    \caption{Onda de riso sin center tap}
    \label{ilus:onda-de-riso-sin-ct}
\end{ilustracion}

En la ilustración \ref{ilus:minima-excursion-regulador-salida-fija} se puede apreciar una comparación entre el voltaje de salida del regulador cuando $V_{srms} > 7.50$ y cuando $V_{srms} < 7.50$. En el primer caso el regulador trabaja con normalidad y podemos observar una tensión de salida sin ruido de $5V$, en el segundo caso podemos observar que el regulador no funciona correctamente y aparece un ruido en la salida debido al efecto riso.

\begin{ilustracion}[ht]
    \centering
    \includegraphics[width=0.8\textwidth]{simulaciones/minima-excursion-regulador-salida-fija.png}
    \caption{Minima excursion regulador de tensión salida fija}
    \label{ilus:minima-excursion-regulador-salida-fija}
\end{ilustracion}

En la ilustración \ref{ilus:regulacion-voltaje-regulador-salida-fija} se puede apreciar que para ambos casos (con carga y sin carga) el voltaje es el mismo ($5V$), por lo tanto la regulación de voltaje es 0\%.

\begin{ilustracion}[ht]
    \centering
    \includegraphics[width=0.8\textwidth]{simulaciones/regulacion-voltaje-regulador-salida-fija.png}
    \caption{Regulación de voltaje del regulador de tensión salida fija}
    \label{ilus:regulacion-voltaje-regulador-salida-fija}
\end{ilustracion}

\FloatBarrier
\subsubsection*{Fuente regulada ajustable}

La ilustración \ref{ilus:simulacion-fuente-ajustable} muestra el montaje de una fuente ajustable usando el valor de $R_1 = 5k \Omega$. En esta ilustración se puede observar que cuando $X=0$ la tensión de salida es $V_0 = 7 V$

\begin{ilustracion}[ht]
    \centering
    \includegraphics[width=0.8\textwidth]{simulaciones/montaje-regulador-salida-variable.png}
    \caption{Simulación de fuente ajustable}
    \label{ilus:simulacion-fuente-ajustable}
\end{ilustracion}

Por otro lado, cuando $X=1$, de la ilustración \ref{ilus:voltaje-salida-max-regulador-ajustable} podemos observar que la tensión de salida es $V_0 = 15.0 V$ que es el valor que se espera para la máxima tensión de salida del regulador de tensión ajustable.

\begin{ilustracion}[ht]
    \centering
    \includegraphics[width=0.5\textwidth]{simulaciones/voltaje-salida-max-regulador-ajustable.png}
    \caption{Voltaje de salida máximo de la fuente ajustable}
    \label{ilus:voltaje-salida-max-regulador-ajustable}
\end{ilustracion}

en la ilustración \ref{ilus:corriente-polarizacion-regulador-ajustable} se observa que la corriente a traves de la resistencia $R_1$ es $I = 920 \mu A$.

\begin{ilustracion}
    \centering
    \includegraphics[width=0.6\textwidth]{simulaciones/corriente-polarizacion-regulador-ajustable.png}
    \caption{Corriente de polarización del regulador de tensión ajustable}
    \label{ilus:corriente-polarizacion-regulador-ajustable}
\end{ilustracion}

En la ilustración \ref{ilus:minima-excursion-regulador-salida-variable} se observa que cuando la tensión en el secundario del transformador es menor a $11 Vrms$ el regulador no es capaz de suministrar los $15V$ y también se puede observar el ruido del voltaje de riso $V_r$.

\begin{ilustracion}
    \centering
    \includegraphics[width=0.6\textwidth]{simulaciones/minima-excursion-regulador-salida-variable.png}
    \caption{Minima excursion regulador de tensión salida variable}
    \label{ilus:minima-excursion-regulador-salida-variable}
\end{ilustracion}

\FloatBarrier
\subsubsection*{Fuente de corriente variable}

En la ilustración \ref{ilus:simulacion-fuente-corriente} se puede observar el montaje de la fuente de corriente variable.

\begin{ilustracion}[ht]
    \centering
    \includegraphics[width=0.8\textwidth]{simulaciones/montaje-fuente-corriente.png}
    \caption{Simulación de fuente de corriente variable}
    \label{ilus:simulacion-fuente-corriente}
\end{ilustracion}

En la ilustración \ref{ilus:rango-corriente-salida}, se puede observar una comparativa de la corriente de salida mínima y máxima.

Cuando $x=0$, $I_o = 5.58 mA$ y cuando $x=1$, $I_o = 20.9mA$.

\begin{ilustracion}[ht]
    \centering
    \includegraphics[width=0.8\textwidth]{simulaciones/rango-corriente-salida.png}
    \caption{Comparación corriente de salida minima y máxima}
    \label{ilus:rango-corriente-salida}
\end{ilustracion}

\subsubsection{Procedimiento ensayo de laboratorio}

\subsubsection*{Rectificador de tensión}

\begin{enumerate}
    \item Se realiza el montaje de la etapa del rectificador de onda completa y filtro capacitivo.
    \item en caso de contar con center tap, se deja el jumper $\jumper{1}$ abierto, si no se tiene toma central, se cierra el jumper $\jumper{1}$ y se asegura que en ese nodo haya una referencia.
    \item se conecta el primario del transformador a la toma de la mesa de trabajo y el secundario del transformador a los puntos $J_1$, $J_3$ y $j_2$ en caso de contar con center tap.
    \item se conecta la referencia del osciloscopio al nodo b.
\end{enumerate}

\subsubsection*{Regulador de tensión de salida fija}

\begin{enumerate}
    \item Se realiza el montaje del circuito de la ilustración \ref{ilus:regulador-lineal-tension-fija}
    \item se coloca una carga de $68 \Omega$, luego se mide y se fotografía la tensión de riso en el condensador $C1$ 
    \item se coloca el transformador de manera que suministre más de $7Vrms$ y se mide y fotografía, el voltaje de salida del regulador.
    \item se repite el paso anterior, esta vez con el transformador suministrando menos de $7Vrms$.
    \item Ahora se quita la carga y se mide la tensión de salida, para luego colocar una carga de $50 \Omega$ y se vuelve a medir la tensión de salida.
\end{enumerate}

\subsubsection*{Regulador de tensión de salida variable}

\begin{enumerate}
    \item Se realiza el montaje del circuito de la ilustración \ref{ilus:regulador-tension-salida-ajustable}
    \item se ajusta el potenciómetro al mínimo ($x = 0$) y se mide la tensión de salida.
    \item se ajusta el potenciómetro al máximo ($x = 1$) y se mide la tensión de salida.
    \item Se mide la diferencia de tensión entre la resistencia $R_1$ (una medición a cada lado de la resistencia).
    \item se coloca el transformador de manera que suministre más de $12Vrms$ y se mide y fotografía, el voltaje de salida del regulador.
    \item se repite el paso anterior, esta vez con el transformador suministrando menos de $12Vrms$.
\end{enumerate}

\subsubsection*{Fuente de corriente variable}

\begin{enumerate}
    \item Se realiza el montaje del circuito de la ilustración \ref{ilus:fuente-corriente-variable}
    \item se conecta la carga de $100 \Omega$.
    \item se ajusta el potenciómetro al mínimo ($x = 0$) y se mide la tensión de salida (debería dar $I= 4.4 mA$).
    \item se ajusta el potenciómetro al máximo ($x = 1$) y se mide la tensión de salida (debería dar $I= 20.9 mA$).
\end{enumerate}
\FloatBarrier
\subsection{Amplificador operacional real}
\subsubsection{Regulador con tensión de salida fija}

\begin{figure}[ht]
    \centering
    \includegraphics[width=0.8\textwidth]{regulador-tension-salida-fija.png}
    \caption{Regulador lineal con tensión de salida fija}
    \label{fig:regulador-lineal-tension-fija}
\end{figure}

\subsubsection*{Explicar la función de los condensadores $C_2$ y $C_3$ en la figura \ref{fig:regulador-lineal-tension-fija}}

El dispositivo siempre debe estar equipado con un capacitor de entrada para reducir los efectos de la inductancia parásita en los cables de entrada, especialmente si el regulador está ubicado lejos de la fuente no regulada, y un capacitor de salida para ayudar a mejorar la respuesta a los cambios repentinos en la corriente de carga. Para obtener los mejores resultados, use cables y trazos gruesos, mantenga los cables cortos y monte ambos capacitores lo más cerca posible del regulador. Dependiendo del caso, puede ser necesario un disipador de calor para mantener la temperatura interna dentro de niveles tolerables.

\subsubsection*{Explicar cómo conectar el puente de diodos si el transformador no tiene toma central (CT).}

Si el transformador tiene toma central, se deja el jumper $\jumper{1}$ abierto, si no se tiene toma central, se cierra el jumper $\jumper{1}$ y se asegura que en ese nodo haya una referencia.

\subsubsection*{Suponiendo una carga de $80mA$ determine la tensión de rizado pico-pico que se va a presentar en $C_1$}.

Se tiene que el voltaje de rizo pico pico $V_{rpp}$ viene dada por la siguiente ecuación:

\begin{equation}
    V_{rpp} = \frac{I_{cd}}{2 f C}
\end{equation}

teniendo en cuenta que $f= 60Hz$ y $C= 470\mu F$  nos queda.

\begin{align*}
    V_{rpp} &= \frac{80m A}{2 \cdot 60Hz \cdot 470\mu F} \\
    V_{rpp} &= 1.42 V
\end{align*}

\subsubsection*{Determinar la tensión mínima del secundario del transformador en función de la corriente de salida, de manera que el regulador puede mantener la regulación.}

Según los datos del datasheet, la tensión de entrada mínima para mantener la regulación tiene que  ser de $ V_{ir} = 7.5V$. Ahora, para calcular la tensión en el secundario se debe calcular la caida de tensión tomando en cuenta los diodos, el voltaje de riso y el voltaje mínimo del regulador:

\begin{equation}
    V_{s} = 2 V_{d} + V_{rpp} + V_{ir}
    \label{eq:tension-min-secundario}
\end{equation}

$$ V_{s} = 10.32 V $$
 
O si se expresa en rms:

$$ V_{srms} = \frac{V_s}{\sqrt{2}} $$

$$ V_{srms} = 7.30 V $$

\subsubsection*{Determinar la regulación de voltaje que se va a obtener al colocar unas cargas de 100mA}

Recordando que la regulación de voltaje viene dada por:

\begin{equation}
    reg = \frac{V_{cc} - V_{sc}}{V_{sc}} 100 %
\end{equation}

Una carga de $100mA$ viene dada por:

$$ R = \frac{V}{I} = \frac{5V}{100mA} = 50 \Omega $$

Por tanto la regulación de voltaje será:

\begin{equation}
    reg = \frac{5 -5}{5} 100 = 0\%
\end{equation}

\subsubsection{Fuente regulada ajustable}

\begin{figure}[ht]
    \centering
    \includegraphics[width=0.8\textwidth]{regulador-tension-salida-ajustable.png}
    \caption{Fuente regulada ajustable}
    \label{fig:regulador-tension-salida-ajustable}
\end{figure}

\subsubsection*{Determinar el rango de tensiones de salida en función del accionamiento <<x>>}

sabemos que:

$$ I = \frac{5 V}{R_1}$$

$$ V_2 = I . x R_{v1}$$

y de la ecuación

\begin{equation}
    V_o = V_1 + I(V_2)
\end{equation}
 
tenemos 

\begin{align*}
    V_o &= 5V + \frac{5 V}{R_1} (x R_{v1}) \\
    V_o &= 5 \WrapParenthesis{1 + \frac{x R_{v1}}{R_{1}}} \\
\end{align*}

los valores de $x$ son $ 0 \leq x \leq 1$ por lo tanto.

\begin{equation}
    5 \leq V_o \leq 5 + \frac{R_{v1}}{R_{1}}
\end{equation}

\subsubsection*{Asignar el valor de $R_1$ de modo que la fuente suministre tensiones de hasta al menos 15V}

Partiendo de la expresión del valor máximo de $V_o$ podemos obtener el valor de $R_1$ que cumple con la condición:

\begin{align*}
    15V &= 5(1 + \frac{R_{v1}}{R_{1}}) \\
    3V &= 1 + \frac{R_{v1}}{R_{1}} \\
    2V &= \frac{R_{v1}}{R_{1}} \\
    R_{1} &= \frac{R_{v1}}{2} \\
    R_{1} &=  \frac{10k}{2} \\
\end{align*}

\begin{equation}
    \boxed{R_{1} = 5k \Omega}
\end{equation}

\subsubsection*{Determinar la corriente de polarización que suministra el amplificador operacional}

La corriente de polarización es la corriente que pasa por la resistencia $R_1$, por lo tanto:

\begin{align*}
    I &= \frac{V}{R_1}  \\
    I &= \frac{5V}{5k \Omega} \\
    I &= 1mA
\end{align*}

\subsubsection*{Determinar la tensión minima de secundario del transformador en función de la corriente de salida, de manera que el regulador pueda mantener la regulación}

Para obtener la tensión mínima del secundario se vuelve a utilizar la ecuación \ref{eq:tension-min-secundario}, está vez sustituyendo $V_{ir}$ por 15V:

\begin{align*}
    V_s &= 15V + \frac{I_{dc}}{2\cdot 60 \cdot 470\cdot 10^{-6}} + 2 \cdot 0.7 V \\
    V_s &= 16.40 + 17.73  I_{dc} \\
\end{align*}

Su valor rms sería:

$$ V_{srms} = \frac{V_s}{\sqrt{2}} $$

$$ V_{srms} = (11.60 + 12.53 I_{dc} )V $$

\subsubsection{Fuente de corriente variable}

\begin{figure}[ht]
    \centering
    \includegraphics[width=0.8\textwidth]{fuente-corriente-variable.png}
    \caption{Fuente de corriente variable}
    \label{fig:fuente-corriente-variable}
\end{figure}

\subsubsection*{Determinar el rango de corrientes de salida en función del accionamiento <<x>>}

utilizando la relación:

\begin{equation}
    I_o = \frac{5V}{R_1 + xR_{v1}}
\end{equation}

\begin{equation}
    I_o = \frac{5V}{240 \Omega + x 1k \Omega}
\end{equation}

para rangos de $x$ entre $0$ y $1$ se tiene que:

\begin{equation*}
    4.03 mA \leq I_o \leq 20.83 mA 
\end{equation*}

\subsubsection{Simulaciones}

\subsubsection*{Regulador de tensión de salida fija}

\begin{ilustracion}[ht]
    \centering
    \includegraphics[width=0.8\textwidth]{simulaciones/montaje-regulador-salida-fija-sin-ct.png}
    \caption{Simulación de regulador de tensión fija sin center tap}
    \label{ilus:simulacion-regulador-tension-fija-sin-ct}
\end{ilustracion}

La ilustración \ref{ilus:simulacion-regulador-tension-fija-sin-ct} muestra la simulación de un regulador de tensión fija sin center tap, se observa que para una carga de $80mA$ el voltaje de riso es $1.05 Vpp$, el cual es semejante al valor que calculamos $1.42 V$.

Se puede apreciar mejor la forma de la onda de riso en la ilustración \ref{ilus:onda-de-riso-sin-ct} 

\begin{ilustracion}[ht]
    \centering
    \includegraphics[width=0.6\textwidth]{simulaciones/onda-de-riso-sin-ct.png}
    \caption{Onda de riso sin center tap}
    \label{ilus:onda-de-riso-sin-ct}
\end{ilustracion}

En la ilustración \ref{ilus:minima-excursion-regulador-salida-fija} se puede apreciar una comparación entre el voltaje de salida del regulador cuando $V_{srms} > 7.50$ y cuando $V_{srms} < 7.50$. En el primer caso el regulador trabaja con normalidad y podemos observar una tensión de salida sin ruido de $5V$, en el segundo caso podemos observar que el regulador no funciona correctamente y aparece un ruido en la salida debido al efecto riso.

\begin{ilustracion}[ht]
    \centering
    \includegraphics[width=0.8\textwidth]{simulaciones/minima-excursion-regulador-salida-fija.png}
    \caption{Minima excursion regulador de tensión salida fija}
    \label{ilus:minima-excursion-regulador-salida-fija}
\end{ilustracion}

En la ilustración \ref{ilus:regulacion-voltaje-regulador-salida-fija} se puede apreciar que para ambos casos (con carga y sin carga) el voltaje es el mismo ($5V$), por lo tanto la regulación de voltaje es 0\%.

\begin{ilustracion}[ht]
    \centering
    \includegraphics[width=0.8\textwidth]{simulaciones/regulacion-voltaje-regulador-salida-fija.png}
    \caption{Regulación de voltaje del regulador de tensión salida fija}
    \label{ilus:regulacion-voltaje-regulador-salida-fija}
\end{ilustracion}

\FloatBarrier
\subsubsection*{Fuente regulada ajustable}

La ilustración \ref{ilus:simulacion-fuente-ajustable} muestra el montaje de una fuente ajustable usando el valor de $R_1 = 5k \Omega$. En esta ilustración se puede observar que cuando $X=0$ la tensión de salida es $V_0 = 7 V$

\begin{ilustracion}[ht]
    \centering
    \includegraphics[width=0.8\textwidth]{simulaciones/montaje-regulador-salida-variable.png}
    \caption{Simulación de fuente ajustable}
    \label{ilus:simulacion-fuente-ajustable}
\end{ilustracion}

Por otro lado, cuando $X=1$, de la ilustración \ref{ilus:voltaje-salida-max-regulador-ajustable} podemos observar que la tensión de salida es $V_0 = 15.0 V$ que es el valor que se espera para la máxima tensión de salida del regulador de tensión ajustable.

\begin{ilustracion}[ht]
    \centering
    \includegraphics[width=0.5\textwidth]{simulaciones/voltaje-salida-max-regulador-ajustable.png}
    \caption{Voltaje de salida máximo de la fuente ajustable}
    \label{ilus:voltaje-salida-max-regulador-ajustable}
\end{ilustracion}

en la ilustración \ref{ilus:corriente-polarizacion-regulador-ajustable} se observa que la corriente a traves de la resistencia $R_1$ es $I = 920 \mu A$.

\begin{ilustracion}
    \centering
    \includegraphics[width=0.6\textwidth]{simulaciones/corriente-polarizacion-regulador-ajustable.png}
    \caption{Corriente de polarización del regulador de tensión ajustable}
    \label{ilus:corriente-polarizacion-regulador-ajustable}
\end{ilustracion}

En la ilustración \ref{ilus:minima-excursion-regulador-salida-variable} se observa que cuando la tensión en el secundario del transformador es menor a $11 Vrms$ el regulador no es capaz de suministrar los $15V$ y también se puede observar el ruido del voltaje de riso $V_r$.

\begin{ilustracion}
    \centering
    \includegraphics[width=0.6\textwidth]{simulaciones/minima-excursion-regulador-salida-variable.png}
    \caption{Minima excursion regulador de tensión salida variable}
    \label{ilus:minima-excursion-regulador-salida-variable}
\end{ilustracion}

\FloatBarrier
\subsubsection*{Fuente de corriente variable}

En la ilustración \ref{ilus:simulacion-fuente-corriente} se puede observar el montaje de la fuente de corriente variable.

\begin{ilustracion}[ht]
    \centering
    \includegraphics[width=0.8\textwidth]{simulaciones/montaje-fuente-corriente.png}
    \caption{Simulación de fuente de corriente variable}
    \label{ilus:simulacion-fuente-corriente}
\end{ilustracion}

En la ilustración \ref{ilus:rango-corriente-salida}, se puede observar una comparativa de la corriente de salida mínima y máxima.

Cuando $x=0$, $I_o = 5.58 mA$ y cuando $x=1$, $I_o = 20.9mA$.

\begin{ilustracion}[ht]
    \centering
    \includegraphics[width=0.8\textwidth]{simulaciones/rango-corriente-salida.png}
    \caption{Comparación corriente de salida minima y máxima}
    \label{ilus:rango-corriente-salida}
\end{ilustracion}

\subsubsection{Procedimiento ensayo de laboratorio}

\subsubsection*{Rectificador de tensión}

\begin{enumerate}
    \item Se realiza el montaje de la etapa del rectificador de onda completa y filtro capacitivo.
    \item en caso de contar con center tap, se deja el jumper $\jumper{1}$ abierto, si no se tiene toma central, se cierra el jumper $\jumper{1}$ y se asegura que en ese nodo haya una referencia.
    \item se conecta el primario del transformador a la toma de la mesa de trabajo y el secundario del transformador a los puntos $J_1$, $J_3$ y $j_2$ en caso de contar con center tap.
    \item se conecta la referencia del osciloscopio al nodo b.
\end{enumerate}

\subsubsection*{Regulador de tensión de salida fija}

\begin{enumerate}
    \item Se realiza el montaje del circuito de la ilustración \ref{ilus:regulador-lineal-tension-fija}
    \item se coloca una carga de $68 \Omega$, luego se mide y se fotografía la tensión de riso en el condensador $C1$ 
    \item se coloca el transformador de manera que suministre más de $7Vrms$ y se mide y fotografía, el voltaje de salida del regulador.
    \item se repite el paso anterior, esta vez con el transformador suministrando menos de $7Vrms$.
    \item Ahora se quita la carga y se mide la tensión de salida, para luego colocar una carga de $50 \Omega$ y se vuelve a medir la tensión de salida.
\end{enumerate}

\subsubsection*{Regulador de tensión de salida variable}

\begin{enumerate}
    \item Se realiza el montaje del circuito de la ilustración \ref{ilus:regulador-tension-salida-ajustable}
    \item se ajusta el potenciómetro al mínimo ($x = 0$) y se mide la tensión de salida.
    \item se ajusta el potenciómetro al máximo ($x = 1$) y se mide la tensión de salida.
    \item Se mide la diferencia de tensión entre la resistencia $R_1$ (una medición a cada lado de la resistencia).
    \item se coloca el transformador de manera que suministre más de $12Vrms$ y se mide y fotografía, el voltaje de salida del regulador.
    \item se repite el paso anterior, esta vez con el transformador suministrando menos de $12Vrms$.
\end{enumerate}

\subsubsection*{Fuente de corriente variable}

\begin{enumerate}
    \item Se realiza el montaje del circuito de la ilustración \ref{ilus:fuente-corriente-variable}
    \item se conecta la carga de $100 \Omega$.
    \item se ajusta el potenciómetro al mínimo ($x = 0$) y se mide la tensión de salida (debería dar $I= 4.4 mA$).
    \item se ajusta el potenciómetro al máximo ($x = 1$) y se mide la tensión de salida (debería dar $I= 20.9 mA$).
\end{enumerate}
\FloatBarrier
\subsection{Filtros activos}
\subsubsection{Regulador con tensión de salida fija}

\begin{figure}[ht]
    \centering
    \includegraphics[width=0.8\textwidth]{regulador-tension-salida-fija.png}
    \caption{Regulador lineal con tensión de salida fija}
    \label{fig:regulador-lineal-tension-fija}
\end{figure}

\subsubsection*{Explicar la función de los condensadores $C_2$ y $C_3$ en la figura \ref{fig:regulador-lineal-tension-fija}}

El dispositivo siempre debe estar equipado con un capacitor de entrada para reducir los efectos de la inductancia parásita en los cables de entrada, especialmente si el regulador está ubicado lejos de la fuente no regulada, y un capacitor de salida para ayudar a mejorar la respuesta a los cambios repentinos en la corriente de carga. Para obtener los mejores resultados, use cables y trazos gruesos, mantenga los cables cortos y monte ambos capacitores lo más cerca posible del regulador. Dependiendo del caso, puede ser necesario un disipador de calor para mantener la temperatura interna dentro de niveles tolerables.

\subsubsection*{Explicar cómo conectar el puente de diodos si el transformador no tiene toma central (CT).}

Si el transformador tiene toma central, se deja el jumper $\jumper{1}$ abierto, si no se tiene toma central, se cierra el jumper $\jumper{1}$ y se asegura que en ese nodo haya una referencia.

\subsubsection*{Suponiendo una carga de $80mA$ determine la tensión de rizado pico-pico que se va a presentar en $C_1$}.

Se tiene que el voltaje de rizo pico pico $V_{rpp}$ viene dada por la siguiente ecuación:

\begin{equation}
    V_{rpp} = \frac{I_{cd}}{2 f C}
\end{equation}

teniendo en cuenta que $f= 60Hz$ y $C= 470\mu F$  nos queda.

\begin{align*}
    V_{rpp} &= \frac{80m A}{2 \cdot 60Hz \cdot 470\mu F} \\
    V_{rpp} &= 1.42 V
\end{align*}

\subsubsection*{Determinar la tensión mínima del secundario del transformador en función de la corriente de salida, de manera que el regulador puede mantener la regulación.}

Según los datos del datasheet, la tensión de entrada mínima para mantener la regulación tiene que  ser de $ V_{ir} = 7.5V$. Ahora, para calcular la tensión en el secundario se debe calcular la caida de tensión tomando en cuenta los diodos, el voltaje de riso y el voltaje mínimo del regulador:

\begin{equation}
    V_{s} = 2 V_{d} + V_{rpp} + V_{ir}
    \label{eq:tension-min-secundario}
\end{equation}

$$ V_{s} = 10.32 V $$
 
O si se expresa en rms:

$$ V_{srms} = \frac{V_s}{\sqrt{2}} $$

$$ V_{srms} = 7.30 V $$

\subsubsection*{Determinar la regulación de voltaje que se va a obtener al colocar unas cargas de 100mA}

Recordando que la regulación de voltaje viene dada por:

\begin{equation}
    reg = \frac{V_{cc} - V_{sc}}{V_{sc}} 100 %
\end{equation}

Una carga de $100mA$ viene dada por:

$$ R = \frac{V}{I} = \frac{5V}{100mA} = 50 \Omega $$

Por tanto la regulación de voltaje será:

\begin{equation}
    reg = \frac{5 -5}{5} 100 = 0\%
\end{equation}

\subsubsection{Fuente regulada ajustable}

\begin{figure}[ht]
    \centering
    \includegraphics[width=0.8\textwidth]{regulador-tension-salida-ajustable.png}
    \caption{Fuente regulada ajustable}
    \label{fig:regulador-tension-salida-ajustable}
\end{figure}

\subsubsection*{Determinar el rango de tensiones de salida en función del accionamiento <<x>>}

sabemos que:

$$ I = \frac{5 V}{R_1}$$

$$ V_2 = I . x R_{v1}$$

y de la ecuación

\begin{equation}
    V_o = V_1 + I(V_2)
\end{equation}
 
tenemos 

\begin{align*}
    V_o &= 5V + \frac{5 V}{R_1} (x R_{v1}) \\
    V_o &= 5 \WrapParenthesis{1 + \frac{x R_{v1}}{R_{1}}} \\
\end{align*}

los valores de $x$ son $ 0 \leq x \leq 1$ por lo tanto.

\begin{equation}
    5 \leq V_o \leq 5 + \frac{R_{v1}}{R_{1}}
\end{equation}

\subsubsection*{Asignar el valor de $R_1$ de modo que la fuente suministre tensiones de hasta al menos 15V}

Partiendo de la expresión del valor máximo de $V_o$ podemos obtener el valor de $R_1$ que cumple con la condición:

\begin{align*}
    15V &= 5(1 + \frac{R_{v1}}{R_{1}}) \\
    3V &= 1 + \frac{R_{v1}}{R_{1}} \\
    2V &= \frac{R_{v1}}{R_{1}} \\
    R_{1} &= \frac{R_{v1}}{2} \\
    R_{1} &=  \frac{10k}{2} \\
\end{align*}

\begin{equation}
    \boxed{R_{1} = 5k \Omega}
\end{equation}

\subsubsection*{Determinar la corriente de polarización que suministra el amplificador operacional}

La corriente de polarización es la corriente que pasa por la resistencia $R_1$, por lo tanto:

\begin{align*}
    I &= \frac{V}{R_1}  \\
    I &= \frac{5V}{5k \Omega} \\
    I &= 1mA
\end{align*}

\subsubsection*{Determinar la tensión minima de secundario del transformador en función de la corriente de salida, de manera que el regulador pueda mantener la regulación}

Para obtener la tensión mínima del secundario se vuelve a utilizar la ecuación \ref{eq:tension-min-secundario}, está vez sustituyendo $V_{ir}$ por 15V:

\begin{align*}
    V_s &= 15V + \frac{I_{dc}}{2\cdot 60 \cdot 470\cdot 10^{-6}} + 2 \cdot 0.7 V \\
    V_s &= 16.40 + 17.73  I_{dc} \\
\end{align*}

Su valor rms sería:

$$ V_{srms} = \frac{V_s}{\sqrt{2}} $$

$$ V_{srms} = (11.60 + 12.53 I_{dc} )V $$

\subsubsection{Fuente de corriente variable}

\begin{figure}[ht]
    \centering
    \includegraphics[width=0.8\textwidth]{fuente-corriente-variable.png}
    \caption{Fuente de corriente variable}
    \label{fig:fuente-corriente-variable}
\end{figure}

\subsubsection*{Determinar el rango de corrientes de salida en función del accionamiento <<x>>}

utilizando la relación:

\begin{equation}
    I_o = \frac{5V}{R_1 + xR_{v1}}
\end{equation}

\begin{equation}
    I_o = \frac{5V}{240 \Omega + x 1k \Omega}
\end{equation}

para rangos de $x$ entre $0$ y $1$ se tiene que:

\begin{equation*}
    4.03 mA \leq I_o \leq 20.83 mA 
\end{equation*}

\subsubsection{Simulaciones}

\subsubsection*{Regulador de tensión de salida fija}

\begin{ilustracion}[ht]
    \centering
    \includegraphics[width=0.8\textwidth]{simulaciones/montaje-regulador-salida-fija-sin-ct.png}
    \caption{Simulación de regulador de tensión fija sin center tap}
    \label{ilus:simulacion-regulador-tension-fija-sin-ct}
\end{ilustracion}

La ilustración \ref{ilus:simulacion-regulador-tension-fija-sin-ct} muestra la simulación de un regulador de tensión fija sin center tap, se observa que para una carga de $80mA$ el voltaje de riso es $1.05 Vpp$, el cual es semejante al valor que calculamos $1.42 V$.

Se puede apreciar mejor la forma de la onda de riso en la ilustración \ref{ilus:onda-de-riso-sin-ct} 

\begin{ilustracion}[ht]
    \centering
    \includegraphics[width=0.6\textwidth]{simulaciones/onda-de-riso-sin-ct.png}
    \caption{Onda de riso sin center tap}
    \label{ilus:onda-de-riso-sin-ct}
\end{ilustracion}

En la ilustración \ref{ilus:minima-excursion-regulador-salida-fija} se puede apreciar una comparación entre el voltaje de salida del regulador cuando $V_{srms} > 7.50$ y cuando $V_{srms} < 7.50$. En el primer caso el regulador trabaja con normalidad y podemos observar una tensión de salida sin ruido de $5V$, en el segundo caso podemos observar que el regulador no funciona correctamente y aparece un ruido en la salida debido al efecto riso.

\begin{ilustracion}[ht]
    \centering
    \includegraphics[width=0.8\textwidth]{simulaciones/minima-excursion-regulador-salida-fija.png}
    \caption{Minima excursion regulador de tensión salida fija}
    \label{ilus:minima-excursion-regulador-salida-fija}
\end{ilustracion}

En la ilustración \ref{ilus:regulacion-voltaje-regulador-salida-fija} se puede apreciar que para ambos casos (con carga y sin carga) el voltaje es el mismo ($5V$), por lo tanto la regulación de voltaje es 0\%.

\begin{ilustracion}[ht]
    \centering
    \includegraphics[width=0.8\textwidth]{simulaciones/regulacion-voltaje-regulador-salida-fija.png}
    \caption{Regulación de voltaje del regulador de tensión salida fija}
    \label{ilus:regulacion-voltaje-regulador-salida-fija}
\end{ilustracion}

\FloatBarrier
\subsubsection*{Fuente regulada ajustable}

La ilustración \ref{ilus:simulacion-fuente-ajustable} muestra el montaje de una fuente ajustable usando el valor de $R_1 = 5k \Omega$. En esta ilustración se puede observar que cuando $X=0$ la tensión de salida es $V_0 = 7 V$

\begin{ilustracion}[ht]
    \centering
    \includegraphics[width=0.8\textwidth]{simulaciones/montaje-regulador-salida-variable.png}
    \caption{Simulación de fuente ajustable}
    \label{ilus:simulacion-fuente-ajustable}
\end{ilustracion}

Por otro lado, cuando $X=1$, de la ilustración \ref{ilus:voltaje-salida-max-regulador-ajustable} podemos observar que la tensión de salida es $V_0 = 15.0 V$ que es el valor que se espera para la máxima tensión de salida del regulador de tensión ajustable.

\begin{ilustracion}[ht]
    \centering
    \includegraphics[width=0.5\textwidth]{simulaciones/voltaje-salida-max-regulador-ajustable.png}
    \caption{Voltaje de salida máximo de la fuente ajustable}
    \label{ilus:voltaje-salida-max-regulador-ajustable}
\end{ilustracion}

en la ilustración \ref{ilus:corriente-polarizacion-regulador-ajustable} se observa que la corriente a traves de la resistencia $R_1$ es $I = 920 \mu A$.

\begin{ilustracion}
    \centering
    \includegraphics[width=0.6\textwidth]{simulaciones/corriente-polarizacion-regulador-ajustable.png}
    \caption{Corriente de polarización del regulador de tensión ajustable}
    \label{ilus:corriente-polarizacion-regulador-ajustable}
\end{ilustracion}

En la ilustración \ref{ilus:minima-excursion-regulador-salida-variable} se observa que cuando la tensión en el secundario del transformador es menor a $11 Vrms$ el regulador no es capaz de suministrar los $15V$ y también se puede observar el ruido del voltaje de riso $V_r$.

\begin{ilustracion}
    \centering
    \includegraphics[width=0.6\textwidth]{simulaciones/minima-excursion-regulador-salida-variable.png}
    \caption{Minima excursion regulador de tensión salida variable}
    \label{ilus:minima-excursion-regulador-salida-variable}
\end{ilustracion}

\FloatBarrier
\subsubsection*{Fuente de corriente variable}

En la ilustración \ref{ilus:simulacion-fuente-corriente} se puede observar el montaje de la fuente de corriente variable.

\begin{ilustracion}[ht]
    \centering
    \includegraphics[width=0.8\textwidth]{simulaciones/montaje-fuente-corriente.png}
    \caption{Simulación de fuente de corriente variable}
    \label{ilus:simulacion-fuente-corriente}
\end{ilustracion}

En la ilustración \ref{ilus:rango-corriente-salida}, se puede observar una comparativa de la corriente de salida mínima y máxima.

Cuando $x=0$, $I_o = 5.58 mA$ y cuando $x=1$, $I_o = 20.9mA$.

\begin{ilustracion}[ht]
    \centering
    \includegraphics[width=0.8\textwidth]{simulaciones/rango-corriente-salida.png}
    \caption{Comparación corriente de salida minima y máxima}
    \label{ilus:rango-corriente-salida}
\end{ilustracion}

\subsubsection{Procedimiento ensayo de laboratorio}

\subsubsection*{Rectificador de tensión}

\begin{enumerate}
    \item Se realiza el montaje de la etapa del rectificador de onda completa y filtro capacitivo.
    \item en caso de contar con center tap, se deja el jumper $\jumper{1}$ abierto, si no se tiene toma central, se cierra el jumper $\jumper{1}$ y se asegura que en ese nodo haya una referencia.
    \item se conecta el primario del transformador a la toma de la mesa de trabajo y el secundario del transformador a los puntos $J_1$, $J_3$ y $j_2$ en caso de contar con center tap.
    \item se conecta la referencia del osciloscopio al nodo b.
\end{enumerate}

\subsubsection*{Regulador de tensión de salida fija}

\begin{enumerate}
    \item Se realiza el montaje del circuito de la ilustración \ref{ilus:regulador-lineal-tension-fija}
    \item se coloca una carga de $68 \Omega$, luego se mide y se fotografía la tensión de riso en el condensador $C1$ 
    \item se coloca el transformador de manera que suministre más de $7Vrms$ y se mide y fotografía, el voltaje de salida del regulador.
    \item se repite el paso anterior, esta vez con el transformador suministrando menos de $7Vrms$.
    \item Ahora se quita la carga y se mide la tensión de salida, para luego colocar una carga de $50 \Omega$ y se vuelve a medir la tensión de salida.
\end{enumerate}

\subsubsection*{Regulador de tensión de salida variable}

\begin{enumerate}
    \item Se realiza el montaje del circuito de la ilustración \ref{ilus:regulador-tension-salida-ajustable}
    \item se ajusta el potenciómetro al mínimo ($x = 0$) y se mide la tensión de salida.
    \item se ajusta el potenciómetro al máximo ($x = 1$) y se mide la tensión de salida.
    \item Se mide la diferencia de tensión entre la resistencia $R_1$ (una medición a cada lado de la resistencia).
    \item se coloca el transformador de manera que suministre más de $12Vrms$ y se mide y fotografía, el voltaje de salida del regulador.
    \item se repite el paso anterior, esta vez con el transformador suministrando menos de $12Vrms$.
\end{enumerate}

\subsubsection*{Fuente de corriente variable}

\begin{enumerate}
    \item Se realiza el montaje del circuito de la ilustración \ref{ilus:fuente-corriente-variable}
    \item se conecta la carga de $100 \Omega$.
    \item se ajusta el potenciómetro al mínimo ($x = 0$) y se mide la tensión de salida (debería dar $I= 4.4 mA$).
    \item se ajusta el potenciómetro al máximo ($x = 1$) y se mide la tensión de salida (debería dar $I= 20.9 mA$).
\end{enumerate}
\FloatBarrier
\subsection{Fuentes lineales y reguladores monolíticos}
\subsubsection{Regulador con tensión de salida fija}

\begin{figure}[ht]
    \centering
    \includegraphics[width=0.8\textwidth]{regulador-tension-salida-fija.png}
    \caption{Regulador lineal con tensión de salida fija}
    \label{fig:regulador-lineal-tension-fija}
\end{figure}

\subsubsection*{Explicar la función de los condensadores $C_2$ y $C_3$ en la figura \ref{fig:regulador-lineal-tension-fija}}

El dispositivo siempre debe estar equipado con un capacitor de entrada para reducir los efectos de la inductancia parásita en los cables de entrada, especialmente si el regulador está ubicado lejos de la fuente no regulada, y un capacitor de salida para ayudar a mejorar la respuesta a los cambios repentinos en la corriente de carga. Para obtener los mejores resultados, use cables y trazos gruesos, mantenga los cables cortos y monte ambos capacitores lo más cerca posible del regulador. Dependiendo del caso, puede ser necesario un disipador de calor para mantener la temperatura interna dentro de niveles tolerables.

\subsubsection*{Explicar cómo conectar el puente de diodos si el transformador no tiene toma central (CT).}

Si el transformador tiene toma central, se deja el jumper $\jumper{1}$ abierto, si no se tiene toma central, se cierra el jumper $\jumper{1}$ y se asegura que en ese nodo haya una referencia.

\subsubsection*{Suponiendo una carga de $80mA$ determine la tensión de rizado pico-pico que se va a presentar en $C_1$}.

Se tiene que el voltaje de rizo pico pico $V_{rpp}$ viene dada por la siguiente ecuación:

\begin{equation}
    V_{rpp} = \frac{I_{cd}}{2 f C}
\end{equation}

teniendo en cuenta que $f= 60Hz$ y $C= 470\mu F$  nos queda.

\begin{align*}
    V_{rpp} &= \frac{80m A}{2 \cdot 60Hz \cdot 470\mu F} \\
    V_{rpp} &= 1.42 V
\end{align*}

\subsubsection*{Determinar la tensión mínima del secundario del transformador en función de la corriente de salida, de manera que el regulador puede mantener la regulación.}

Según los datos del datasheet, la tensión de entrada mínima para mantener la regulación tiene que  ser de $ V_{ir} = 7.5V$. Ahora, para calcular la tensión en el secundario se debe calcular la caida de tensión tomando en cuenta los diodos, el voltaje de riso y el voltaje mínimo del regulador:

\begin{equation}
    V_{s} = 2 V_{d} + V_{rpp} + V_{ir}
    \label{eq:tension-min-secundario}
\end{equation}

$$ V_{s} = 10.32 V $$
 
O si se expresa en rms:

$$ V_{srms} = \frac{V_s}{\sqrt{2}} $$

$$ V_{srms} = 7.30 V $$

\subsubsection*{Determinar la regulación de voltaje que se va a obtener al colocar unas cargas de 100mA}

Recordando que la regulación de voltaje viene dada por:

\begin{equation}
    reg = \frac{V_{cc} - V_{sc}}{V_{sc}} 100 %
\end{equation}

Una carga de $100mA$ viene dada por:

$$ R = \frac{V}{I} = \frac{5V}{100mA} = 50 \Omega $$

Por tanto la regulación de voltaje será:

\begin{equation}
    reg = \frac{5 -5}{5} 100 = 0\%
\end{equation}

\subsubsection{Fuente regulada ajustable}

\begin{figure}[ht]
    \centering
    \includegraphics[width=0.8\textwidth]{regulador-tension-salida-ajustable.png}
    \caption{Fuente regulada ajustable}
    \label{fig:regulador-tension-salida-ajustable}
\end{figure}

\subsubsection*{Determinar el rango de tensiones de salida en función del accionamiento <<x>>}

sabemos que:

$$ I = \frac{5 V}{R_1}$$

$$ V_2 = I . x R_{v1}$$

y de la ecuación

\begin{equation}
    V_o = V_1 + I(V_2)
\end{equation}
 
tenemos 

\begin{align*}
    V_o &= 5V + \frac{5 V}{R_1} (x R_{v1}) \\
    V_o &= 5 \WrapParenthesis{1 + \frac{x R_{v1}}{R_{1}}} \\
\end{align*}

los valores de $x$ son $ 0 \leq x \leq 1$ por lo tanto.

\begin{equation}
    5 \leq V_o \leq 5 + \frac{R_{v1}}{R_{1}}
\end{equation}

\subsubsection*{Asignar el valor de $R_1$ de modo que la fuente suministre tensiones de hasta al menos 15V}

Partiendo de la expresión del valor máximo de $V_o$ podemos obtener el valor de $R_1$ que cumple con la condición:

\begin{align*}
    15V &= 5(1 + \frac{R_{v1}}{R_{1}}) \\
    3V &= 1 + \frac{R_{v1}}{R_{1}} \\
    2V &= \frac{R_{v1}}{R_{1}} \\
    R_{1} &= \frac{R_{v1}}{2} \\
    R_{1} &=  \frac{10k}{2} \\
\end{align*}

\begin{equation}
    \boxed{R_{1} = 5k \Omega}
\end{equation}

\subsubsection*{Determinar la corriente de polarización que suministra el amplificador operacional}

La corriente de polarización es la corriente que pasa por la resistencia $R_1$, por lo tanto:

\begin{align*}
    I &= \frac{V}{R_1}  \\
    I &= \frac{5V}{5k \Omega} \\
    I &= 1mA
\end{align*}

\subsubsection*{Determinar la tensión minima de secundario del transformador en función de la corriente de salida, de manera que el regulador pueda mantener la regulación}

Para obtener la tensión mínima del secundario se vuelve a utilizar la ecuación \ref{eq:tension-min-secundario}, está vez sustituyendo $V_{ir}$ por 15V:

\begin{align*}
    V_s &= 15V + \frac{I_{dc}}{2\cdot 60 \cdot 470\cdot 10^{-6}} + 2 \cdot 0.7 V \\
    V_s &= 16.40 + 17.73  I_{dc} \\
\end{align*}

Su valor rms sería:

$$ V_{srms} = \frac{V_s}{\sqrt{2}} $$

$$ V_{srms} = (11.60 + 12.53 I_{dc} )V $$

\subsubsection{Fuente de corriente variable}

\begin{figure}[ht]
    \centering
    \includegraphics[width=0.8\textwidth]{fuente-corriente-variable.png}
    \caption{Fuente de corriente variable}
    \label{fig:fuente-corriente-variable}
\end{figure}

\subsubsection*{Determinar el rango de corrientes de salida en función del accionamiento <<x>>}

utilizando la relación:

\begin{equation}
    I_o = \frac{5V}{R_1 + xR_{v1}}
\end{equation}

\begin{equation}
    I_o = \frac{5V}{240 \Omega + x 1k \Omega}
\end{equation}

para rangos de $x$ entre $0$ y $1$ se tiene que:

\begin{equation*}
    4.03 mA \leq I_o \leq 20.83 mA 
\end{equation*}

\subsubsection{Simulaciones}

\subsubsection*{Regulador de tensión de salida fija}

\begin{ilustracion}[ht]
    \centering
    \includegraphics[width=0.8\textwidth]{simulaciones/montaje-regulador-salida-fija-sin-ct.png}
    \caption{Simulación de regulador de tensión fija sin center tap}
    \label{ilus:simulacion-regulador-tension-fija-sin-ct}
\end{ilustracion}

La ilustración \ref{ilus:simulacion-regulador-tension-fija-sin-ct} muestra la simulación de un regulador de tensión fija sin center tap, se observa que para una carga de $80mA$ el voltaje de riso es $1.05 Vpp$, el cual es semejante al valor que calculamos $1.42 V$.

Se puede apreciar mejor la forma de la onda de riso en la ilustración \ref{ilus:onda-de-riso-sin-ct} 

\begin{ilustracion}[ht]
    \centering
    \includegraphics[width=0.6\textwidth]{simulaciones/onda-de-riso-sin-ct.png}
    \caption{Onda de riso sin center tap}
    \label{ilus:onda-de-riso-sin-ct}
\end{ilustracion}

En la ilustración \ref{ilus:minima-excursion-regulador-salida-fija} se puede apreciar una comparación entre el voltaje de salida del regulador cuando $V_{srms} > 7.50$ y cuando $V_{srms} < 7.50$. En el primer caso el regulador trabaja con normalidad y podemos observar una tensión de salida sin ruido de $5V$, en el segundo caso podemos observar que el regulador no funciona correctamente y aparece un ruido en la salida debido al efecto riso.

\begin{ilustracion}[ht]
    \centering
    \includegraphics[width=0.8\textwidth]{simulaciones/minima-excursion-regulador-salida-fija.png}
    \caption{Minima excursion regulador de tensión salida fija}
    \label{ilus:minima-excursion-regulador-salida-fija}
\end{ilustracion}

En la ilustración \ref{ilus:regulacion-voltaje-regulador-salida-fija} se puede apreciar que para ambos casos (con carga y sin carga) el voltaje es el mismo ($5V$), por lo tanto la regulación de voltaje es 0\%.

\begin{ilustracion}[ht]
    \centering
    \includegraphics[width=0.8\textwidth]{simulaciones/regulacion-voltaje-regulador-salida-fija.png}
    \caption{Regulación de voltaje del regulador de tensión salida fija}
    \label{ilus:regulacion-voltaje-regulador-salida-fija}
\end{ilustracion}

\FloatBarrier
\subsubsection*{Fuente regulada ajustable}

La ilustración \ref{ilus:simulacion-fuente-ajustable} muestra el montaje de una fuente ajustable usando el valor de $R_1 = 5k \Omega$. En esta ilustración se puede observar que cuando $X=0$ la tensión de salida es $V_0 = 7 V$

\begin{ilustracion}[ht]
    \centering
    \includegraphics[width=0.8\textwidth]{simulaciones/montaje-regulador-salida-variable.png}
    \caption{Simulación de fuente ajustable}
    \label{ilus:simulacion-fuente-ajustable}
\end{ilustracion}

Por otro lado, cuando $X=1$, de la ilustración \ref{ilus:voltaje-salida-max-regulador-ajustable} podemos observar que la tensión de salida es $V_0 = 15.0 V$ que es el valor que se espera para la máxima tensión de salida del regulador de tensión ajustable.

\begin{ilustracion}[ht]
    \centering
    \includegraphics[width=0.5\textwidth]{simulaciones/voltaje-salida-max-regulador-ajustable.png}
    \caption{Voltaje de salida máximo de la fuente ajustable}
    \label{ilus:voltaje-salida-max-regulador-ajustable}
\end{ilustracion}

en la ilustración \ref{ilus:corriente-polarizacion-regulador-ajustable} se observa que la corriente a traves de la resistencia $R_1$ es $I = 920 \mu A$.

\begin{ilustracion}
    \centering
    \includegraphics[width=0.6\textwidth]{simulaciones/corriente-polarizacion-regulador-ajustable.png}
    \caption{Corriente de polarización del regulador de tensión ajustable}
    \label{ilus:corriente-polarizacion-regulador-ajustable}
\end{ilustracion}

En la ilustración \ref{ilus:minima-excursion-regulador-salida-variable} se observa que cuando la tensión en el secundario del transformador es menor a $11 Vrms$ el regulador no es capaz de suministrar los $15V$ y también se puede observar el ruido del voltaje de riso $V_r$.

\begin{ilustracion}
    \centering
    \includegraphics[width=0.6\textwidth]{simulaciones/minima-excursion-regulador-salida-variable.png}
    \caption{Minima excursion regulador de tensión salida variable}
    \label{ilus:minima-excursion-regulador-salida-variable}
\end{ilustracion}

\FloatBarrier
\subsubsection*{Fuente de corriente variable}

En la ilustración \ref{ilus:simulacion-fuente-corriente} se puede observar el montaje de la fuente de corriente variable.

\begin{ilustracion}[ht]
    \centering
    \includegraphics[width=0.8\textwidth]{simulaciones/montaje-fuente-corriente.png}
    \caption{Simulación de fuente de corriente variable}
    \label{ilus:simulacion-fuente-corriente}
\end{ilustracion}

En la ilustración \ref{ilus:rango-corriente-salida}, se puede observar una comparativa de la corriente de salida mínima y máxima.

Cuando $x=0$, $I_o = 5.58 mA$ y cuando $x=1$, $I_o = 20.9mA$.

\begin{ilustracion}[ht]
    \centering
    \includegraphics[width=0.8\textwidth]{simulaciones/rango-corriente-salida.png}
    \caption{Comparación corriente de salida minima y máxima}
    \label{ilus:rango-corriente-salida}
\end{ilustracion}

\subsubsection{Procedimiento ensayo de laboratorio}

\subsubsection*{Rectificador de tensión}

\begin{enumerate}
    \item Se realiza el montaje de la etapa del rectificador de onda completa y filtro capacitivo.
    \item en caso de contar con center tap, se deja el jumper $\jumper{1}$ abierto, si no se tiene toma central, se cierra el jumper $\jumper{1}$ y se asegura que en ese nodo haya una referencia.
    \item se conecta el primario del transformador a la toma de la mesa de trabajo y el secundario del transformador a los puntos $J_1$, $J_3$ y $j_2$ en caso de contar con center tap.
    \item se conecta la referencia del osciloscopio al nodo b.
\end{enumerate}

\subsubsection*{Regulador de tensión de salida fija}

\begin{enumerate}
    \item Se realiza el montaje del circuito de la ilustración \ref{ilus:regulador-lineal-tension-fija}
    \item se coloca una carga de $68 \Omega$, luego se mide y se fotografía la tensión de riso en el condensador $C1$ 
    \item se coloca el transformador de manera que suministre más de $7Vrms$ y se mide y fotografía, el voltaje de salida del regulador.
    \item se repite el paso anterior, esta vez con el transformador suministrando menos de $7Vrms$.
    \item Ahora se quita la carga y se mide la tensión de salida, para luego colocar una carga de $50 \Omega$ y se vuelve a medir la tensión de salida.
\end{enumerate}

\subsubsection*{Regulador de tensión de salida variable}

\begin{enumerate}
    \item Se realiza el montaje del circuito de la ilustración \ref{ilus:regulador-tension-salida-ajustable}
    \item se ajusta el potenciómetro al mínimo ($x = 0$) y se mide la tensión de salida.
    \item se ajusta el potenciómetro al máximo ($x = 1$) y se mide la tensión de salida.
    \item Se mide la diferencia de tensión entre la resistencia $R_1$ (una medición a cada lado de la resistencia).
    \item se coloca el transformador de manera que suministre más de $12Vrms$ y se mide y fotografía, el voltaje de salida del regulador.
    \item se repite el paso anterior, esta vez con el transformador suministrando menos de $12Vrms$.
\end{enumerate}

\subsubsection*{Fuente de corriente variable}

\begin{enumerate}
    \item Se realiza el montaje del circuito de la ilustración \ref{ilus:fuente-corriente-variable}
    \item se conecta la carga de $100 \Omega$.
    \item se ajusta el potenciómetro al mínimo ($x = 0$) y se mide la tensión de salida (debería dar $I= 4.4 mA$).
    \item se ajusta el potenciómetro al máximo ($x = 1$) y se mide la tensión de salida (debería dar $I= 20.9 mA$).
\end{enumerate}

\section{Instrumentos y componentes}

A continuación se listan los instrumentos y componentes utilizados en la práctica de laboratorio.

\begin{itemize}
    \item Generador de ondas N°7 del laboratorio. 
    \item Fuente DC N°1 del laboratorio
    \item Osciloscopio N°7 del laboratorio
    \item Resistencias con tolerancia del 5\% y potencia de 1/4 W de valores: 100k, 20k, 40k, 22M, 100, 6.8, 1k, 91k, 910, 2k, 8.9k, 1.20k, 640, 240, 120, 5k, 10k.
    \item Condensadores de 10nF, 100nF, $470\mu F$ $0.1 \mu F$, 1uF, 100uF.
    \item Potenciometro de 10k y 1k.
    \item Regulador 7805
    \item Amplificadores $\mu A741$, $lm741$ y $mc1741$
    \item Dos protoboards.
    \item Transformador con center tap y puente rectificador de dos diodos.
\end{itemize}

\section{Presentación de resultados}
\subsection{Aplicaciones de las topologías clásicas}

\section{Presentación de resultados}

\subsection{Análisis de la etapa de potencia}


\begin{itemize}
    \item Al medir los puntos de operación en la etapa de potencia podemos observar que el máximo error en la medición de voltaje $V_{ce}$ es de tan solo $4.84\%$ por lo que nuestra predeterminación para este valor es correcto. Por otro lado podemos observar que el error en la medición $I_c$ está entre 60\% y 529\%, estos errores son grandísimos, esto se puede deber a que dichas corrientes de colector son muy pequeñas.
    \item Para el modelo dinámico del amplificador podemos observar que la impedancia de entrada $Z_i$ tiene un error de solo $7.15\%$ siendo un valor aceptable, igualmente la medición de ganancia tiene un error de $3.85\%$ indicando que la predeterminación es correcta, por su parte la medición de impedancia de salida $Z_o$ tiene un error de $91.16\%$ lo cual es muy alto. El valor de la medición es de 11, lo cual se acerca más al valor de la resistencia $r_{13} = 100\Omega$.
    \item Al modificar la etapa para que se comporte como un amplificador clase C podemos observar que los voltajes $V_{ce}$ medidos corresponden con los teóricos, mientras que las corrientes $I_e$ medidas siguen teniendo un error altísimos. No se observan cambios notables entre el punto de operación del amplificador clase C y el AB. 
    \item En la ilustración \ref{ilus:efecto-crossover-amplificador-clase-c} podemos observar el efecto crossover del amplificador clase C el cual consiste en una pequeña distorsión en la zona de cruce de la onde. No se persive ningún cambio en la ganancia con respecto al otro modo. .
\end{itemize}
\FloatBarrier
\subsection{Análisis de la etapa diferencial}


\begin{itemize}
    \item Observando el punto estático de operación del cuadro \ref{tab:med-puntos-reposo-etapa-diferencial} se observa que el error en la medición $V_{ce}$ es de 0.13\% y el de $I_c$ es de 3.91\%. Por lo tanto la predeterminación realizada para este valor es correcta.
    \item Para el modelo dinámico del amplificador en la etapa diferencial del cuadro \ref{tab:med-modelo-dinamico-etapa-diferencial-modo-diferencial} podemos observar que la impedancia de entrada $Z_i$ tiene un error de solo $6.47\%$ siendo un valor aceptable, igualmente la medición de ganancia tiene un error de $8.11\%$ indicando que la predeterminación es correcta, por su parte la medición de impedancia de salida $Z_o$ tiene un error de $0.00\%$ el valor práctico es igual al valor teórico.
    \item Para el modo común el cuadro \ref{tab:med-modelo-dinamico-etapa-diferencial-modo-comun} muestra que la impedancia de entrada $Z_i$ tiene error de $16.80\%$ lo cual es un poco alto, esto es debido a que la resistencia patrón utilizada de $48k\Omega$ dista bastante del valor real $24.50k\Omega$. Por otro lado la medición de impedancia de salida es el mismo que en el modelo dinámico del modo diferencial, con un error de $0.00\%$ y el error de la ganancia es de tan solo $3.23\%$. Por lo cual queda que los valores prácticos corresponden con los valores teóricos.
    \item Al observar las señales de la ilustracións \ref{ilus:max-excursion-mod-diff} y \ref{ilus:max-excursion-mod-comun} podemos observar que la señal de entrada para la máxima excursión en el modo diferencial es de $1V$ mientras que la máxima excursión en modo común es de 6V pico.

\end{itemize}
\FloatBarrier
\subsection{Práctica 3}

\subsubsection{Puntos de operación amplificador multietapas desacoplado}

\begin{table}[h!]
\centering
\begin{tabular}{|c|c|c|c|c|c|c|c|c|c|}
\hline
\textbf{Transistor} & \textbf{\(Vc[V]\)} & \textbf{\(\varDelta Vc[V]\)} & \textbf{\(Vb[V]\)} & \textbf{\(\varDelta Vb[V]\)} & \textbf{\(Ve[V]\)} & \textbf{\(\varDelta Ve[V]\)} & \textbf{\(Re[\Omega]\)} & \textbf{\(\varDelta Re[\Omega]\)} \\ \hline
Q1 & 7.2 & 0.4 & -0.04 & 0.002 & -0.6 & 0.04 & 4700 & 470 \\ \hline
Q2 & 7.6 & 0.4 & -0.068 & 0.004 & -0.64 & 0.04 & 4700 & 470 \\ \hline
Q3 & 8 & 0.4 & 8 & 1 & 7.4 & 0.4 & 6800 & 680 \\ \hline
Q4 & 0.68 & 0.04 & -0.6 & 0.04 & -0.6 & 0.04 & 5000 & 500 \\ \hline
Q5 & 10 & 1 & 0.68 & 0.04 & 0.2 & 0.01 & 20 & 1 \\ \hline
Q6 & -10 & 1 & -0.56 & 0.04 & -0.2 & 0.01 & 20 & 1 \\ \hline
\end{tabular}
\caption{Mediciones voltaje DC de los transistores en el multietapas desacoplado}
\label{tab:med-voltaje-dc-transistores-multietapas-desacoplado}
\end{table}

\begin{table}[h!]
\centering
\begin{tabular}{|c|c|c|c|c|c|c|}
\hline
\textbf{Parámetro} & \textbf{Transistor} & \textbf{Valor Teórico} & \textbf{Medición} & \textbf{Incertidumbre} & \textbf{Error Absoluto} & \textbf{Error Relativo} \\ \hline
$I_{c}$ & Q1 & 0.00062 & 0.000595745 & 0.000236773 & 0.00002426 & 3.91\% \\ \hline
$I_{c}$ & Q2 & 0.00062 & 0.000510638 & 0.000234776 & 0.00010936 & 17.64\% \\ \hline
$I_{c}$ & Q3 & -0.00237 & -0.002647059 & 0.000308473 & 0.00027706 & 11.69\% \\ \hline
$I_{c}$ & Q4 & $0.30\times10^{-04}$ & 0 & 1.13137E-05 & 0.00030236 & 100.00\% \\ \hline
$I_{c}$ & Q5 & $0.35\times10^{-03}$ & 0.02 & 0.001224745 & 0.01965000 & 5614.29\% \\ \hline
$I_{c}$ & Q6 & $0.35\times10^{-03}$ & -0.02 & 0.001224745 & 0.02035000 & 5814.29\% \\ \hline
$V_{ce}$ & Q1 & 7.79 & 7.8 & 0.401995025 & 0.01000000 & 0.13\% \\ \hline
$V_{ce}$ & Q2 & 7.79 & 8.24 & 0.401995025 & 0.45000000 & 5.78\% \\ \hline
$V_{ce}$ & Q3 & 2.27 & 0.6 & 0.565685425 & 1.67000000 & 73.57\% \\ \hline
$V_{ce}$ & Q4 & 1.24 & 1.28 & 0.056568542 & 0.04000000 & 3.23\% \\ \hline
$V_{ce}$ & Q5 & 9.99 & 9.8 & 1.000049999 & 0.19000000 & 1.90\% \\ \hline
$V_{ce}$ & Q6 & -9.99 & -9.8 & 1.000049999 & 0.19000000 & 1.90\% \\ \hline
\end{tabular}
\caption{Puntos estáticos de operación transistor multietapas desacoplado}
\label{tab:med-puntos-estaticos-operacion-transistor-multietapas-desacoplado}
\end{table}

\FloatBarrier
\subsubsection{modelo dinámico etapa impulsora}

\begin{table}[h!]
\centering
\begin{tabular}{|c|c|c|c|}
\hline
\textbf{\(Vi[V]\)} & \textbf{\(\varDelta Vi[V]\)} & \textbf{\(Vo[V]\)} & \textbf{\(\varDelta Vo[V]\)} \\ \hline
0.48 & 0.04 & 10 & 1 \\ \hline
\end{tabular}
\caption{Ganancia etapa impulsora}
\label{tab:med-ganancia-etapa-impulsora}
\end{table}

\begin{table}[h!]
\centering
\begin{tabular}{|c|c|c|c|c|c|}
\hline
\textbf{Parámetro} & \textbf{Valor} & \textbf{Medición} & \textbf{Incertidumbre} & \textbf{Error Absoluto} & \textbf{Error Relativo} \\ \hline
$[Z] i$ & 2310 & & & & \\ \hline
$[Z] o$ & 6800 & & & & \\ \hline
$[A]$ & 619.9 & 20.8333333333333 & 2.711892249 & 599.0666667 & 96.64\% \\ \hline
\end{tabular}
\caption{Modelo dinámico etapa impulsora}
\label{tab:med-modelo-dinamico-etapa-impulsora}
\end{table}

\begin{ilustracion}
    \centering
    \includegraphics[width=0.8\textwidth]{src/images/resultados/p3/ganancia-etapa-impulsora.png}
    \caption{Ganancia etapa impulsora}
    \label{ilus:ganancia-etapa-impulsora}
\end{ilustracion}

\begin{ilustracion}
    \centering
    \includegraphics[width=0.8\textwidth]{src/images/resultados/p3/max-excursion-etapa-impulsora.png}
    \caption{Máxima excursión etapa impulsora}
    \label{ilus:max-excursion-etapa-impulsora}
\end{ilustracion}

\FloatBarrier
\subsubsection{Puntos de operación amplificador multietapas acoplado}

\begin{table}[h!]
\centering
\begin{tabular}{|c|c|c|c|c|c|c|c|c|c|}
\hline
\textbf{Transistor} & \textbf{\(Vc[V]\)} & \textbf{\(\varDelta Vc[V]\)} & \textbf{\(Vb[V]\)} & \textbf{\(\varDelta Vb[V]\)} & \textbf{\(Ve[V]\)} & \textbf{\(\varDelta Ve[V]\)} & \textbf{\(Re[\Omega]\)} & \textbf{\(\varDelta Re[\Omega]\)} \\ \hline
Q1 & 7.2 & 0.4 & -0.02 & 0.01 & -0.8 & 0.04 & 4700 & 235 \\ \hline
Q2 & 7.6 & 0.4 & -0.06 & 0.01 & -0.64 & 0.04 & 4700 & 235 \\ \hline
Q3 & 8 & 1 & 8 & 1 & 9 & 1 & 6800 & 340 \\ \hline
Q4 & 10 & 1 & 2 & 0.2 & 3.6 & 0.4 & 5000 & 500 \\ \hline
Q5 & 10 & 1 & 3.6 & 0.4 & 2.6 & 0.2 & 20 & 1 \\ \hline
Q6 & -10 & 1 & 2 & 0.2 & -2 & 0.2 & 20 & 1 \\ \hline
\end{tabular}
\caption{Mediciones de voltaje DC transistores en amplificador multietapas acoplado}
\label{tab:med-mediciones-voltaje-dc-transistores-amplificador-multietapas-acoplado}
\end{table}

\begin{table}[h!]
\centering
\begin{tabular}{|c|c|c|c|c|c|c|}
\hline
\textbf{Parámetro} & \textbf{Transistor} & \textbf{Valor Teórico} & \textbf{Medición} & \textbf{Incertidumbre} & \textbf{Error Absoluto} & \textbf{Error Relativo} \\ \hline
$I_{c}$ & Q1 & 0.00062 & 0.000595745 & 0.000231084 & 0.00002426 & 3.91\% \\ \hline
$I_{c}$ & Q2 & 0.00062 & 0.000510638 & 0.000230574 & 0.00010936 & 17.64\% \\ \hline
$I_{c}$ & Q3 & -0.00237 & -0.002647059 & 0.000246516 & 0.00027706 & 11.69\% \\ \hline
$I_{c}$ & Q4 & 3.02E-04 & 0.00032 & 9.49947E-05 & 0.00001764 & 5.83\% \\ \hline
$I_{c}$ & Q5 & 3.50E-04 & 0.23 & 0.018227726 & 0.22965000 & 65614.29\% \\ \hline
$I_{c}$ & Q6 & 3.50E-04 & 0.23 & 0.018227726 & 0.22965000 & 65614.29\% \\ \hline
$V_{ce}$ & Q1 & 7.79 & 8 & 0.401995025 & 0.21000000 & 2.70\% \\ \hline
$V_{ce}$ & Q2 & 7.79 & 8.24 & 0.401995025 & 0.45000000 & 5.78\% \\ \hline
$V_{ce}$ & Q3 & 2.27 & 1 & 1.414213562 & 1.27000000 & 55.95\% \\ \hline
$V_{ce}$ & Q4 & 1.24 & 6.4 & 1.077032961 & 5.16000000 & 416.13\% \\ \hline
$V_{ce}$ & Q5 & 9.99 & 7.4 & 1.019803903 & 2.59000000 & 25.93\% \\ \hline
$V_{ce}$ & Q6 & -9.99 & -8 & 1.019803903 & 1.99000000 & 19.92\% \\ \hline
\end{tabular}
\caption{Puntos estáticos de operación transistor multietapas acoplado}
\label{tab:med-puntos-estaticos-operacion-transistor-multietapas-acoplado}
\end{table}

\FloatBarrier
\subsubsection{modelo dinámico amplificador multietapas modo diferencial}

\begin{table}[h!]
\centering
\begin{tabular}{|c|c|c|c|c|c|}
\hline
\textbf{\(Vg[V]\)} & \textbf{\(\varDelta Vg[V]\)} & \textbf{\(Vi[V]\)} & \textbf{\(\varDelta Vi[V]\)} & \textbf{\(Rp[\Omega]\)} & \textbf{\(\varDelta Rp[\Omega]\)} \\ \hline
0.52 & 0.01 & 0.17 & 0.01 & 42000 & 4200 \\ \hline
\end{tabular}
\caption{Mediciones impedancias de entrada circuito multietapas modo diferencial}
\label{tab:med-impedancias-entrada-circuito-multietapas-modo-diferencial}
\end{table}

\begin{table}[h!]
\centering
\begin{tabular}{|c|c|c|c|c|c|}
\hline
\textbf{\(Vo_{sc}[V]\)} & \textbf{\(\varDelta Vo_{sc}[V]\)} & \textbf{\(Vo_{cc}[V]\)} & \textbf{\(\varDelta Vo_{cc}[V]\)} & \textbf{\(Rp[\Omega]\)} & \textbf{\(\varDelta Rp[\Omega]\)} \\ \hline
0.52 & 0.04 & 0.32 & 0.02 & 100 & 5 \\ \hline
\end{tabular}
\caption{Mediciones impedancia de salida amplifiador multietapa}
\label{tab:med-impedancia-salida-amplifiador-multietapa}
\end{table}

\begin{table}[h!]
\centering
\begin{tabular}{|c|c|c|c|}
\hline
\textbf{\(Vi[V]\)} & \textbf{\(\varDelta Vi[V]\)} & \textbf{\(Vo[V]\)} & \textbf{\(\varDelta Vo[V]\)} \\ \hline
0.0036 & 0.0002 & 0.52 & 0.04 \\ \hline
\end{tabular}
\caption{Ganancia amplificador multietapas modo diferencial}
\label{tab:med-ganancia-amplificador-multietapas-modo-diferencial}
\end{table}


\begin{table}[h!]
\centering
\begin{tabular}{|c|c|c|c|c|c|}
\hline
\textbf{Parámetro} & \textbf{Valor} & \textbf{Medición} & \textbf{Incertidumbre} & \textbf{Error Absoluto} & \textbf{Error Relativo} \\ \hline
$[Z] d$ & 43,990 & 20400 & 2771.263618 & 23590 & 53.63\% \\ \hline
$[Z] o$ & 132 & 62.5 & 16.40625 & 69.5 & 52.65\% \\ \hline
$[A]$ & 379.22 & 144.444444444444 & 13.70592797 & 234.7755556 & 61.91\% \\ \hline
\end{tabular}
\caption{Modelo dinámico amplificador multietapas modo diferencial}
\label{tab:med-modelo-dinamico-amplificador-multietapas-modo-diferencial}
\end{table}

\begin{ilustracion}[ht]
    \centering
    \includegraphics[width=0.8\textwidth]{src/images/resultados/p3/ganancia-multietapas-mod-diff.png}
    \caption{Ganancia del modelo dinámico amplificador multietapas modo diferencial}
    \label{ilus:ganancia-multietapas-mod-diff}
\end{ilustracion}

\begin{ilustracion}[ht]
    \centering
    \includegraphics[width=0.8\textwidth]{src/images/resultados/p3/max-excursion-multietapas-mod-diff.png}
    \caption{Máxima excursión del modelo dinámico amplificador multietapas modo diferencial}
    \label{ilus:med-max-excursion-multietapas-mod-diff}
\end{ilustracion}

\FloatBarrier
\subsubsection{modelo dinámico amplificador multietapas modo común}

\begin{table}[h!]
\centering
\begin{tabular}{|c|c|c|c|c|c|}
\hline
\textbf{\(Vg[V]\)} & \textbf{\(\varDelta Vg[V]\)} & \textbf{\(Vi[V]\)} & \textbf{\(\varDelta Vi[V]\)} & \textbf{\(Rp[\Omega]\)} & \textbf{\(\varDelta Rp[\Omega]\)} \\ \hline
0.032 & 0.004 & 0.008 & 0.0004 & 20000 & 1000 \\ \hline
\end{tabular}
\caption{Impedancia de entrada amplificador multietapas modo común}
\label{tab:med-impedancia-entrada-amplificador-multietapas-modo-comun}
\end{table}


\begin{table}[h!]
\centering
\begin{tabular}{|c|c|c|c|}
\hline
\textbf{\(Vi[V]\)} & \textbf{\(\varDelta Vi[V]\)} & \textbf{\(Vo[V]\)} & \textbf{\(\varDelta Vo[V]\)} \\ \hline
0.024 & 0.002 & 0.38 & 0.02 \\ \hline
\end{tabular}
\caption{Ganancia amplificador multietapas modo común}
\label{tab:med-ganancia-amplificador-multietapas-modo-comun}
\end{table}

\begin{table}[h!]
\centering
\begin{tabular}{|c|c|c|c|c|c|}
\hline
\textbf{Parámetro} & \textbf{Valor} & \textbf{Medición} & \textbf{Incertidumbre} & \textbf{Error Absoluto} & \textbf{Error Relativo} \\ \hline
$[Z] d$ & 49000 & 41142.85714 & 5974.907966 & 7857.142857 & 16.03\% \\ \hline
$[Z] o$ & 132 & 62.5 & 602.0083575 & 69.5 & 52.65\% \\ \hline
$[A]$ & 40.05 & 15.83333333 & 1.560569795 & 24.21666667 & 60.47\% \\ \hline
\end{tabular}
\caption{Modelo dinámico amplificador multietapas modo común}
\label{tab:med-modelo-dinamico-amplificador-multietapas-modo-comun}
\end{table}

\begin{ilustracion}
    \centering
    \includegraphics[width=0.8\textwidth]{src/images/resultados/p3/ganancia-multietapas-mod-comun.png}
    \caption{Ganancia del modelo dinámico amplificador multietapas modo común}
    \label{ilus:ganancia-multietapas-mod-comun}
\end{ilustracion}

\begin{ilustracion}[ht]
    \centering
    \includegraphics[width=0.8\textwidth]{src/images/resultados/p3/max-excursion-multietapas-mod-comun.png}
    \caption{Máxima excursión del modelo dinámico amplificador multietapas modo común}
    \label{ilus:med-max-excursion-multietapas-mod-comun}
\end{ilustracion}


\FloatBarrier
\subsection{Práctica N° 4}

\subsubsection{Puntos estaticos de operación amplificador multietapas}

\begin{table}[h!]
\centering
\begin{tabular}{|c|c|c|c|c|c|c|c|c|c|}
\hline
\textbf{Transistor} & \textbf{\(Vc[V]\)} & \textbf{\(\varDelta Vc[V]\)} & \textbf{\(Vb[V]\)} & \textbf{\(\varDelta Vb[V]\)} & \textbf{\(Ve[V]\)} & \textbf{\(\varDelta Ve[V]\)} & \textbf{\(Re[\Omega]\)} & \textbf{\(\varDelta Re[\Omega]\)} \\ \hline
Q1 & 7.2 & 0.4 & -0.016 & 0.002 & -0.6 & 0.04 & 4700 & 235 \\ \hline
Q2 & 7.6 & 0.4 & 0.048 & 0.004 & -0.64 & 0.04 & 4700 & 235 \\ \hline
Q3 & 7.6 & 0.4 & 8 & 1 & 9 & 1 & 6800 & 340 \\ \hline
Q4 & 0.68 & 0.04 & 0 & 0.1 & -0.56 & 0.04 & 5000 & 500 \\ \hline
Q5 & 10 & 1 & 0.6 & 0.1 & 0.1 & 0.02 & 20 & 1 \\ \hline
Q6 & -10 & 1 & -0.5 & 0.1 & 0.2 & 0.02 & 20 & 1 \\ \hline
\end{tabular}
\caption{Mediciones de voltaje amplificador multietapas en respuesta en frecuencia}
\label{tab:med-voltaje-amplificador-multietapas-respuesta-frecuencia}
\end{table}


\begin{table}[h!]
\centering
\begin{tabular}{|c|c|c|c|c|c|c|}
\hline
\textbf{Parámetro} & \textbf{Transistor} & \textbf{Valor Teórico} & \textbf{Medición} & \textbf{Incertidumbre} & \textbf{Error Absoluto} & \textbf{Error Relativo} \\ \hline
$I_{c}$ & Q1 & 0.00062 & 0.000595745 & 0.000231084 & 0.00002426 & 3.91\% \\ \hline
$I_{c}$ & Q2 & 0.00062 & 0.000510638 & 0.000230574 & 0.00010936 & 17.64\% \\ \hline
$I_{c}$ & Q3 & -0.00237 & -0.002588235 & 0.000204533 & 0.00021824 & 9.21\% \\ \hline
$I_{c}$ & Q4 & 3.02E-04 & 0.000112 & 2.42784E-05 & 0.00019036 & 62.96\% \\ \hline
$I_{c}$ & Q5 & 3.50E-04 & 0.005 & 0.001436141 & 0.00465000 & 1328.57\% \\ \hline
$I_{c}$ & Q6 & 3.50E-04 & 0.005 & 0.001436141 & 0.00465000 & 1328.57\% \\ \hline
$V_{ce}$ & Q1 & 7.79 & 7.8 & 0.401995025 & 0.01000000 & 0.13\% \\ \hline
$V_{ce}$ & Q2 & 7.79 & 8.24 & 0.401995025 & 0.45000000 & 5.78\% \\ \hline
$V_{ce}$ & Q3 & 2.27 & 1.4 & 1.077032961 & 0.87000000 & 38.33\% \\ \hline
$V_{ce}$ & Q4 & 1.24 & 1.24 & 0.056568542 & 0.00000000 & 0.00\% \\ \hline
$V_{ce}$ & Q5 & 9.99 & 9.9 & 1.00019998 & 0.09000000 & 0.90\% \\ \hline
$V_{ce}$ & Q6 & -9.99 & -10.2 & 1.00019998 & 0.21000000 & 2.10\% \\ \hline
\end{tabular}
\caption{Puntos estáticos de operación amplificador multietapa para respuesta en frecuencia}
\label{tab:med-puntos-estaticos-operacion-amplificador-multietapa-respuesta-frecuencia}
\end{table}


\FloatBarrier
\subsubsection{Respuesta en frecuencia amplificador multietapas}

\begin{table}[h!]
\centering
\begin{tabular}{|c|c|c|c|c|c|c|}
\hline
\textbf{N} & \textbf{\(Vi[V]\)} & \textbf{\(\varDelta Vi[V]\)} & \textbf{\(Vo[V]\)} & \textbf{\(\varDelta Vo[V]\)} & \textbf{\(T\)} & \textbf{\(\varDelta T\)} \\ \hline
1 & 0.0032 & 0.0004 & 0.8 & 0.1 & 0.001 & 0.00004 \\ \hline
2 & 0.0032 & 0.0004 & 0.56 & 0.04 & 8.4E-05 & 0.002 \\ \hline
3 & 0.0032 & 0.0004 & 0.48 & 0.04 & 7.20046E-05 & 0.002 \\ \hline
4 & 0.0032 & 0.0004 & 0.56 & 0.04 & 0.014400922 & 0.0004 \\ \hline
5 & 0.0032 & 0.0004 & 0.36 & 0.04 & 0.0330033 & 0.001 \\ \hline
6 & 0.0032 & 0.0004 & 0.1 & 0.04 & 0.107991361 & 0.004 \\ \hline
7 & 0.0032 & 0.0004 & 0.6 & 0.04 & 0.011599582 & 0.0004 \\ \hline
8 & 0.0032 & 0.0004 & 0.76 & 0.04 & 0.005 & 0.0002 \\ \hline
9 & 0.0032 & 0.0004 & 0.68 & 0.04 & 0.00016 & 0.00001 \\ \hline
10 & 0.0032 & 0.0004 & 0.6 & 0.04 & 0.0001 & 0.000004 \\ \hline
11 & 0.0032 & 0.0004 & 0.48 & 0.04 & 7E-05 & 0.0000024 \\ \hline
\\ 
\\ \hline
\textbf{N} & \textbf{\(A\)} & \textbf{\(\varDelta A\)} & \textbf{\(A[dB]\)} & \textbf{\(\varDelta A[dB]\)} & \textbf{\(f[Hz]\)} & \textbf{\(\varDelta f[Hz]\)} \\ \hline
1 & 250 & 44.19417382 & 47.95880017 & 1.535462866 & 1000 & 40 \\ \hline
2 & 175 & 25.19455546 & 44.86076097 & 1.250497876 & 11904.76 & 283446.6213 \\ \hline
3 & 150 & 22.53469547 & 43.52182518 & 1.304892519 & 13888 & 385753.088 \\ \hline
4 & 175 & 25.19455546 & 44.86076097 & 1.250497876 & 69.44 & 1.92876544 \\ \hline
5 & 112.5 & 18.81499153 & 41.02305045 & 1.452666133 & 30.3 & 0.91809 \\ \hline
6 & 31.25 & 13.09613642 & 29.89700043 & 3.640051059 & 9.26 & 0.3429904 \\ \hline
7 & 187.5 & 26.5625 & 45.46002544 & 1.230501032 & 86.21 & 2.97286564 \\ \hline
8 & 237.5 & 32.2117627 & 47.51327228 & 1.178053962 & 200 & 8 \\ \hline
9 & 212.5 & 29.35670973 & 46.54717869 & 1.199948898 & 6250 & 390.625 \\ \hline
10 & 187.5 & 26.5625 & 45.46002544 & 1.230501032 & 10000 & 400 \\ \hline
11 & 150 & 22.53469547 & 43.52182518 & 1.304892519 & 14285.71 & 489.7956245 \\ \hline
\end{tabular}
\caption{Mediciones respuesta en frecuencia amplificador multietapas acoplado por condensadores}
\label{tab:med-respuesta-frecuencia-amplificador-multietapas-acoplado-condensadores}
\end{table}

\begin{table}[h!]
\centering
\begin{tabular}{|c|c|c|c|c|c|}
\hline
\textbf{Parámetro} & \textbf{Valor Teórico} & \textbf{Medición} & \textbf{Incertidumbre} & \textbf{Error Absoluto} & \textbf{Error Relativo} \\ \hline
$F_L$ [Hz] & 70.41 & 69.44 & 1.92876544 & 0.97000000 & 1.38\% \\ \hline
$H_H$ [Hz] & 10890 & 11904.76 & 283446.6213 & 1014.76000000 & 9.32\% \\ \hline
\end{tabular}
\caption{Medición de frecuencias de corte}
\label{tab:med-frecuencias-corte}
\end{table}


\begin{ilustracion}[ht]
    \centering
    \includegraphics[width=0.8\textwidth]{src/images/resultados/p4/respuesta en frecuencia practica 4 con condensadores.png}
    \caption{Respuesta en frecuencia amplificador multietapas acoplado por condensadores}
    \label{ilus:respuesta-frecuencia-amplificador-multietapas-acoplado-condensadores}
\end{ilustracion}

\begin{ilustracion}[ht]
    \centering
    \includegraphics[width=0.8\textwidth]{src/images/resultados/p4/superposicion respuesta en frecuencia con condensadores.png}
    \caption{Superposición de respuesta en frecuencia amplificador multietapas acoplado por condensadores con su simulación}
    \label{ilus:superposicion-respuesta-frecuencia-amplificador-multietapas-acoplado-condensadores}
\end{ilustracion}

\FloatBarrier
\subsubsection{Respuesta en frecuencia amplificador multietapas sin condensadores de acople}

\begin{table}[h!]
\centering
\begin{tabular}{|c|c|c|c|c|c|c|}
\hline
\textbf{N} & \textbf{\(Vi[V]\)} & \textbf{\(\varDelta Vi[V]\)} & \textbf{\(Vo[V]\)} & \textbf{\(\varDelta Vo[V]\)} & \textbf{\(T\)} & \textbf{\(\varDelta T\)} \\ \hline
1 & 1 & 0.1 & 0.026 & 0.002 & 0.001 & 0.00004 \\ \hline
2 & 1 & 0.1 & 0.0072 & 0.0004 & 0.006024096 & 0.0002 \\ \hline
3 & 1 & 0.1 & 0.01 & 0.001 & 0.003003003 & 0.0001 \\ \hline
4 & 1 & 0.1 & 0.015 & 0.001 & 0.002304147 & 0.0001 \\ \hline
5 & 1 & 0.1 & 0.024 & 0.002 & 0.001499993 & 0.0001 \\ \hline
6 & 1 & 0.1 & 0.034 & 0.002 & 0.00076 & 0.00004 \\ \hline
7 & 1 & 0.1 & 0.052 & 0.004 & 0.0005 & 0.00002 \\ \hline
8 & 1 & 0.1 & 0.09 & 0.01 & 0.00025 & 0.00001 \\ \hline
9 & 1 & 0.1 & 0.14 & 0.01 & 0.0001 & 0.000004 \\ \hline
10 & 1 & 0.1 & 0.15 & 0.01 & 5.99988E-05 & 0.000004 \\ \hline
11 & 1 & 0.1 & 0.15 & 0.01 & 0.00005 & 0.000002 \\ \hline
12 & 1 & 0.1 & 0.15 & 0.01 & 0.0001 & 0.00000014 \\ \hline
&&&&&&\\ \hline
\textbf{N} & \textbf{\(A\)} & \textbf{\(\varDelta A\)} & \textbf{\(A[dB]\)} & \textbf{\(\varDelta A[dB]\)} & \textbf{\(f[Hz]\)} & \textbf{\(\varDelta f[Hz]\)} \\ \hline
1 & 0.026 & 0.003280244 & -31.70053304 & 1.095839863 & 1000 & 40 \\ \hline
2 & 0.0072 & 0.00082365 & -42.85335007 & 0.99363008 & 166 & 5.5112 \\ \hline
3 & 0.01 & 0.001414214 & -40 & 1.228370293 & 333 & 11.0889 \\ \hline
4 & 0.015 & 0.001802776 & -36.47817482 & 1.043914015 & 434 & 18.8356 \\ \hline
5 & 0.024 & 0.0031241 & -32.39577517 & 1.130649446 & 666.67 & 44.44488889 \\ \hline
6 & 0.034 & 0.003944617 & -29.37042166 & 1.007720715 & 1315.79 & 69.25213296 \\ \hline
7 & 0.052 & 0.006560488 & -25.67993313 & 1.095839863 & 2000 & 80 \\ \hline
8 & 0.09 & 0.013453624 & -20.91514981 & 1.298407708 & 4000 & 160 \\ \hline
9 & 0.14 & 0.017204651 & -17.07743929 & 1.067412113 & 10000 & 400 \\ \hline
10 & 0.15 & 0.018027756 & -16.47817482 & 1.043914015 & 16667 & 1111.155556 \\ \hline
11 & 0.15 & 0.018027756 & -16.47817482 & 1.043914015 & 20000 & 800 \\ \hline
12 & 0.15 & 0.018027756 & -16.47817482 & 1.043914015 & 10000 & 14 \\ \hline
\end{tabular}
\caption{Mediciones respuesta en frecuencia amplificador multietapas sin condensadores de acople}
\label{tab:med-respuesta-frecuencia-amplificador-multietapas-sin-condensadores-acople}
\end{table}

\begin{ilustracion}[ht]
    \centering
    \includegraphics[width=0.8\textwidth]{src/images/resultados/p4/respuesta en frecuencia practica 4 sin condensadores de acople.png}
    \caption{Respuesta en frecuencia amplificador multietapas acoplado sin condensadores}
    \label{ilus:respuesta-frecuencia-amplificador-multietapas-acoplado-sin-condensadores}
\end{ilustracion}

\FloatBarrier

La figura \ref{fig:amplificador-base} muestra el modelo dinámico del amplificador base. Cuyos parámetros son los mostrados en la tabla \ref{tab:amplificador-base-dinamico}.

\begin{figure}[ht]
    \centering
    \includegraphics[width=0.6\textwidth]{src/images/p5/modelo-amplificador.png}
    \caption{Modelo dinámico del amplificador base}
    \label{fig:amplificador-base}
\end{figure}

\begin{table}[ht]
    \centering
    \begin{tabular}{|c|c|c|c|}
        \hline
        \textbf{Parámetro} & \textbf{Valor} \\ \hline
        $Z_d$ & $43.99$ [$k\Omega$] \\ \hline
        $Z_o$ & $12.13 $ [$\Omega$] \\ \hline
        $A_b$ & $300$ \\ \hline
        $f_L$ & $69.61$ [$Hz$] \\ \hline
        $f_H$ & $11.00$ [$kHz$] \\ \hline
    \end{tabular}
    \caption{Valores de los parámetros dinámicos del amplificador Realimentado}
    \label{tab:amplificador-base-dinamico}

\end{table}

Ahora calculamos los parámetros del amplificador realimentado negativamente.

para la impedancia de entrada $Z_i$ tenemos:

$$Z_i = R_s = 3.3 k\Omega$$

Y la impedancia de salida viene dada por la expresión:

$$Z_o = \frac{R_o}{A / (1 + \frac{R_f}{R_s})}$$

Por lo tanto el valor de $Z_o$ es:

$$Z_o = 0.137 \Omega$$

El valor de la ganancia de la realimentación negativa es: 

$$A_{fb} = - \frac{R_f}{R_s} = - \frac{11k\Omega}{3.3k\Omega} = -3.33$$

Y debido a que $A => \infty$ en el amplificador base, el valor de la ganancia de la realimentación negativa es:

$$A = -\frac{1}{\beta}$$

despejando $\beta$ de la expresión anterior, tenemos:

$$\beta = -\frac{1}{A} = -\frac{1}{3.33} = 0.333$$

Para encontrar las frecuencias de corte inferior utilizamos la expresión:

$$f_{Lf} = \frac{f_{Lb}}{1 + A_{b}}$$

entonces:

$$f_{Hf} = \frac{69.61 Hz}{1 + 49.54} = 1.37 Hz$$

Para hallar la frecuencia de corte superior utilizamos la expresión:

$$f_{Hf} = f_{Hb}\cdot (1 + A_{b})$$

Por lo tanto:

$$f_{Hf} = 11kHz\cdot (1 + 49.54) = 555.94KHz$$

Ahora, para el amplificador realimentado positivamente, procedemos a calcular la ganancia:

$$A_{fb} = 1 + \frac{R_f}{R_s} = 1 + \frac{11k\Omega}{3.3k\Omega} = 4.33$$

La impedancia de entrada con realimentación positiva es:

$$ Z_i = \frac{Z_d \cdot A_b}{1 + \frac{R_f}{R_s}} = \frac{43.99k\Omega \cdot 300}{1 + \frac{11k\Omega}{3.3k\Omega}} = 3.05M\Omega$$

Y la impedancia de salida con realimentación positiva es igual a la impedancia de salida de realimentación negativa:

$$Z_o = 0.137 \Omega$$

\section{Simulaciones}

La figura \ref{fig:amplificador-realimentado-negativo} muestra el circuito del amplificador realimentado negativamente construido en multisim.

\begin{figure}[ht]
    \centering
    \includegraphics[width=1.0\textwidth]{src/images/p5/Prelaboratorio 5 - Realimentación negativa - circuito.png}
    \caption{Circuito amplificador con realimentación negativa}
    \label{fig:amplificador-realimentado-negativo}
\end{figure}
\FloatBarrier

En la figura \ref{fig:ganancia-realimentacion-negativa} podemos observar una ganancia de aproximadamente $3.3$ que coincide con la ganancia de los cálculos.

\begin{figure}[ht]
    \centering
    \includegraphics[width=1\textwidth]{src/images/p5/Prelaboratorio 5 - Retroalimentación negativa - Ganancia.png}
    \caption{Ganancia de la realimentación negativa}
    \label{fig:ganancia-realimentacion-negativa}
\end{figure}

En la figura \ref{fig:respuesta-frecuencia-realimentacion-negativa} podemos observar un aumento en el ancho de banda con realimentación negativa. Podemos observar que las frecuencias de corte coinciden con los cálculados previamente.

\begin{figure}[ht]
    \centering
    \includegraphics[width=1.0\textwidth]{src/images/p5/Prelaboratorio 5 - retroalimentación negativa - respuesta en frecuencia.png}
    \caption{Respuesta en frecuencia del amplificador realimentado negativamente}
    \label{fig:respuesta-frecuencia-realimentacion-negativa}
\end{figure}
\FloatBarrier

% Realimentación positiva

La figura \ref{fig:circuito-realimentacion-positiva-sin-condensador} muestra la construcción del circuito del amplificador realimentado positivamente.

\begin{figure}[ht]
    \centering
    \includegraphics[width=\textwidth]{src/images/p5/Prelaboratorio 5 - Realimentacion positiva sin condensador - circuito.png}
    \caption{Circuito de realimentación positiva sin condensador}
    \label{fig:circuito-realimentacion-positiva-sin-condensador}
\end{figure}

La ganancia de este amplificador se puede observar en la figura del \ref{fig:ganancia-circuito-realimentacion-positiva-sin-condensador} y podemos observar que coincide con la ganancia calculada anteriormente de 4.33. Sin embargo podemos observar que despues de un tiempo la ganancia cambia y toma la forma mostrada en la figura \ref{fig:ganancia-circuito-realimentacion-positiva-sin-condensador-despues-de-unos-segundos}.

\begin{figure}[ht]
    \centering
    \includegraphics[width=\textwidth]{src/images/p5/Prelaboratorio 5 - Realimentacion positiva sin condensador - ganancia.png}
    \caption{Ganancia del circuito de realimentación positiva sin condensador}
    \label{fig:ganancia-circuito-realimentacion-positiva-sin-condensador}
\end{figure}
\FloatBarrier


\begin{figure}[ht]
    \centering
    \includegraphics[width=\textwidth]{src/images/p5/Prelaboratorio 5 - Retroalimentación positiva sin condensador despues de unos segundos.png}
    \caption{Ganancia del circuito de realimentación positiva sin condensador despues de unos segundos}
    \label{fig:ganancia-circuito-realimentacion-positiva-sin-condensador-despues-de-unos-segundos}
\end{figure}
\FloatBarrier

Podemos observar la respuesta en frecuencia del amplificador realimentado positivamente en la figura \ref{fig:respuesta-amplificador-realimentacion-positiva-sin-condensador}.

\begin{figure}[ht]
    \centering
    \includegraphics[width=\textwidth]{src/images/p5/Prelaboratorio 5 - realimentacion positiva sin condensador - respuesta en frecuencia.png}
    \caption{Respuesta en frecuencia del amplificador con Realimentación positiva sin condensador} 
    \label{fig:respuesta-amplificador-realimentacion-positiva-sin-condensador}
\end{figure}

Amplificador realimentado positiva y negativamente

La figura \ref{fig:circuito-amplificador-realimentacion-positiva-y-negativa} muestra el circuito del amplificador con realimentación positiva y negativa con condensador.

\begin{figure}[ht]
    \centering
    \includegraphics[width=\textwidth]{src/images/p5/prelaboratorio 5 - circuito realimentación positiva con condensador.png}
    \caption{Circuito con realimentación positiva y negativa} 
    \label{fig:circuito-amplificador-realimentacion-positiva-y-negativa}
\end{figure}
\FloatBarrier

La ganancia este amplificador se puede observar en la figura \ref{fig:ganancia-amplificador-realimentacion-positiva-y-negativa}.
\begin{figure}[ht]
    \centering
    \includegraphics[width=\textwidth]{src/images/p5/Prelaboratorio 5 - Ganancia realimentacion positiva con condensador.png}
    \caption{Ganancia con realimentación positiva y negativa} 
    \label{fig:ganancia-amplificador-realimentacion-positiva-y-negativa}
\end{figure}
\FloatBarrier

La respuesta en frecuencia del amplificador se puede observar en la figura \ref{fig:respuesta-amplificador-realimentacion-positiva-y-negativa}.

\begin{figure}[ht]
    \centering
    \includegraphics[width=\textwidth]{src/images/p5/Prelaboratorio 5 - Respuesta en frecuencia - realimentacion positiva con condensador.png}
    \caption{Circuito con realimentación positiva y negativa} 
    \label{fig:respuesta-amplificador-realimentacion-positiva-y-negativa}
\end{figure}

\FloatBarrier
\subsection{Amplificador operacional real}

\section{Presentación de resultados}

\subsection{Análisis de la etapa de potencia}


\begin{itemize}
    \item Al medir los puntos de operación en la etapa de potencia podemos observar que el máximo error en la medición de voltaje $V_{ce}$ es de tan solo $4.84\%$ por lo que nuestra predeterminación para este valor es correcto. Por otro lado podemos observar que el error en la medición $I_c$ está entre 60\% y 529\%, estos errores son grandísimos, esto se puede deber a que dichas corrientes de colector son muy pequeñas.
    \item Para el modelo dinámico del amplificador podemos observar que la impedancia de entrada $Z_i$ tiene un error de solo $7.15\%$ siendo un valor aceptable, igualmente la medición de ganancia tiene un error de $3.85\%$ indicando que la predeterminación es correcta, por su parte la medición de impedancia de salida $Z_o$ tiene un error de $91.16\%$ lo cual es muy alto. El valor de la medición es de 11, lo cual se acerca más al valor de la resistencia $r_{13} = 100\Omega$.
    \item Al modificar la etapa para que se comporte como un amplificador clase C podemos observar que los voltajes $V_{ce}$ medidos corresponden con los teóricos, mientras que las corrientes $I_e$ medidas siguen teniendo un error altísimos. No se observan cambios notables entre el punto de operación del amplificador clase C y el AB. 
    \item En la ilustración \ref{ilus:efecto-crossover-amplificador-clase-c} podemos observar el efecto crossover del amplificador clase C el cual consiste en una pequeña distorsión en la zona de cruce de la onde. No se persive ningún cambio en la ganancia con respecto al otro modo. .
\end{itemize}
\FloatBarrier
\subsection{Análisis de la etapa diferencial}


\begin{itemize}
    \item Observando el punto estático de operación del cuadro \ref{tab:med-puntos-reposo-etapa-diferencial} se observa que el error en la medición $V_{ce}$ es de 0.13\% y el de $I_c$ es de 3.91\%. Por lo tanto la predeterminación realizada para este valor es correcta.
    \item Para el modelo dinámico del amplificador en la etapa diferencial del cuadro \ref{tab:med-modelo-dinamico-etapa-diferencial-modo-diferencial} podemos observar que la impedancia de entrada $Z_i$ tiene un error de solo $6.47\%$ siendo un valor aceptable, igualmente la medición de ganancia tiene un error de $8.11\%$ indicando que la predeterminación es correcta, por su parte la medición de impedancia de salida $Z_o$ tiene un error de $0.00\%$ el valor práctico es igual al valor teórico.
    \item Para el modo común el cuadro \ref{tab:med-modelo-dinamico-etapa-diferencial-modo-comun} muestra que la impedancia de entrada $Z_i$ tiene error de $16.80\%$ lo cual es un poco alto, esto es debido a que la resistencia patrón utilizada de $48k\Omega$ dista bastante del valor real $24.50k\Omega$. Por otro lado la medición de impedancia de salida es el mismo que en el modelo dinámico del modo diferencial, con un error de $0.00\%$ y el error de la ganancia es de tan solo $3.23\%$. Por lo cual queda que los valores prácticos corresponden con los valores teóricos.
    \item Al observar las señales de la ilustracións \ref{ilus:max-excursion-mod-diff} y \ref{ilus:max-excursion-mod-comun} podemos observar que la señal de entrada para la máxima excursión en el modo diferencial es de $1V$ mientras que la máxima excursión en modo común es de 6V pico.

\end{itemize}
\FloatBarrier
\subsection{Práctica 3}

\subsubsection{Puntos de operación amplificador multietapas desacoplado}

\begin{table}[h!]
\centering
\begin{tabular}{|c|c|c|c|c|c|c|c|c|c|}
\hline
\textbf{Transistor} & \textbf{\(Vc[V]\)} & \textbf{\(\varDelta Vc[V]\)} & \textbf{\(Vb[V]\)} & \textbf{\(\varDelta Vb[V]\)} & \textbf{\(Ve[V]\)} & \textbf{\(\varDelta Ve[V]\)} & \textbf{\(Re[\Omega]\)} & \textbf{\(\varDelta Re[\Omega]\)} \\ \hline
Q1 & 7.2 & 0.4 & -0.04 & 0.002 & -0.6 & 0.04 & 4700 & 470 \\ \hline
Q2 & 7.6 & 0.4 & -0.068 & 0.004 & -0.64 & 0.04 & 4700 & 470 \\ \hline
Q3 & 8 & 0.4 & 8 & 1 & 7.4 & 0.4 & 6800 & 680 \\ \hline
Q4 & 0.68 & 0.04 & -0.6 & 0.04 & -0.6 & 0.04 & 5000 & 500 \\ \hline
Q5 & 10 & 1 & 0.68 & 0.04 & 0.2 & 0.01 & 20 & 1 \\ \hline
Q6 & -10 & 1 & -0.56 & 0.04 & -0.2 & 0.01 & 20 & 1 \\ \hline
\end{tabular}
\caption{Mediciones voltaje DC de los transistores en el multietapas desacoplado}
\label{tab:med-voltaje-dc-transistores-multietapas-desacoplado}
\end{table}

\begin{table}[h!]
\centering
\begin{tabular}{|c|c|c|c|c|c|c|}
\hline
\textbf{Parámetro} & \textbf{Transistor} & \textbf{Valor Teórico} & \textbf{Medición} & \textbf{Incertidumbre} & \textbf{Error Absoluto} & \textbf{Error Relativo} \\ \hline
$I_{c}$ & Q1 & 0.00062 & 0.000595745 & 0.000236773 & 0.00002426 & 3.91\% \\ \hline
$I_{c}$ & Q2 & 0.00062 & 0.000510638 & 0.000234776 & 0.00010936 & 17.64\% \\ \hline
$I_{c}$ & Q3 & -0.00237 & -0.002647059 & 0.000308473 & 0.00027706 & 11.69\% \\ \hline
$I_{c}$ & Q4 & $0.30\times10^{-04}$ & 0 & 1.13137E-05 & 0.00030236 & 100.00\% \\ \hline
$I_{c}$ & Q5 & $0.35\times10^{-03}$ & 0.02 & 0.001224745 & 0.01965000 & 5614.29\% \\ \hline
$I_{c}$ & Q6 & $0.35\times10^{-03}$ & -0.02 & 0.001224745 & 0.02035000 & 5814.29\% \\ \hline
$V_{ce}$ & Q1 & 7.79 & 7.8 & 0.401995025 & 0.01000000 & 0.13\% \\ \hline
$V_{ce}$ & Q2 & 7.79 & 8.24 & 0.401995025 & 0.45000000 & 5.78\% \\ \hline
$V_{ce}$ & Q3 & 2.27 & 0.6 & 0.565685425 & 1.67000000 & 73.57\% \\ \hline
$V_{ce}$ & Q4 & 1.24 & 1.28 & 0.056568542 & 0.04000000 & 3.23\% \\ \hline
$V_{ce}$ & Q5 & 9.99 & 9.8 & 1.000049999 & 0.19000000 & 1.90\% \\ \hline
$V_{ce}$ & Q6 & -9.99 & -9.8 & 1.000049999 & 0.19000000 & 1.90\% \\ \hline
\end{tabular}
\caption{Puntos estáticos de operación transistor multietapas desacoplado}
\label{tab:med-puntos-estaticos-operacion-transistor-multietapas-desacoplado}
\end{table}

\FloatBarrier
\subsubsection{modelo dinámico etapa impulsora}

\begin{table}[h!]
\centering
\begin{tabular}{|c|c|c|c|}
\hline
\textbf{\(Vi[V]\)} & \textbf{\(\varDelta Vi[V]\)} & \textbf{\(Vo[V]\)} & \textbf{\(\varDelta Vo[V]\)} \\ \hline
0.48 & 0.04 & 10 & 1 \\ \hline
\end{tabular}
\caption{Ganancia etapa impulsora}
\label{tab:med-ganancia-etapa-impulsora}
\end{table}

\begin{table}[h!]
\centering
\begin{tabular}{|c|c|c|c|c|c|}
\hline
\textbf{Parámetro} & \textbf{Valor} & \textbf{Medición} & \textbf{Incertidumbre} & \textbf{Error Absoluto} & \textbf{Error Relativo} \\ \hline
$[Z] i$ & 2310 & & & & \\ \hline
$[Z] o$ & 6800 & & & & \\ \hline
$[A]$ & 619.9 & 20.8333333333333 & 2.711892249 & 599.0666667 & 96.64\% \\ \hline
\end{tabular}
\caption{Modelo dinámico etapa impulsora}
\label{tab:med-modelo-dinamico-etapa-impulsora}
\end{table}

\begin{ilustracion}
    \centering
    \includegraphics[width=0.8\textwidth]{src/images/resultados/p3/ganancia-etapa-impulsora.png}
    \caption{Ganancia etapa impulsora}
    \label{ilus:ganancia-etapa-impulsora}
\end{ilustracion}

\begin{ilustracion}
    \centering
    \includegraphics[width=0.8\textwidth]{src/images/resultados/p3/max-excursion-etapa-impulsora.png}
    \caption{Máxima excursión etapa impulsora}
    \label{ilus:max-excursion-etapa-impulsora}
\end{ilustracion}

\FloatBarrier
\subsubsection{Puntos de operación amplificador multietapas acoplado}

\begin{table}[h!]
\centering
\begin{tabular}{|c|c|c|c|c|c|c|c|c|c|}
\hline
\textbf{Transistor} & \textbf{\(Vc[V]\)} & \textbf{\(\varDelta Vc[V]\)} & \textbf{\(Vb[V]\)} & \textbf{\(\varDelta Vb[V]\)} & \textbf{\(Ve[V]\)} & \textbf{\(\varDelta Ve[V]\)} & \textbf{\(Re[\Omega]\)} & \textbf{\(\varDelta Re[\Omega]\)} \\ \hline
Q1 & 7.2 & 0.4 & -0.02 & 0.01 & -0.8 & 0.04 & 4700 & 235 \\ \hline
Q2 & 7.6 & 0.4 & -0.06 & 0.01 & -0.64 & 0.04 & 4700 & 235 \\ \hline
Q3 & 8 & 1 & 8 & 1 & 9 & 1 & 6800 & 340 \\ \hline
Q4 & 10 & 1 & 2 & 0.2 & 3.6 & 0.4 & 5000 & 500 \\ \hline
Q5 & 10 & 1 & 3.6 & 0.4 & 2.6 & 0.2 & 20 & 1 \\ \hline
Q6 & -10 & 1 & 2 & 0.2 & -2 & 0.2 & 20 & 1 \\ \hline
\end{tabular}
\caption{Mediciones de voltaje DC transistores en amplificador multietapas acoplado}
\label{tab:med-mediciones-voltaje-dc-transistores-amplificador-multietapas-acoplado}
\end{table}

\begin{table}[h!]
\centering
\begin{tabular}{|c|c|c|c|c|c|c|}
\hline
\textbf{Parámetro} & \textbf{Transistor} & \textbf{Valor Teórico} & \textbf{Medición} & \textbf{Incertidumbre} & \textbf{Error Absoluto} & \textbf{Error Relativo} \\ \hline
$I_{c}$ & Q1 & 0.00062 & 0.000595745 & 0.000231084 & 0.00002426 & 3.91\% \\ \hline
$I_{c}$ & Q2 & 0.00062 & 0.000510638 & 0.000230574 & 0.00010936 & 17.64\% \\ \hline
$I_{c}$ & Q3 & -0.00237 & -0.002647059 & 0.000246516 & 0.00027706 & 11.69\% \\ \hline
$I_{c}$ & Q4 & 3.02E-04 & 0.00032 & 9.49947E-05 & 0.00001764 & 5.83\% \\ \hline
$I_{c}$ & Q5 & 3.50E-04 & 0.23 & 0.018227726 & 0.22965000 & 65614.29\% \\ \hline
$I_{c}$ & Q6 & 3.50E-04 & 0.23 & 0.018227726 & 0.22965000 & 65614.29\% \\ \hline
$V_{ce}$ & Q1 & 7.79 & 8 & 0.401995025 & 0.21000000 & 2.70\% \\ \hline
$V_{ce}$ & Q2 & 7.79 & 8.24 & 0.401995025 & 0.45000000 & 5.78\% \\ \hline
$V_{ce}$ & Q3 & 2.27 & 1 & 1.414213562 & 1.27000000 & 55.95\% \\ \hline
$V_{ce}$ & Q4 & 1.24 & 6.4 & 1.077032961 & 5.16000000 & 416.13\% \\ \hline
$V_{ce}$ & Q5 & 9.99 & 7.4 & 1.019803903 & 2.59000000 & 25.93\% \\ \hline
$V_{ce}$ & Q6 & -9.99 & -8 & 1.019803903 & 1.99000000 & 19.92\% \\ \hline
\end{tabular}
\caption{Puntos estáticos de operación transistor multietapas acoplado}
\label{tab:med-puntos-estaticos-operacion-transistor-multietapas-acoplado}
\end{table}

\FloatBarrier
\subsubsection{modelo dinámico amplificador multietapas modo diferencial}

\begin{table}[h!]
\centering
\begin{tabular}{|c|c|c|c|c|c|}
\hline
\textbf{\(Vg[V]\)} & \textbf{\(\varDelta Vg[V]\)} & \textbf{\(Vi[V]\)} & \textbf{\(\varDelta Vi[V]\)} & \textbf{\(Rp[\Omega]\)} & \textbf{\(\varDelta Rp[\Omega]\)} \\ \hline
0.52 & 0.01 & 0.17 & 0.01 & 42000 & 4200 \\ \hline
\end{tabular}
\caption{Mediciones impedancias de entrada circuito multietapas modo diferencial}
\label{tab:med-impedancias-entrada-circuito-multietapas-modo-diferencial}
\end{table}

\begin{table}[h!]
\centering
\begin{tabular}{|c|c|c|c|c|c|}
\hline
\textbf{\(Vo_{sc}[V]\)} & \textbf{\(\varDelta Vo_{sc}[V]\)} & \textbf{\(Vo_{cc}[V]\)} & \textbf{\(\varDelta Vo_{cc}[V]\)} & \textbf{\(Rp[\Omega]\)} & \textbf{\(\varDelta Rp[\Omega]\)} \\ \hline
0.52 & 0.04 & 0.32 & 0.02 & 100 & 5 \\ \hline
\end{tabular}
\caption{Mediciones impedancia de salida amplifiador multietapa}
\label{tab:med-impedancia-salida-amplifiador-multietapa}
\end{table}

\begin{table}[h!]
\centering
\begin{tabular}{|c|c|c|c|}
\hline
\textbf{\(Vi[V]\)} & \textbf{\(\varDelta Vi[V]\)} & \textbf{\(Vo[V]\)} & \textbf{\(\varDelta Vo[V]\)} \\ \hline
0.0036 & 0.0002 & 0.52 & 0.04 \\ \hline
\end{tabular}
\caption{Ganancia amplificador multietapas modo diferencial}
\label{tab:med-ganancia-amplificador-multietapas-modo-diferencial}
\end{table}


\begin{table}[h!]
\centering
\begin{tabular}{|c|c|c|c|c|c|}
\hline
\textbf{Parámetro} & \textbf{Valor} & \textbf{Medición} & \textbf{Incertidumbre} & \textbf{Error Absoluto} & \textbf{Error Relativo} \\ \hline
$[Z] d$ & 43,990 & 20400 & 2771.263618 & 23590 & 53.63\% \\ \hline
$[Z] o$ & 132 & 62.5 & 16.40625 & 69.5 & 52.65\% \\ \hline
$[A]$ & 379.22 & 144.444444444444 & 13.70592797 & 234.7755556 & 61.91\% \\ \hline
\end{tabular}
\caption{Modelo dinámico amplificador multietapas modo diferencial}
\label{tab:med-modelo-dinamico-amplificador-multietapas-modo-diferencial}
\end{table}

\begin{ilustracion}[ht]
    \centering
    \includegraphics[width=0.8\textwidth]{src/images/resultados/p3/ganancia-multietapas-mod-diff.png}
    \caption{Ganancia del modelo dinámico amplificador multietapas modo diferencial}
    \label{ilus:ganancia-multietapas-mod-diff}
\end{ilustracion}

\begin{ilustracion}[ht]
    \centering
    \includegraphics[width=0.8\textwidth]{src/images/resultados/p3/max-excursion-multietapas-mod-diff.png}
    \caption{Máxima excursión del modelo dinámico amplificador multietapas modo diferencial}
    \label{ilus:med-max-excursion-multietapas-mod-diff}
\end{ilustracion}

\FloatBarrier
\subsubsection{modelo dinámico amplificador multietapas modo común}

\begin{table}[h!]
\centering
\begin{tabular}{|c|c|c|c|c|c|}
\hline
\textbf{\(Vg[V]\)} & \textbf{\(\varDelta Vg[V]\)} & \textbf{\(Vi[V]\)} & \textbf{\(\varDelta Vi[V]\)} & \textbf{\(Rp[\Omega]\)} & \textbf{\(\varDelta Rp[\Omega]\)} \\ \hline
0.032 & 0.004 & 0.008 & 0.0004 & 20000 & 1000 \\ \hline
\end{tabular}
\caption{Impedancia de entrada amplificador multietapas modo común}
\label{tab:med-impedancia-entrada-amplificador-multietapas-modo-comun}
\end{table}


\begin{table}[h!]
\centering
\begin{tabular}{|c|c|c|c|}
\hline
\textbf{\(Vi[V]\)} & \textbf{\(\varDelta Vi[V]\)} & \textbf{\(Vo[V]\)} & \textbf{\(\varDelta Vo[V]\)} \\ \hline
0.024 & 0.002 & 0.38 & 0.02 \\ \hline
\end{tabular}
\caption{Ganancia amplificador multietapas modo común}
\label{tab:med-ganancia-amplificador-multietapas-modo-comun}
\end{table}

\begin{table}[h!]
\centering
\begin{tabular}{|c|c|c|c|c|c|}
\hline
\textbf{Parámetro} & \textbf{Valor} & \textbf{Medición} & \textbf{Incertidumbre} & \textbf{Error Absoluto} & \textbf{Error Relativo} \\ \hline
$[Z] d$ & 49000 & 41142.85714 & 5974.907966 & 7857.142857 & 16.03\% \\ \hline
$[Z] o$ & 132 & 62.5 & 602.0083575 & 69.5 & 52.65\% \\ \hline
$[A]$ & 40.05 & 15.83333333 & 1.560569795 & 24.21666667 & 60.47\% \\ \hline
\end{tabular}
\caption{Modelo dinámico amplificador multietapas modo común}
\label{tab:med-modelo-dinamico-amplificador-multietapas-modo-comun}
\end{table}

\begin{ilustracion}
    \centering
    \includegraphics[width=0.8\textwidth]{src/images/resultados/p3/ganancia-multietapas-mod-comun.png}
    \caption{Ganancia del modelo dinámico amplificador multietapas modo común}
    \label{ilus:ganancia-multietapas-mod-comun}
\end{ilustracion}

\begin{ilustracion}[ht]
    \centering
    \includegraphics[width=0.8\textwidth]{src/images/resultados/p3/max-excursion-multietapas-mod-comun.png}
    \caption{Máxima excursión del modelo dinámico amplificador multietapas modo común}
    \label{ilus:med-max-excursion-multietapas-mod-comun}
\end{ilustracion}


\FloatBarrier
\subsection{Práctica N° 4}

\subsubsection{Puntos estaticos de operación amplificador multietapas}

\begin{table}[h!]
\centering
\begin{tabular}{|c|c|c|c|c|c|c|c|c|c|}
\hline
\textbf{Transistor} & \textbf{\(Vc[V]\)} & \textbf{\(\varDelta Vc[V]\)} & \textbf{\(Vb[V]\)} & \textbf{\(\varDelta Vb[V]\)} & \textbf{\(Ve[V]\)} & \textbf{\(\varDelta Ve[V]\)} & \textbf{\(Re[\Omega]\)} & \textbf{\(\varDelta Re[\Omega]\)} \\ \hline
Q1 & 7.2 & 0.4 & -0.016 & 0.002 & -0.6 & 0.04 & 4700 & 235 \\ \hline
Q2 & 7.6 & 0.4 & 0.048 & 0.004 & -0.64 & 0.04 & 4700 & 235 \\ \hline
Q3 & 7.6 & 0.4 & 8 & 1 & 9 & 1 & 6800 & 340 \\ \hline
Q4 & 0.68 & 0.04 & 0 & 0.1 & -0.56 & 0.04 & 5000 & 500 \\ \hline
Q5 & 10 & 1 & 0.6 & 0.1 & 0.1 & 0.02 & 20 & 1 \\ \hline
Q6 & -10 & 1 & -0.5 & 0.1 & 0.2 & 0.02 & 20 & 1 \\ \hline
\end{tabular}
\caption{Mediciones de voltaje amplificador multietapas en respuesta en frecuencia}
\label{tab:med-voltaje-amplificador-multietapas-respuesta-frecuencia}
\end{table}


\begin{table}[h!]
\centering
\begin{tabular}{|c|c|c|c|c|c|c|}
\hline
\textbf{Parámetro} & \textbf{Transistor} & \textbf{Valor Teórico} & \textbf{Medición} & \textbf{Incertidumbre} & \textbf{Error Absoluto} & \textbf{Error Relativo} \\ \hline
$I_{c}$ & Q1 & 0.00062 & 0.000595745 & 0.000231084 & 0.00002426 & 3.91\% \\ \hline
$I_{c}$ & Q2 & 0.00062 & 0.000510638 & 0.000230574 & 0.00010936 & 17.64\% \\ \hline
$I_{c}$ & Q3 & -0.00237 & -0.002588235 & 0.000204533 & 0.00021824 & 9.21\% \\ \hline
$I_{c}$ & Q4 & 3.02E-04 & 0.000112 & 2.42784E-05 & 0.00019036 & 62.96\% \\ \hline
$I_{c}$ & Q5 & 3.50E-04 & 0.005 & 0.001436141 & 0.00465000 & 1328.57\% \\ \hline
$I_{c}$ & Q6 & 3.50E-04 & 0.005 & 0.001436141 & 0.00465000 & 1328.57\% \\ \hline
$V_{ce}$ & Q1 & 7.79 & 7.8 & 0.401995025 & 0.01000000 & 0.13\% \\ \hline
$V_{ce}$ & Q2 & 7.79 & 8.24 & 0.401995025 & 0.45000000 & 5.78\% \\ \hline
$V_{ce}$ & Q3 & 2.27 & 1.4 & 1.077032961 & 0.87000000 & 38.33\% \\ \hline
$V_{ce}$ & Q4 & 1.24 & 1.24 & 0.056568542 & 0.00000000 & 0.00\% \\ \hline
$V_{ce}$ & Q5 & 9.99 & 9.9 & 1.00019998 & 0.09000000 & 0.90\% \\ \hline
$V_{ce}$ & Q6 & -9.99 & -10.2 & 1.00019998 & 0.21000000 & 2.10\% \\ \hline
\end{tabular}
\caption{Puntos estáticos de operación amplificador multietapa para respuesta en frecuencia}
\label{tab:med-puntos-estaticos-operacion-amplificador-multietapa-respuesta-frecuencia}
\end{table}


\FloatBarrier
\subsubsection{Respuesta en frecuencia amplificador multietapas}

\begin{table}[h!]
\centering
\begin{tabular}{|c|c|c|c|c|c|c|}
\hline
\textbf{N} & \textbf{\(Vi[V]\)} & \textbf{\(\varDelta Vi[V]\)} & \textbf{\(Vo[V]\)} & \textbf{\(\varDelta Vo[V]\)} & \textbf{\(T\)} & \textbf{\(\varDelta T\)} \\ \hline
1 & 0.0032 & 0.0004 & 0.8 & 0.1 & 0.001 & 0.00004 \\ \hline
2 & 0.0032 & 0.0004 & 0.56 & 0.04 & 8.4E-05 & 0.002 \\ \hline
3 & 0.0032 & 0.0004 & 0.48 & 0.04 & 7.20046E-05 & 0.002 \\ \hline
4 & 0.0032 & 0.0004 & 0.56 & 0.04 & 0.014400922 & 0.0004 \\ \hline
5 & 0.0032 & 0.0004 & 0.36 & 0.04 & 0.0330033 & 0.001 \\ \hline
6 & 0.0032 & 0.0004 & 0.1 & 0.04 & 0.107991361 & 0.004 \\ \hline
7 & 0.0032 & 0.0004 & 0.6 & 0.04 & 0.011599582 & 0.0004 \\ \hline
8 & 0.0032 & 0.0004 & 0.76 & 0.04 & 0.005 & 0.0002 \\ \hline
9 & 0.0032 & 0.0004 & 0.68 & 0.04 & 0.00016 & 0.00001 \\ \hline
10 & 0.0032 & 0.0004 & 0.6 & 0.04 & 0.0001 & 0.000004 \\ \hline
11 & 0.0032 & 0.0004 & 0.48 & 0.04 & 7E-05 & 0.0000024 \\ \hline
\\ 
\\ \hline
\textbf{N} & \textbf{\(A\)} & \textbf{\(\varDelta A\)} & \textbf{\(A[dB]\)} & \textbf{\(\varDelta A[dB]\)} & \textbf{\(f[Hz]\)} & \textbf{\(\varDelta f[Hz]\)} \\ \hline
1 & 250 & 44.19417382 & 47.95880017 & 1.535462866 & 1000 & 40 \\ \hline
2 & 175 & 25.19455546 & 44.86076097 & 1.250497876 & 11904.76 & 283446.6213 \\ \hline
3 & 150 & 22.53469547 & 43.52182518 & 1.304892519 & 13888 & 385753.088 \\ \hline
4 & 175 & 25.19455546 & 44.86076097 & 1.250497876 & 69.44 & 1.92876544 \\ \hline
5 & 112.5 & 18.81499153 & 41.02305045 & 1.452666133 & 30.3 & 0.91809 \\ \hline
6 & 31.25 & 13.09613642 & 29.89700043 & 3.640051059 & 9.26 & 0.3429904 \\ \hline
7 & 187.5 & 26.5625 & 45.46002544 & 1.230501032 & 86.21 & 2.97286564 \\ \hline
8 & 237.5 & 32.2117627 & 47.51327228 & 1.178053962 & 200 & 8 \\ \hline
9 & 212.5 & 29.35670973 & 46.54717869 & 1.199948898 & 6250 & 390.625 \\ \hline
10 & 187.5 & 26.5625 & 45.46002544 & 1.230501032 & 10000 & 400 \\ \hline
11 & 150 & 22.53469547 & 43.52182518 & 1.304892519 & 14285.71 & 489.7956245 \\ \hline
\end{tabular}
\caption{Mediciones respuesta en frecuencia amplificador multietapas acoplado por condensadores}
\label{tab:med-respuesta-frecuencia-amplificador-multietapas-acoplado-condensadores}
\end{table}

\begin{table}[h!]
\centering
\begin{tabular}{|c|c|c|c|c|c|}
\hline
\textbf{Parámetro} & \textbf{Valor Teórico} & \textbf{Medición} & \textbf{Incertidumbre} & \textbf{Error Absoluto} & \textbf{Error Relativo} \\ \hline
$F_L$ [Hz] & 70.41 & 69.44 & 1.92876544 & 0.97000000 & 1.38\% \\ \hline
$H_H$ [Hz] & 10890 & 11904.76 & 283446.6213 & 1014.76000000 & 9.32\% \\ \hline
\end{tabular}
\caption{Medición de frecuencias de corte}
\label{tab:med-frecuencias-corte}
\end{table}


\begin{ilustracion}[ht]
    \centering
    \includegraphics[width=0.8\textwidth]{src/images/resultados/p4/respuesta en frecuencia practica 4 con condensadores.png}
    \caption{Respuesta en frecuencia amplificador multietapas acoplado por condensadores}
    \label{ilus:respuesta-frecuencia-amplificador-multietapas-acoplado-condensadores}
\end{ilustracion}

\begin{ilustracion}[ht]
    \centering
    \includegraphics[width=0.8\textwidth]{src/images/resultados/p4/superposicion respuesta en frecuencia con condensadores.png}
    \caption{Superposición de respuesta en frecuencia amplificador multietapas acoplado por condensadores con su simulación}
    \label{ilus:superposicion-respuesta-frecuencia-amplificador-multietapas-acoplado-condensadores}
\end{ilustracion}

\FloatBarrier
\subsubsection{Respuesta en frecuencia amplificador multietapas sin condensadores de acople}

\begin{table}[h!]
\centering
\begin{tabular}{|c|c|c|c|c|c|c|}
\hline
\textbf{N} & \textbf{\(Vi[V]\)} & \textbf{\(\varDelta Vi[V]\)} & \textbf{\(Vo[V]\)} & \textbf{\(\varDelta Vo[V]\)} & \textbf{\(T\)} & \textbf{\(\varDelta T\)} \\ \hline
1 & 1 & 0.1 & 0.026 & 0.002 & 0.001 & 0.00004 \\ \hline
2 & 1 & 0.1 & 0.0072 & 0.0004 & 0.006024096 & 0.0002 \\ \hline
3 & 1 & 0.1 & 0.01 & 0.001 & 0.003003003 & 0.0001 \\ \hline
4 & 1 & 0.1 & 0.015 & 0.001 & 0.002304147 & 0.0001 \\ \hline
5 & 1 & 0.1 & 0.024 & 0.002 & 0.001499993 & 0.0001 \\ \hline
6 & 1 & 0.1 & 0.034 & 0.002 & 0.00076 & 0.00004 \\ \hline
7 & 1 & 0.1 & 0.052 & 0.004 & 0.0005 & 0.00002 \\ \hline
8 & 1 & 0.1 & 0.09 & 0.01 & 0.00025 & 0.00001 \\ \hline
9 & 1 & 0.1 & 0.14 & 0.01 & 0.0001 & 0.000004 \\ \hline
10 & 1 & 0.1 & 0.15 & 0.01 & 5.99988E-05 & 0.000004 \\ \hline
11 & 1 & 0.1 & 0.15 & 0.01 & 0.00005 & 0.000002 \\ \hline
12 & 1 & 0.1 & 0.15 & 0.01 & 0.0001 & 0.00000014 \\ \hline
&&&&&&\\ \hline
\textbf{N} & \textbf{\(A\)} & \textbf{\(\varDelta A\)} & \textbf{\(A[dB]\)} & \textbf{\(\varDelta A[dB]\)} & \textbf{\(f[Hz]\)} & \textbf{\(\varDelta f[Hz]\)} \\ \hline
1 & 0.026 & 0.003280244 & -31.70053304 & 1.095839863 & 1000 & 40 \\ \hline
2 & 0.0072 & 0.00082365 & -42.85335007 & 0.99363008 & 166 & 5.5112 \\ \hline
3 & 0.01 & 0.001414214 & -40 & 1.228370293 & 333 & 11.0889 \\ \hline
4 & 0.015 & 0.001802776 & -36.47817482 & 1.043914015 & 434 & 18.8356 \\ \hline
5 & 0.024 & 0.0031241 & -32.39577517 & 1.130649446 & 666.67 & 44.44488889 \\ \hline
6 & 0.034 & 0.003944617 & -29.37042166 & 1.007720715 & 1315.79 & 69.25213296 \\ \hline
7 & 0.052 & 0.006560488 & -25.67993313 & 1.095839863 & 2000 & 80 \\ \hline
8 & 0.09 & 0.013453624 & -20.91514981 & 1.298407708 & 4000 & 160 \\ \hline
9 & 0.14 & 0.017204651 & -17.07743929 & 1.067412113 & 10000 & 400 \\ \hline
10 & 0.15 & 0.018027756 & -16.47817482 & 1.043914015 & 16667 & 1111.155556 \\ \hline
11 & 0.15 & 0.018027756 & -16.47817482 & 1.043914015 & 20000 & 800 \\ \hline
12 & 0.15 & 0.018027756 & -16.47817482 & 1.043914015 & 10000 & 14 \\ \hline
\end{tabular}
\caption{Mediciones respuesta en frecuencia amplificador multietapas sin condensadores de acople}
\label{tab:med-respuesta-frecuencia-amplificador-multietapas-sin-condensadores-acople}
\end{table}

\begin{ilustracion}[ht]
    \centering
    \includegraphics[width=0.8\textwidth]{src/images/resultados/p4/respuesta en frecuencia practica 4 sin condensadores de acople.png}
    \caption{Respuesta en frecuencia amplificador multietapas acoplado sin condensadores}
    \label{ilus:respuesta-frecuencia-amplificador-multietapas-acoplado-sin-condensadores}
\end{ilustracion}

\FloatBarrier

La figura \ref{fig:amplificador-base} muestra el modelo dinámico del amplificador base. Cuyos parámetros son los mostrados en la tabla \ref{tab:amplificador-base-dinamico}.

\begin{figure}[ht]
    \centering
    \includegraphics[width=0.6\textwidth]{src/images/p5/modelo-amplificador.png}
    \caption{Modelo dinámico del amplificador base}
    \label{fig:amplificador-base}
\end{figure}

\begin{table}[ht]
    \centering
    \begin{tabular}{|c|c|c|c|}
        \hline
        \textbf{Parámetro} & \textbf{Valor} \\ \hline
        $Z_d$ & $43.99$ [$k\Omega$] \\ \hline
        $Z_o$ & $12.13 $ [$\Omega$] \\ \hline
        $A_b$ & $300$ \\ \hline
        $f_L$ & $69.61$ [$Hz$] \\ \hline
        $f_H$ & $11.00$ [$kHz$] \\ \hline
    \end{tabular}
    \caption{Valores de los parámetros dinámicos del amplificador Realimentado}
    \label{tab:amplificador-base-dinamico}

\end{table}

Ahora calculamos los parámetros del amplificador realimentado negativamente.

para la impedancia de entrada $Z_i$ tenemos:

$$Z_i = R_s = 3.3 k\Omega$$

Y la impedancia de salida viene dada por la expresión:

$$Z_o = \frac{R_o}{A / (1 + \frac{R_f}{R_s})}$$

Por lo tanto el valor de $Z_o$ es:

$$Z_o = 0.137 \Omega$$

El valor de la ganancia de la realimentación negativa es: 

$$A_{fb} = - \frac{R_f}{R_s} = - \frac{11k\Omega}{3.3k\Omega} = -3.33$$

Y debido a que $A => \infty$ en el amplificador base, el valor de la ganancia de la realimentación negativa es:

$$A = -\frac{1}{\beta}$$

despejando $\beta$ de la expresión anterior, tenemos:

$$\beta = -\frac{1}{A} = -\frac{1}{3.33} = 0.333$$

Para encontrar las frecuencias de corte inferior utilizamos la expresión:

$$f_{Lf} = \frac{f_{Lb}}{1 + A_{b}}$$

entonces:

$$f_{Hf} = \frac{69.61 Hz}{1 + 49.54} = 1.37 Hz$$

Para hallar la frecuencia de corte superior utilizamos la expresión:

$$f_{Hf} = f_{Hb}\cdot (1 + A_{b})$$

Por lo tanto:

$$f_{Hf} = 11kHz\cdot (1 + 49.54) = 555.94KHz$$

Ahora, para el amplificador realimentado positivamente, procedemos a calcular la ganancia:

$$A_{fb} = 1 + \frac{R_f}{R_s} = 1 + \frac{11k\Omega}{3.3k\Omega} = 4.33$$

La impedancia de entrada con realimentación positiva es:

$$ Z_i = \frac{Z_d \cdot A_b}{1 + \frac{R_f}{R_s}} = \frac{43.99k\Omega \cdot 300}{1 + \frac{11k\Omega}{3.3k\Omega}} = 3.05M\Omega$$

Y la impedancia de salida con realimentación positiva es igual a la impedancia de salida de realimentación negativa:

$$Z_o = 0.137 \Omega$$

\section{Simulaciones}

La figura \ref{fig:amplificador-realimentado-negativo} muestra el circuito del amplificador realimentado negativamente construido en multisim.

\begin{figure}[ht]
    \centering
    \includegraphics[width=1.0\textwidth]{src/images/p5/Prelaboratorio 5 - Realimentación negativa - circuito.png}
    \caption{Circuito amplificador con realimentación negativa}
    \label{fig:amplificador-realimentado-negativo}
\end{figure}
\FloatBarrier

En la figura \ref{fig:ganancia-realimentacion-negativa} podemos observar una ganancia de aproximadamente $3.3$ que coincide con la ganancia de los cálculos.

\begin{figure}[ht]
    \centering
    \includegraphics[width=1\textwidth]{src/images/p5/Prelaboratorio 5 - Retroalimentación negativa - Ganancia.png}
    \caption{Ganancia de la realimentación negativa}
    \label{fig:ganancia-realimentacion-negativa}
\end{figure}

En la figura \ref{fig:respuesta-frecuencia-realimentacion-negativa} podemos observar un aumento en el ancho de banda con realimentación negativa. Podemos observar que las frecuencias de corte coinciden con los cálculados previamente.

\begin{figure}[ht]
    \centering
    \includegraphics[width=1.0\textwidth]{src/images/p5/Prelaboratorio 5 - retroalimentación negativa - respuesta en frecuencia.png}
    \caption{Respuesta en frecuencia del amplificador realimentado negativamente}
    \label{fig:respuesta-frecuencia-realimentacion-negativa}
\end{figure}
\FloatBarrier

% Realimentación positiva

La figura \ref{fig:circuito-realimentacion-positiva-sin-condensador} muestra la construcción del circuito del amplificador realimentado positivamente.

\begin{figure}[ht]
    \centering
    \includegraphics[width=\textwidth]{src/images/p5/Prelaboratorio 5 - Realimentacion positiva sin condensador - circuito.png}
    \caption{Circuito de realimentación positiva sin condensador}
    \label{fig:circuito-realimentacion-positiva-sin-condensador}
\end{figure}

La ganancia de este amplificador se puede observar en la figura del \ref{fig:ganancia-circuito-realimentacion-positiva-sin-condensador} y podemos observar que coincide con la ganancia calculada anteriormente de 4.33. Sin embargo podemos observar que despues de un tiempo la ganancia cambia y toma la forma mostrada en la figura \ref{fig:ganancia-circuito-realimentacion-positiva-sin-condensador-despues-de-unos-segundos}.

\begin{figure}[ht]
    \centering
    \includegraphics[width=\textwidth]{src/images/p5/Prelaboratorio 5 - Realimentacion positiva sin condensador - ganancia.png}
    \caption{Ganancia del circuito de realimentación positiva sin condensador}
    \label{fig:ganancia-circuito-realimentacion-positiva-sin-condensador}
\end{figure}
\FloatBarrier


\begin{figure}[ht]
    \centering
    \includegraphics[width=\textwidth]{src/images/p5/Prelaboratorio 5 - Retroalimentación positiva sin condensador despues de unos segundos.png}
    \caption{Ganancia del circuito de realimentación positiva sin condensador despues de unos segundos}
    \label{fig:ganancia-circuito-realimentacion-positiva-sin-condensador-despues-de-unos-segundos}
\end{figure}
\FloatBarrier

Podemos observar la respuesta en frecuencia del amplificador realimentado positivamente en la figura \ref{fig:respuesta-amplificador-realimentacion-positiva-sin-condensador}.

\begin{figure}[ht]
    \centering
    \includegraphics[width=\textwidth]{src/images/p5/Prelaboratorio 5 - realimentacion positiva sin condensador - respuesta en frecuencia.png}
    \caption{Respuesta en frecuencia del amplificador con Realimentación positiva sin condensador} 
    \label{fig:respuesta-amplificador-realimentacion-positiva-sin-condensador}
\end{figure}

Amplificador realimentado positiva y negativamente

La figura \ref{fig:circuito-amplificador-realimentacion-positiva-y-negativa} muestra el circuito del amplificador con realimentación positiva y negativa con condensador.

\begin{figure}[ht]
    \centering
    \includegraphics[width=\textwidth]{src/images/p5/prelaboratorio 5 - circuito realimentación positiva con condensador.png}
    \caption{Circuito con realimentación positiva y negativa} 
    \label{fig:circuito-amplificador-realimentacion-positiva-y-negativa}
\end{figure}
\FloatBarrier

La ganancia este amplificador se puede observar en la figura \ref{fig:ganancia-amplificador-realimentacion-positiva-y-negativa}.
\begin{figure}[ht]
    \centering
    \includegraphics[width=\textwidth]{src/images/p5/Prelaboratorio 5 - Ganancia realimentacion positiva con condensador.png}
    \caption{Ganancia con realimentación positiva y negativa} 
    \label{fig:ganancia-amplificador-realimentacion-positiva-y-negativa}
\end{figure}
\FloatBarrier

La respuesta en frecuencia del amplificador se puede observar en la figura \ref{fig:respuesta-amplificador-realimentacion-positiva-y-negativa}.

\begin{figure}[ht]
    \centering
    \includegraphics[width=\textwidth]{src/images/p5/Prelaboratorio 5 - Respuesta en frecuencia - realimentacion positiva con condensador.png}
    \caption{Circuito con realimentación positiva y negativa} 
    \label{fig:respuesta-amplificador-realimentacion-positiva-y-negativa}
\end{figure}

\FloatBarrier
\subsection{Filtros activos}

\section{Presentación de resultados}

\subsection{Análisis de la etapa de potencia}


\begin{itemize}
    \item Al medir los puntos de operación en la etapa de potencia podemos observar que el máximo error en la medición de voltaje $V_{ce}$ es de tan solo $4.84\%$ por lo que nuestra predeterminación para este valor es correcto. Por otro lado podemos observar que el error en la medición $I_c$ está entre 60\% y 529\%, estos errores son grandísimos, esto se puede deber a que dichas corrientes de colector son muy pequeñas.
    \item Para el modelo dinámico del amplificador podemos observar que la impedancia de entrada $Z_i$ tiene un error de solo $7.15\%$ siendo un valor aceptable, igualmente la medición de ganancia tiene un error de $3.85\%$ indicando que la predeterminación es correcta, por su parte la medición de impedancia de salida $Z_o$ tiene un error de $91.16\%$ lo cual es muy alto. El valor de la medición es de 11, lo cual se acerca más al valor de la resistencia $r_{13} = 100\Omega$.
    \item Al modificar la etapa para que se comporte como un amplificador clase C podemos observar que los voltajes $V_{ce}$ medidos corresponden con los teóricos, mientras que las corrientes $I_e$ medidas siguen teniendo un error altísimos. No se observan cambios notables entre el punto de operación del amplificador clase C y el AB. 
    \item En la ilustración \ref{ilus:efecto-crossover-amplificador-clase-c} podemos observar el efecto crossover del amplificador clase C el cual consiste en una pequeña distorsión en la zona de cruce de la onde. No se persive ningún cambio en la ganancia con respecto al otro modo. .
\end{itemize}
\FloatBarrier
\subsection{Análisis de la etapa diferencial}


\begin{itemize}
    \item Observando el punto estático de operación del cuadro \ref{tab:med-puntos-reposo-etapa-diferencial} se observa que el error en la medición $V_{ce}$ es de 0.13\% y el de $I_c$ es de 3.91\%. Por lo tanto la predeterminación realizada para este valor es correcta.
    \item Para el modelo dinámico del amplificador en la etapa diferencial del cuadro \ref{tab:med-modelo-dinamico-etapa-diferencial-modo-diferencial} podemos observar que la impedancia de entrada $Z_i$ tiene un error de solo $6.47\%$ siendo un valor aceptable, igualmente la medición de ganancia tiene un error de $8.11\%$ indicando que la predeterminación es correcta, por su parte la medición de impedancia de salida $Z_o$ tiene un error de $0.00\%$ el valor práctico es igual al valor teórico.
    \item Para el modo común el cuadro \ref{tab:med-modelo-dinamico-etapa-diferencial-modo-comun} muestra que la impedancia de entrada $Z_i$ tiene error de $16.80\%$ lo cual es un poco alto, esto es debido a que la resistencia patrón utilizada de $48k\Omega$ dista bastante del valor real $24.50k\Omega$. Por otro lado la medición de impedancia de salida es el mismo que en el modelo dinámico del modo diferencial, con un error de $0.00\%$ y el error de la ganancia es de tan solo $3.23\%$. Por lo cual queda que los valores prácticos corresponden con los valores teóricos.
    \item Al observar las señales de la ilustracións \ref{ilus:max-excursion-mod-diff} y \ref{ilus:max-excursion-mod-comun} podemos observar que la señal de entrada para la máxima excursión en el modo diferencial es de $1V$ mientras que la máxima excursión en modo común es de 6V pico.

\end{itemize}
\FloatBarrier
\subsection{Práctica 3}

\subsubsection{Puntos de operación amplificador multietapas desacoplado}

\begin{table}[h!]
\centering
\begin{tabular}{|c|c|c|c|c|c|c|c|c|c|}
\hline
\textbf{Transistor} & \textbf{\(Vc[V]\)} & \textbf{\(\varDelta Vc[V]\)} & \textbf{\(Vb[V]\)} & \textbf{\(\varDelta Vb[V]\)} & \textbf{\(Ve[V]\)} & \textbf{\(\varDelta Ve[V]\)} & \textbf{\(Re[\Omega]\)} & \textbf{\(\varDelta Re[\Omega]\)} \\ \hline
Q1 & 7.2 & 0.4 & -0.04 & 0.002 & -0.6 & 0.04 & 4700 & 470 \\ \hline
Q2 & 7.6 & 0.4 & -0.068 & 0.004 & -0.64 & 0.04 & 4700 & 470 \\ \hline
Q3 & 8 & 0.4 & 8 & 1 & 7.4 & 0.4 & 6800 & 680 \\ \hline
Q4 & 0.68 & 0.04 & -0.6 & 0.04 & -0.6 & 0.04 & 5000 & 500 \\ \hline
Q5 & 10 & 1 & 0.68 & 0.04 & 0.2 & 0.01 & 20 & 1 \\ \hline
Q6 & -10 & 1 & -0.56 & 0.04 & -0.2 & 0.01 & 20 & 1 \\ \hline
\end{tabular}
\caption{Mediciones voltaje DC de los transistores en el multietapas desacoplado}
\label{tab:med-voltaje-dc-transistores-multietapas-desacoplado}
\end{table}

\begin{table}[h!]
\centering
\begin{tabular}{|c|c|c|c|c|c|c|}
\hline
\textbf{Parámetro} & \textbf{Transistor} & \textbf{Valor Teórico} & \textbf{Medición} & \textbf{Incertidumbre} & \textbf{Error Absoluto} & \textbf{Error Relativo} \\ \hline
$I_{c}$ & Q1 & 0.00062 & 0.000595745 & 0.000236773 & 0.00002426 & 3.91\% \\ \hline
$I_{c}$ & Q2 & 0.00062 & 0.000510638 & 0.000234776 & 0.00010936 & 17.64\% \\ \hline
$I_{c}$ & Q3 & -0.00237 & -0.002647059 & 0.000308473 & 0.00027706 & 11.69\% \\ \hline
$I_{c}$ & Q4 & $0.30\times10^{-04}$ & 0 & 1.13137E-05 & 0.00030236 & 100.00\% \\ \hline
$I_{c}$ & Q5 & $0.35\times10^{-03}$ & 0.02 & 0.001224745 & 0.01965000 & 5614.29\% \\ \hline
$I_{c}$ & Q6 & $0.35\times10^{-03}$ & -0.02 & 0.001224745 & 0.02035000 & 5814.29\% \\ \hline
$V_{ce}$ & Q1 & 7.79 & 7.8 & 0.401995025 & 0.01000000 & 0.13\% \\ \hline
$V_{ce}$ & Q2 & 7.79 & 8.24 & 0.401995025 & 0.45000000 & 5.78\% \\ \hline
$V_{ce}$ & Q3 & 2.27 & 0.6 & 0.565685425 & 1.67000000 & 73.57\% \\ \hline
$V_{ce}$ & Q4 & 1.24 & 1.28 & 0.056568542 & 0.04000000 & 3.23\% \\ \hline
$V_{ce}$ & Q5 & 9.99 & 9.8 & 1.000049999 & 0.19000000 & 1.90\% \\ \hline
$V_{ce}$ & Q6 & -9.99 & -9.8 & 1.000049999 & 0.19000000 & 1.90\% \\ \hline
\end{tabular}
\caption{Puntos estáticos de operación transistor multietapas desacoplado}
\label{tab:med-puntos-estaticos-operacion-transistor-multietapas-desacoplado}
\end{table}

\FloatBarrier
\subsubsection{modelo dinámico etapa impulsora}

\begin{table}[h!]
\centering
\begin{tabular}{|c|c|c|c|}
\hline
\textbf{\(Vi[V]\)} & \textbf{\(\varDelta Vi[V]\)} & \textbf{\(Vo[V]\)} & \textbf{\(\varDelta Vo[V]\)} \\ \hline
0.48 & 0.04 & 10 & 1 \\ \hline
\end{tabular}
\caption{Ganancia etapa impulsora}
\label{tab:med-ganancia-etapa-impulsora}
\end{table}

\begin{table}[h!]
\centering
\begin{tabular}{|c|c|c|c|c|c|}
\hline
\textbf{Parámetro} & \textbf{Valor} & \textbf{Medición} & \textbf{Incertidumbre} & \textbf{Error Absoluto} & \textbf{Error Relativo} \\ \hline
$[Z] i$ & 2310 & & & & \\ \hline
$[Z] o$ & 6800 & & & & \\ \hline
$[A]$ & 619.9 & 20.8333333333333 & 2.711892249 & 599.0666667 & 96.64\% \\ \hline
\end{tabular}
\caption{Modelo dinámico etapa impulsora}
\label{tab:med-modelo-dinamico-etapa-impulsora}
\end{table}

\begin{ilustracion}
    \centering
    \includegraphics[width=0.8\textwidth]{src/images/resultados/p3/ganancia-etapa-impulsora.png}
    \caption{Ganancia etapa impulsora}
    \label{ilus:ganancia-etapa-impulsora}
\end{ilustracion}

\begin{ilustracion}
    \centering
    \includegraphics[width=0.8\textwidth]{src/images/resultados/p3/max-excursion-etapa-impulsora.png}
    \caption{Máxima excursión etapa impulsora}
    \label{ilus:max-excursion-etapa-impulsora}
\end{ilustracion}

\FloatBarrier
\subsubsection{Puntos de operación amplificador multietapas acoplado}

\begin{table}[h!]
\centering
\begin{tabular}{|c|c|c|c|c|c|c|c|c|c|}
\hline
\textbf{Transistor} & \textbf{\(Vc[V]\)} & \textbf{\(\varDelta Vc[V]\)} & \textbf{\(Vb[V]\)} & \textbf{\(\varDelta Vb[V]\)} & \textbf{\(Ve[V]\)} & \textbf{\(\varDelta Ve[V]\)} & \textbf{\(Re[\Omega]\)} & \textbf{\(\varDelta Re[\Omega]\)} \\ \hline
Q1 & 7.2 & 0.4 & -0.02 & 0.01 & -0.8 & 0.04 & 4700 & 235 \\ \hline
Q2 & 7.6 & 0.4 & -0.06 & 0.01 & -0.64 & 0.04 & 4700 & 235 \\ \hline
Q3 & 8 & 1 & 8 & 1 & 9 & 1 & 6800 & 340 \\ \hline
Q4 & 10 & 1 & 2 & 0.2 & 3.6 & 0.4 & 5000 & 500 \\ \hline
Q5 & 10 & 1 & 3.6 & 0.4 & 2.6 & 0.2 & 20 & 1 \\ \hline
Q6 & -10 & 1 & 2 & 0.2 & -2 & 0.2 & 20 & 1 \\ \hline
\end{tabular}
\caption{Mediciones de voltaje DC transistores en amplificador multietapas acoplado}
\label{tab:med-mediciones-voltaje-dc-transistores-amplificador-multietapas-acoplado}
\end{table}

\begin{table}[h!]
\centering
\begin{tabular}{|c|c|c|c|c|c|c|}
\hline
\textbf{Parámetro} & \textbf{Transistor} & \textbf{Valor Teórico} & \textbf{Medición} & \textbf{Incertidumbre} & \textbf{Error Absoluto} & \textbf{Error Relativo} \\ \hline
$I_{c}$ & Q1 & 0.00062 & 0.000595745 & 0.000231084 & 0.00002426 & 3.91\% \\ \hline
$I_{c}$ & Q2 & 0.00062 & 0.000510638 & 0.000230574 & 0.00010936 & 17.64\% \\ \hline
$I_{c}$ & Q3 & -0.00237 & -0.002647059 & 0.000246516 & 0.00027706 & 11.69\% \\ \hline
$I_{c}$ & Q4 & 3.02E-04 & 0.00032 & 9.49947E-05 & 0.00001764 & 5.83\% \\ \hline
$I_{c}$ & Q5 & 3.50E-04 & 0.23 & 0.018227726 & 0.22965000 & 65614.29\% \\ \hline
$I_{c}$ & Q6 & 3.50E-04 & 0.23 & 0.018227726 & 0.22965000 & 65614.29\% \\ \hline
$V_{ce}$ & Q1 & 7.79 & 8 & 0.401995025 & 0.21000000 & 2.70\% \\ \hline
$V_{ce}$ & Q2 & 7.79 & 8.24 & 0.401995025 & 0.45000000 & 5.78\% \\ \hline
$V_{ce}$ & Q3 & 2.27 & 1 & 1.414213562 & 1.27000000 & 55.95\% \\ \hline
$V_{ce}$ & Q4 & 1.24 & 6.4 & 1.077032961 & 5.16000000 & 416.13\% \\ \hline
$V_{ce}$ & Q5 & 9.99 & 7.4 & 1.019803903 & 2.59000000 & 25.93\% \\ \hline
$V_{ce}$ & Q6 & -9.99 & -8 & 1.019803903 & 1.99000000 & 19.92\% \\ \hline
\end{tabular}
\caption{Puntos estáticos de operación transistor multietapas acoplado}
\label{tab:med-puntos-estaticos-operacion-transistor-multietapas-acoplado}
\end{table}

\FloatBarrier
\subsubsection{modelo dinámico amplificador multietapas modo diferencial}

\begin{table}[h!]
\centering
\begin{tabular}{|c|c|c|c|c|c|}
\hline
\textbf{\(Vg[V]\)} & \textbf{\(\varDelta Vg[V]\)} & \textbf{\(Vi[V]\)} & \textbf{\(\varDelta Vi[V]\)} & \textbf{\(Rp[\Omega]\)} & \textbf{\(\varDelta Rp[\Omega]\)} \\ \hline
0.52 & 0.01 & 0.17 & 0.01 & 42000 & 4200 \\ \hline
\end{tabular}
\caption{Mediciones impedancias de entrada circuito multietapas modo diferencial}
\label{tab:med-impedancias-entrada-circuito-multietapas-modo-diferencial}
\end{table}

\begin{table}[h!]
\centering
\begin{tabular}{|c|c|c|c|c|c|}
\hline
\textbf{\(Vo_{sc}[V]\)} & \textbf{\(\varDelta Vo_{sc}[V]\)} & \textbf{\(Vo_{cc}[V]\)} & \textbf{\(\varDelta Vo_{cc}[V]\)} & \textbf{\(Rp[\Omega]\)} & \textbf{\(\varDelta Rp[\Omega]\)} \\ \hline
0.52 & 0.04 & 0.32 & 0.02 & 100 & 5 \\ \hline
\end{tabular}
\caption{Mediciones impedancia de salida amplifiador multietapa}
\label{tab:med-impedancia-salida-amplifiador-multietapa}
\end{table}

\begin{table}[h!]
\centering
\begin{tabular}{|c|c|c|c|}
\hline
\textbf{\(Vi[V]\)} & \textbf{\(\varDelta Vi[V]\)} & \textbf{\(Vo[V]\)} & \textbf{\(\varDelta Vo[V]\)} \\ \hline
0.0036 & 0.0002 & 0.52 & 0.04 \\ \hline
\end{tabular}
\caption{Ganancia amplificador multietapas modo diferencial}
\label{tab:med-ganancia-amplificador-multietapas-modo-diferencial}
\end{table}


\begin{table}[h!]
\centering
\begin{tabular}{|c|c|c|c|c|c|}
\hline
\textbf{Parámetro} & \textbf{Valor} & \textbf{Medición} & \textbf{Incertidumbre} & \textbf{Error Absoluto} & \textbf{Error Relativo} \\ \hline
$[Z] d$ & 43,990 & 20400 & 2771.263618 & 23590 & 53.63\% \\ \hline
$[Z] o$ & 132 & 62.5 & 16.40625 & 69.5 & 52.65\% \\ \hline
$[A]$ & 379.22 & 144.444444444444 & 13.70592797 & 234.7755556 & 61.91\% \\ \hline
\end{tabular}
\caption{Modelo dinámico amplificador multietapas modo diferencial}
\label{tab:med-modelo-dinamico-amplificador-multietapas-modo-diferencial}
\end{table}

\begin{ilustracion}[ht]
    \centering
    \includegraphics[width=0.8\textwidth]{src/images/resultados/p3/ganancia-multietapas-mod-diff.png}
    \caption{Ganancia del modelo dinámico amplificador multietapas modo diferencial}
    \label{ilus:ganancia-multietapas-mod-diff}
\end{ilustracion}

\begin{ilustracion}[ht]
    \centering
    \includegraphics[width=0.8\textwidth]{src/images/resultados/p3/max-excursion-multietapas-mod-diff.png}
    \caption{Máxima excursión del modelo dinámico amplificador multietapas modo diferencial}
    \label{ilus:med-max-excursion-multietapas-mod-diff}
\end{ilustracion}

\FloatBarrier
\subsubsection{modelo dinámico amplificador multietapas modo común}

\begin{table}[h!]
\centering
\begin{tabular}{|c|c|c|c|c|c|}
\hline
\textbf{\(Vg[V]\)} & \textbf{\(\varDelta Vg[V]\)} & \textbf{\(Vi[V]\)} & \textbf{\(\varDelta Vi[V]\)} & \textbf{\(Rp[\Omega]\)} & \textbf{\(\varDelta Rp[\Omega]\)} \\ \hline
0.032 & 0.004 & 0.008 & 0.0004 & 20000 & 1000 \\ \hline
\end{tabular}
\caption{Impedancia de entrada amplificador multietapas modo común}
\label{tab:med-impedancia-entrada-amplificador-multietapas-modo-comun}
\end{table}


\begin{table}[h!]
\centering
\begin{tabular}{|c|c|c|c|}
\hline
\textbf{\(Vi[V]\)} & \textbf{\(\varDelta Vi[V]\)} & \textbf{\(Vo[V]\)} & \textbf{\(\varDelta Vo[V]\)} \\ \hline
0.024 & 0.002 & 0.38 & 0.02 \\ \hline
\end{tabular}
\caption{Ganancia amplificador multietapas modo común}
\label{tab:med-ganancia-amplificador-multietapas-modo-comun}
\end{table}

\begin{table}[h!]
\centering
\begin{tabular}{|c|c|c|c|c|c|}
\hline
\textbf{Parámetro} & \textbf{Valor} & \textbf{Medición} & \textbf{Incertidumbre} & \textbf{Error Absoluto} & \textbf{Error Relativo} \\ \hline
$[Z] d$ & 49000 & 41142.85714 & 5974.907966 & 7857.142857 & 16.03\% \\ \hline
$[Z] o$ & 132 & 62.5 & 602.0083575 & 69.5 & 52.65\% \\ \hline
$[A]$ & 40.05 & 15.83333333 & 1.560569795 & 24.21666667 & 60.47\% \\ \hline
\end{tabular}
\caption{Modelo dinámico amplificador multietapas modo común}
\label{tab:med-modelo-dinamico-amplificador-multietapas-modo-comun}
\end{table}

\begin{ilustracion}
    \centering
    \includegraphics[width=0.8\textwidth]{src/images/resultados/p3/ganancia-multietapas-mod-comun.png}
    \caption{Ganancia del modelo dinámico amplificador multietapas modo común}
    \label{ilus:ganancia-multietapas-mod-comun}
\end{ilustracion}

\begin{ilustracion}[ht]
    \centering
    \includegraphics[width=0.8\textwidth]{src/images/resultados/p3/max-excursion-multietapas-mod-comun.png}
    \caption{Máxima excursión del modelo dinámico amplificador multietapas modo común}
    \label{ilus:med-max-excursion-multietapas-mod-comun}
\end{ilustracion}


\FloatBarrier
\subsection{Práctica N° 4}

\subsubsection{Puntos estaticos de operación amplificador multietapas}

\begin{table}[h!]
\centering
\begin{tabular}{|c|c|c|c|c|c|c|c|c|c|}
\hline
\textbf{Transistor} & \textbf{\(Vc[V]\)} & \textbf{\(\varDelta Vc[V]\)} & \textbf{\(Vb[V]\)} & \textbf{\(\varDelta Vb[V]\)} & \textbf{\(Ve[V]\)} & \textbf{\(\varDelta Ve[V]\)} & \textbf{\(Re[\Omega]\)} & \textbf{\(\varDelta Re[\Omega]\)} \\ \hline
Q1 & 7.2 & 0.4 & -0.016 & 0.002 & -0.6 & 0.04 & 4700 & 235 \\ \hline
Q2 & 7.6 & 0.4 & 0.048 & 0.004 & -0.64 & 0.04 & 4700 & 235 \\ \hline
Q3 & 7.6 & 0.4 & 8 & 1 & 9 & 1 & 6800 & 340 \\ \hline
Q4 & 0.68 & 0.04 & 0 & 0.1 & -0.56 & 0.04 & 5000 & 500 \\ \hline
Q5 & 10 & 1 & 0.6 & 0.1 & 0.1 & 0.02 & 20 & 1 \\ \hline
Q6 & -10 & 1 & -0.5 & 0.1 & 0.2 & 0.02 & 20 & 1 \\ \hline
\end{tabular}
\caption{Mediciones de voltaje amplificador multietapas en respuesta en frecuencia}
\label{tab:med-voltaje-amplificador-multietapas-respuesta-frecuencia}
\end{table}


\begin{table}[h!]
\centering
\begin{tabular}{|c|c|c|c|c|c|c|}
\hline
\textbf{Parámetro} & \textbf{Transistor} & \textbf{Valor Teórico} & \textbf{Medición} & \textbf{Incertidumbre} & \textbf{Error Absoluto} & \textbf{Error Relativo} \\ \hline
$I_{c}$ & Q1 & 0.00062 & 0.000595745 & 0.000231084 & 0.00002426 & 3.91\% \\ \hline
$I_{c}$ & Q2 & 0.00062 & 0.000510638 & 0.000230574 & 0.00010936 & 17.64\% \\ \hline
$I_{c}$ & Q3 & -0.00237 & -0.002588235 & 0.000204533 & 0.00021824 & 9.21\% \\ \hline
$I_{c}$ & Q4 & 3.02E-04 & 0.000112 & 2.42784E-05 & 0.00019036 & 62.96\% \\ \hline
$I_{c}$ & Q5 & 3.50E-04 & 0.005 & 0.001436141 & 0.00465000 & 1328.57\% \\ \hline
$I_{c}$ & Q6 & 3.50E-04 & 0.005 & 0.001436141 & 0.00465000 & 1328.57\% \\ \hline
$V_{ce}$ & Q1 & 7.79 & 7.8 & 0.401995025 & 0.01000000 & 0.13\% \\ \hline
$V_{ce}$ & Q2 & 7.79 & 8.24 & 0.401995025 & 0.45000000 & 5.78\% \\ \hline
$V_{ce}$ & Q3 & 2.27 & 1.4 & 1.077032961 & 0.87000000 & 38.33\% \\ \hline
$V_{ce}$ & Q4 & 1.24 & 1.24 & 0.056568542 & 0.00000000 & 0.00\% \\ \hline
$V_{ce}$ & Q5 & 9.99 & 9.9 & 1.00019998 & 0.09000000 & 0.90\% \\ \hline
$V_{ce}$ & Q6 & -9.99 & -10.2 & 1.00019998 & 0.21000000 & 2.10\% \\ \hline
\end{tabular}
\caption{Puntos estáticos de operación amplificador multietapa para respuesta en frecuencia}
\label{tab:med-puntos-estaticos-operacion-amplificador-multietapa-respuesta-frecuencia}
\end{table}


\FloatBarrier
\subsubsection{Respuesta en frecuencia amplificador multietapas}

\begin{table}[h!]
\centering
\begin{tabular}{|c|c|c|c|c|c|c|}
\hline
\textbf{N} & \textbf{\(Vi[V]\)} & \textbf{\(\varDelta Vi[V]\)} & \textbf{\(Vo[V]\)} & \textbf{\(\varDelta Vo[V]\)} & \textbf{\(T\)} & \textbf{\(\varDelta T\)} \\ \hline
1 & 0.0032 & 0.0004 & 0.8 & 0.1 & 0.001 & 0.00004 \\ \hline
2 & 0.0032 & 0.0004 & 0.56 & 0.04 & 8.4E-05 & 0.002 \\ \hline
3 & 0.0032 & 0.0004 & 0.48 & 0.04 & 7.20046E-05 & 0.002 \\ \hline
4 & 0.0032 & 0.0004 & 0.56 & 0.04 & 0.014400922 & 0.0004 \\ \hline
5 & 0.0032 & 0.0004 & 0.36 & 0.04 & 0.0330033 & 0.001 \\ \hline
6 & 0.0032 & 0.0004 & 0.1 & 0.04 & 0.107991361 & 0.004 \\ \hline
7 & 0.0032 & 0.0004 & 0.6 & 0.04 & 0.011599582 & 0.0004 \\ \hline
8 & 0.0032 & 0.0004 & 0.76 & 0.04 & 0.005 & 0.0002 \\ \hline
9 & 0.0032 & 0.0004 & 0.68 & 0.04 & 0.00016 & 0.00001 \\ \hline
10 & 0.0032 & 0.0004 & 0.6 & 0.04 & 0.0001 & 0.000004 \\ \hline
11 & 0.0032 & 0.0004 & 0.48 & 0.04 & 7E-05 & 0.0000024 \\ \hline
\\ 
\\ \hline
\textbf{N} & \textbf{\(A\)} & \textbf{\(\varDelta A\)} & \textbf{\(A[dB]\)} & \textbf{\(\varDelta A[dB]\)} & \textbf{\(f[Hz]\)} & \textbf{\(\varDelta f[Hz]\)} \\ \hline
1 & 250 & 44.19417382 & 47.95880017 & 1.535462866 & 1000 & 40 \\ \hline
2 & 175 & 25.19455546 & 44.86076097 & 1.250497876 & 11904.76 & 283446.6213 \\ \hline
3 & 150 & 22.53469547 & 43.52182518 & 1.304892519 & 13888 & 385753.088 \\ \hline
4 & 175 & 25.19455546 & 44.86076097 & 1.250497876 & 69.44 & 1.92876544 \\ \hline
5 & 112.5 & 18.81499153 & 41.02305045 & 1.452666133 & 30.3 & 0.91809 \\ \hline
6 & 31.25 & 13.09613642 & 29.89700043 & 3.640051059 & 9.26 & 0.3429904 \\ \hline
7 & 187.5 & 26.5625 & 45.46002544 & 1.230501032 & 86.21 & 2.97286564 \\ \hline
8 & 237.5 & 32.2117627 & 47.51327228 & 1.178053962 & 200 & 8 \\ \hline
9 & 212.5 & 29.35670973 & 46.54717869 & 1.199948898 & 6250 & 390.625 \\ \hline
10 & 187.5 & 26.5625 & 45.46002544 & 1.230501032 & 10000 & 400 \\ \hline
11 & 150 & 22.53469547 & 43.52182518 & 1.304892519 & 14285.71 & 489.7956245 \\ \hline
\end{tabular}
\caption{Mediciones respuesta en frecuencia amplificador multietapas acoplado por condensadores}
\label{tab:med-respuesta-frecuencia-amplificador-multietapas-acoplado-condensadores}
\end{table}

\begin{table}[h!]
\centering
\begin{tabular}{|c|c|c|c|c|c|}
\hline
\textbf{Parámetro} & \textbf{Valor Teórico} & \textbf{Medición} & \textbf{Incertidumbre} & \textbf{Error Absoluto} & \textbf{Error Relativo} \\ \hline
$F_L$ [Hz] & 70.41 & 69.44 & 1.92876544 & 0.97000000 & 1.38\% \\ \hline
$H_H$ [Hz] & 10890 & 11904.76 & 283446.6213 & 1014.76000000 & 9.32\% \\ \hline
\end{tabular}
\caption{Medición de frecuencias de corte}
\label{tab:med-frecuencias-corte}
\end{table}


\begin{ilustracion}[ht]
    \centering
    \includegraphics[width=0.8\textwidth]{src/images/resultados/p4/respuesta en frecuencia practica 4 con condensadores.png}
    \caption{Respuesta en frecuencia amplificador multietapas acoplado por condensadores}
    \label{ilus:respuesta-frecuencia-amplificador-multietapas-acoplado-condensadores}
\end{ilustracion}

\begin{ilustracion}[ht]
    \centering
    \includegraphics[width=0.8\textwidth]{src/images/resultados/p4/superposicion respuesta en frecuencia con condensadores.png}
    \caption{Superposición de respuesta en frecuencia amplificador multietapas acoplado por condensadores con su simulación}
    \label{ilus:superposicion-respuesta-frecuencia-amplificador-multietapas-acoplado-condensadores}
\end{ilustracion}

\FloatBarrier
\subsubsection{Respuesta en frecuencia amplificador multietapas sin condensadores de acople}

\begin{table}[h!]
\centering
\begin{tabular}{|c|c|c|c|c|c|c|}
\hline
\textbf{N} & \textbf{\(Vi[V]\)} & \textbf{\(\varDelta Vi[V]\)} & \textbf{\(Vo[V]\)} & \textbf{\(\varDelta Vo[V]\)} & \textbf{\(T\)} & \textbf{\(\varDelta T\)} \\ \hline
1 & 1 & 0.1 & 0.026 & 0.002 & 0.001 & 0.00004 \\ \hline
2 & 1 & 0.1 & 0.0072 & 0.0004 & 0.006024096 & 0.0002 \\ \hline
3 & 1 & 0.1 & 0.01 & 0.001 & 0.003003003 & 0.0001 \\ \hline
4 & 1 & 0.1 & 0.015 & 0.001 & 0.002304147 & 0.0001 \\ \hline
5 & 1 & 0.1 & 0.024 & 0.002 & 0.001499993 & 0.0001 \\ \hline
6 & 1 & 0.1 & 0.034 & 0.002 & 0.00076 & 0.00004 \\ \hline
7 & 1 & 0.1 & 0.052 & 0.004 & 0.0005 & 0.00002 \\ \hline
8 & 1 & 0.1 & 0.09 & 0.01 & 0.00025 & 0.00001 \\ \hline
9 & 1 & 0.1 & 0.14 & 0.01 & 0.0001 & 0.000004 \\ \hline
10 & 1 & 0.1 & 0.15 & 0.01 & 5.99988E-05 & 0.000004 \\ \hline
11 & 1 & 0.1 & 0.15 & 0.01 & 0.00005 & 0.000002 \\ \hline
12 & 1 & 0.1 & 0.15 & 0.01 & 0.0001 & 0.00000014 \\ \hline
&&&&&&\\ \hline
\textbf{N} & \textbf{\(A\)} & \textbf{\(\varDelta A\)} & \textbf{\(A[dB]\)} & \textbf{\(\varDelta A[dB]\)} & \textbf{\(f[Hz]\)} & \textbf{\(\varDelta f[Hz]\)} \\ \hline
1 & 0.026 & 0.003280244 & -31.70053304 & 1.095839863 & 1000 & 40 \\ \hline
2 & 0.0072 & 0.00082365 & -42.85335007 & 0.99363008 & 166 & 5.5112 \\ \hline
3 & 0.01 & 0.001414214 & -40 & 1.228370293 & 333 & 11.0889 \\ \hline
4 & 0.015 & 0.001802776 & -36.47817482 & 1.043914015 & 434 & 18.8356 \\ \hline
5 & 0.024 & 0.0031241 & -32.39577517 & 1.130649446 & 666.67 & 44.44488889 \\ \hline
6 & 0.034 & 0.003944617 & -29.37042166 & 1.007720715 & 1315.79 & 69.25213296 \\ \hline
7 & 0.052 & 0.006560488 & -25.67993313 & 1.095839863 & 2000 & 80 \\ \hline
8 & 0.09 & 0.013453624 & -20.91514981 & 1.298407708 & 4000 & 160 \\ \hline
9 & 0.14 & 0.017204651 & -17.07743929 & 1.067412113 & 10000 & 400 \\ \hline
10 & 0.15 & 0.018027756 & -16.47817482 & 1.043914015 & 16667 & 1111.155556 \\ \hline
11 & 0.15 & 0.018027756 & -16.47817482 & 1.043914015 & 20000 & 800 \\ \hline
12 & 0.15 & 0.018027756 & -16.47817482 & 1.043914015 & 10000 & 14 \\ \hline
\end{tabular}
\caption{Mediciones respuesta en frecuencia amplificador multietapas sin condensadores de acople}
\label{tab:med-respuesta-frecuencia-amplificador-multietapas-sin-condensadores-acople}
\end{table}

\begin{ilustracion}[ht]
    \centering
    \includegraphics[width=0.8\textwidth]{src/images/resultados/p4/respuesta en frecuencia practica 4 sin condensadores de acople.png}
    \caption{Respuesta en frecuencia amplificador multietapas acoplado sin condensadores}
    \label{ilus:respuesta-frecuencia-amplificador-multietapas-acoplado-sin-condensadores}
\end{ilustracion}

\FloatBarrier

La figura \ref{fig:amplificador-base} muestra el modelo dinámico del amplificador base. Cuyos parámetros son los mostrados en la tabla \ref{tab:amplificador-base-dinamico}.

\begin{figure}[ht]
    \centering
    \includegraphics[width=0.6\textwidth]{src/images/p5/modelo-amplificador.png}
    \caption{Modelo dinámico del amplificador base}
    \label{fig:amplificador-base}
\end{figure}

\begin{table}[ht]
    \centering
    \begin{tabular}{|c|c|c|c|}
        \hline
        \textbf{Parámetro} & \textbf{Valor} \\ \hline
        $Z_d$ & $43.99$ [$k\Omega$] \\ \hline
        $Z_o$ & $12.13 $ [$\Omega$] \\ \hline
        $A_b$ & $300$ \\ \hline
        $f_L$ & $69.61$ [$Hz$] \\ \hline
        $f_H$ & $11.00$ [$kHz$] \\ \hline
    \end{tabular}
    \caption{Valores de los parámetros dinámicos del amplificador Realimentado}
    \label{tab:amplificador-base-dinamico}

\end{table}

Ahora calculamos los parámetros del amplificador realimentado negativamente.

para la impedancia de entrada $Z_i$ tenemos:

$$Z_i = R_s = 3.3 k\Omega$$

Y la impedancia de salida viene dada por la expresión:

$$Z_o = \frac{R_o}{A / (1 + \frac{R_f}{R_s})}$$

Por lo tanto el valor de $Z_o$ es:

$$Z_o = 0.137 \Omega$$

El valor de la ganancia de la realimentación negativa es: 

$$A_{fb} = - \frac{R_f}{R_s} = - \frac{11k\Omega}{3.3k\Omega} = -3.33$$

Y debido a que $A => \infty$ en el amplificador base, el valor de la ganancia de la realimentación negativa es:

$$A = -\frac{1}{\beta}$$

despejando $\beta$ de la expresión anterior, tenemos:

$$\beta = -\frac{1}{A} = -\frac{1}{3.33} = 0.333$$

Para encontrar las frecuencias de corte inferior utilizamos la expresión:

$$f_{Lf} = \frac{f_{Lb}}{1 + A_{b}}$$

entonces:

$$f_{Hf} = \frac{69.61 Hz}{1 + 49.54} = 1.37 Hz$$

Para hallar la frecuencia de corte superior utilizamos la expresión:

$$f_{Hf} = f_{Hb}\cdot (1 + A_{b})$$

Por lo tanto:

$$f_{Hf} = 11kHz\cdot (1 + 49.54) = 555.94KHz$$

Ahora, para el amplificador realimentado positivamente, procedemos a calcular la ganancia:

$$A_{fb} = 1 + \frac{R_f}{R_s} = 1 + \frac{11k\Omega}{3.3k\Omega} = 4.33$$

La impedancia de entrada con realimentación positiva es:

$$ Z_i = \frac{Z_d \cdot A_b}{1 + \frac{R_f}{R_s}} = \frac{43.99k\Omega \cdot 300}{1 + \frac{11k\Omega}{3.3k\Omega}} = 3.05M\Omega$$

Y la impedancia de salida con realimentación positiva es igual a la impedancia de salida de realimentación negativa:

$$Z_o = 0.137 \Omega$$

\section{Simulaciones}

La figura \ref{fig:amplificador-realimentado-negativo} muestra el circuito del amplificador realimentado negativamente construido en multisim.

\begin{figure}[ht]
    \centering
    \includegraphics[width=1.0\textwidth]{src/images/p5/Prelaboratorio 5 - Realimentación negativa - circuito.png}
    \caption{Circuito amplificador con realimentación negativa}
    \label{fig:amplificador-realimentado-negativo}
\end{figure}
\FloatBarrier

En la figura \ref{fig:ganancia-realimentacion-negativa} podemos observar una ganancia de aproximadamente $3.3$ que coincide con la ganancia de los cálculos.

\begin{figure}[ht]
    \centering
    \includegraphics[width=1\textwidth]{src/images/p5/Prelaboratorio 5 - Retroalimentación negativa - Ganancia.png}
    \caption{Ganancia de la realimentación negativa}
    \label{fig:ganancia-realimentacion-negativa}
\end{figure}

En la figura \ref{fig:respuesta-frecuencia-realimentacion-negativa} podemos observar un aumento en el ancho de banda con realimentación negativa. Podemos observar que las frecuencias de corte coinciden con los cálculados previamente.

\begin{figure}[ht]
    \centering
    \includegraphics[width=1.0\textwidth]{src/images/p5/Prelaboratorio 5 - retroalimentación negativa - respuesta en frecuencia.png}
    \caption{Respuesta en frecuencia del amplificador realimentado negativamente}
    \label{fig:respuesta-frecuencia-realimentacion-negativa}
\end{figure}
\FloatBarrier

% Realimentación positiva

La figura \ref{fig:circuito-realimentacion-positiva-sin-condensador} muestra la construcción del circuito del amplificador realimentado positivamente.

\begin{figure}[ht]
    \centering
    \includegraphics[width=\textwidth]{src/images/p5/Prelaboratorio 5 - Realimentacion positiva sin condensador - circuito.png}
    \caption{Circuito de realimentación positiva sin condensador}
    \label{fig:circuito-realimentacion-positiva-sin-condensador}
\end{figure}

La ganancia de este amplificador se puede observar en la figura del \ref{fig:ganancia-circuito-realimentacion-positiva-sin-condensador} y podemos observar que coincide con la ganancia calculada anteriormente de 4.33. Sin embargo podemos observar que despues de un tiempo la ganancia cambia y toma la forma mostrada en la figura \ref{fig:ganancia-circuito-realimentacion-positiva-sin-condensador-despues-de-unos-segundos}.

\begin{figure}[ht]
    \centering
    \includegraphics[width=\textwidth]{src/images/p5/Prelaboratorio 5 - Realimentacion positiva sin condensador - ganancia.png}
    \caption{Ganancia del circuito de realimentación positiva sin condensador}
    \label{fig:ganancia-circuito-realimentacion-positiva-sin-condensador}
\end{figure}
\FloatBarrier


\begin{figure}[ht]
    \centering
    \includegraphics[width=\textwidth]{src/images/p5/Prelaboratorio 5 - Retroalimentación positiva sin condensador despues de unos segundos.png}
    \caption{Ganancia del circuito de realimentación positiva sin condensador despues de unos segundos}
    \label{fig:ganancia-circuito-realimentacion-positiva-sin-condensador-despues-de-unos-segundos}
\end{figure}
\FloatBarrier

Podemos observar la respuesta en frecuencia del amplificador realimentado positivamente en la figura \ref{fig:respuesta-amplificador-realimentacion-positiva-sin-condensador}.

\begin{figure}[ht]
    \centering
    \includegraphics[width=\textwidth]{src/images/p5/Prelaboratorio 5 - realimentacion positiva sin condensador - respuesta en frecuencia.png}
    \caption{Respuesta en frecuencia del amplificador con Realimentación positiva sin condensador} 
    \label{fig:respuesta-amplificador-realimentacion-positiva-sin-condensador}
\end{figure}

Amplificador realimentado positiva y negativamente

La figura \ref{fig:circuito-amplificador-realimentacion-positiva-y-negativa} muestra el circuito del amplificador con realimentación positiva y negativa con condensador.

\begin{figure}[ht]
    \centering
    \includegraphics[width=\textwidth]{src/images/p5/prelaboratorio 5 - circuito realimentación positiva con condensador.png}
    \caption{Circuito con realimentación positiva y negativa} 
    \label{fig:circuito-amplificador-realimentacion-positiva-y-negativa}
\end{figure}
\FloatBarrier

La ganancia este amplificador se puede observar en la figura \ref{fig:ganancia-amplificador-realimentacion-positiva-y-negativa}.
\begin{figure}[ht]
    \centering
    \includegraphics[width=\textwidth]{src/images/p5/Prelaboratorio 5 - Ganancia realimentacion positiva con condensador.png}
    \caption{Ganancia con realimentación positiva y negativa} 
    \label{fig:ganancia-amplificador-realimentacion-positiva-y-negativa}
\end{figure}
\FloatBarrier

La respuesta en frecuencia del amplificador se puede observar en la figura \ref{fig:respuesta-amplificador-realimentacion-positiva-y-negativa}.

\begin{figure}[ht]
    \centering
    \includegraphics[width=\textwidth]{src/images/p5/Prelaboratorio 5 - Respuesta en frecuencia - realimentacion positiva con condensador.png}
    \caption{Circuito con realimentación positiva y negativa} 
    \label{fig:respuesta-amplificador-realimentacion-positiva-y-negativa}
\end{figure}

\FloatBarrier
\subsection{Fuentes lineales y reguladores monolíticos}

\section{Presentación de resultados}

\subsection{Análisis de la etapa de potencia}


\begin{itemize}
    \item Al medir los puntos de operación en la etapa de potencia podemos observar que el máximo error en la medición de voltaje $V_{ce}$ es de tan solo $4.84\%$ por lo que nuestra predeterminación para este valor es correcto. Por otro lado podemos observar que el error en la medición $I_c$ está entre 60\% y 529\%, estos errores son grandísimos, esto se puede deber a que dichas corrientes de colector son muy pequeñas.
    \item Para el modelo dinámico del amplificador podemos observar que la impedancia de entrada $Z_i$ tiene un error de solo $7.15\%$ siendo un valor aceptable, igualmente la medición de ganancia tiene un error de $3.85\%$ indicando que la predeterminación es correcta, por su parte la medición de impedancia de salida $Z_o$ tiene un error de $91.16\%$ lo cual es muy alto. El valor de la medición es de 11, lo cual se acerca más al valor de la resistencia $r_{13} = 100\Omega$.
    \item Al modificar la etapa para que se comporte como un amplificador clase C podemos observar que los voltajes $V_{ce}$ medidos corresponden con los teóricos, mientras que las corrientes $I_e$ medidas siguen teniendo un error altísimos. No se observan cambios notables entre el punto de operación del amplificador clase C y el AB. 
    \item En la ilustración \ref{ilus:efecto-crossover-amplificador-clase-c} podemos observar el efecto crossover del amplificador clase C el cual consiste en una pequeña distorsión en la zona de cruce de la onde. No se persive ningún cambio en la ganancia con respecto al otro modo. .
\end{itemize}
\FloatBarrier
\subsection{Análisis de la etapa diferencial}


\begin{itemize}
    \item Observando el punto estático de operación del cuadro \ref{tab:med-puntos-reposo-etapa-diferencial} se observa que el error en la medición $V_{ce}$ es de 0.13\% y el de $I_c$ es de 3.91\%. Por lo tanto la predeterminación realizada para este valor es correcta.
    \item Para el modelo dinámico del amplificador en la etapa diferencial del cuadro \ref{tab:med-modelo-dinamico-etapa-diferencial-modo-diferencial} podemos observar que la impedancia de entrada $Z_i$ tiene un error de solo $6.47\%$ siendo un valor aceptable, igualmente la medición de ganancia tiene un error de $8.11\%$ indicando que la predeterminación es correcta, por su parte la medición de impedancia de salida $Z_o$ tiene un error de $0.00\%$ el valor práctico es igual al valor teórico.
    \item Para el modo común el cuadro \ref{tab:med-modelo-dinamico-etapa-diferencial-modo-comun} muestra que la impedancia de entrada $Z_i$ tiene error de $16.80\%$ lo cual es un poco alto, esto es debido a que la resistencia patrón utilizada de $48k\Omega$ dista bastante del valor real $24.50k\Omega$. Por otro lado la medición de impedancia de salida es el mismo que en el modelo dinámico del modo diferencial, con un error de $0.00\%$ y el error de la ganancia es de tan solo $3.23\%$. Por lo cual queda que los valores prácticos corresponden con los valores teóricos.
    \item Al observar las señales de la ilustracións \ref{ilus:max-excursion-mod-diff} y \ref{ilus:max-excursion-mod-comun} podemos observar que la señal de entrada para la máxima excursión en el modo diferencial es de $1V$ mientras que la máxima excursión en modo común es de 6V pico.

\end{itemize}
\FloatBarrier
\subsection{Práctica 3}

\subsubsection{Puntos de operación amplificador multietapas desacoplado}

\begin{table}[h!]
\centering
\begin{tabular}{|c|c|c|c|c|c|c|c|c|c|}
\hline
\textbf{Transistor} & \textbf{\(Vc[V]\)} & \textbf{\(\varDelta Vc[V]\)} & \textbf{\(Vb[V]\)} & \textbf{\(\varDelta Vb[V]\)} & \textbf{\(Ve[V]\)} & \textbf{\(\varDelta Ve[V]\)} & \textbf{\(Re[\Omega]\)} & \textbf{\(\varDelta Re[\Omega]\)} \\ \hline
Q1 & 7.2 & 0.4 & -0.04 & 0.002 & -0.6 & 0.04 & 4700 & 470 \\ \hline
Q2 & 7.6 & 0.4 & -0.068 & 0.004 & -0.64 & 0.04 & 4700 & 470 \\ \hline
Q3 & 8 & 0.4 & 8 & 1 & 7.4 & 0.4 & 6800 & 680 \\ \hline
Q4 & 0.68 & 0.04 & -0.6 & 0.04 & -0.6 & 0.04 & 5000 & 500 \\ \hline
Q5 & 10 & 1 & 0.68 & 0.04 & 0.2 & 0.01 & 20 & 1 \\ \hline
Q6 & -10 & 1 & -0.56 & 0.04 & -0.2 & 0.01 & 20 & 1 \\ \hline
\end{tabular}
\caption{Mediciones voltaje DC de los transistores en el multietapas desacoplado}
\label{tab:med-voltaje-dc-transistores-multietapas-desacoplado}
\end{table}

\begin{table}[h!]
\centering
\begin{tabular}{|c|c|c|c|c|c|c|}
\hline
\textbf{Parámetro} & \textbf{Transistor} & \textbf{Valor Teórico} & \textbf{Medición} & \textbf{Incertidumbre} & \textbf{Error Absoluto} & \textbf{Error Relativo} \\ \hline
$I_{c}$ & Q1 & 0.00062 & 0.000595745 & 0.000236773 & 0.00002426 & 3.91\% \\ \hline
$I_{c}$ & Q2 & 0.00062 & 0.000510638 & 0.000234776 & 0.00010936 & 17.64\% \\ \hline
$I_{c}$ & Q3 & -0.00237 & -0.002647059 & 0.000308473 & 0.00027706 & 11.69\% \\ \hline
$I_{c}$ & Q4 & $0.30\times10^{-04}$ & 0 & 1.13137E-05 & 0.00030236 & 100.00\% \\ \hline
$I_{c}$ & Q5 & $0.35\times10^{-03}$ & 0.02 & 0.001224745 & 0.01965000 & 5614.29\% \\ \hline
$I_{c}$ & Q6 & $0.35\times10^{-03}$ & -0.02 & 0.001224745 & 0.02035000 & 5814.29\% \\ \hline
$V_{ce}$ & Q1 & 7.79 & 7.8 & 0.401995025 & 0.01000000 & 0.13\% \\ \hline
$V_{ce}$ & Q2 & 7.79 & 8.24 & 0.401995025 & 0.45000000 & 5.78\% \\ \hline
$V_{ce}$ & Q3 & 2.27 & 0.6 & 0.565685425 & 1.67000000 & 73.57\% \\ \hline
$V_{ce}$ & Q4 & 1.24 & 1.28 & 0.056568542 & 0.04000000 & 3.23\% \\ \hline
$V_{ce}$ & Q5 & 9.99 & 9.8 & 1.000049999 & 0.19000000 & 1.90\% \\ \hline
$V_{ce}$ & Q6 & -9.99 & -9.8 & 1.000049999 & 0.19000000 & 1.90\% \\ \hline
\end{tabular}
\caption{Puntos estáticos de operación transistor multietapas desacoplado}
\label{tab:med-puntos-estaticos-operacion-transistor-multietapas-desacoplado}
\end{table}

\FloatBarrier
\subsubsection{modelo dinámico etapa impulsora}

\begin{table}[h!]
\centering
\begin{tabular}{|c|c|c|c|}
\hline
\textbf{\(Vi[V]\)} & \textbf{\(\varDelta Vi[V]\)} & \textbf{\(Vo[V]\)} & \textbf{\(\varDelta Vo[V]\)} \\ \hline
0.48 & 0.04 & 10 & 1 \\ \hline
\end{tabular}
\caption{Ganancia etapa impulsora}
\label{tab:med-ganancia-etapa-impulsora}
\end{table}

\begin{table}[h!]
\centering
\begin{tabular}{|c|c|c|c|c|c|}
\hline
\textbf{Parámetro} & \textbf{Valor} & \textbf{Medición} & \textbf{Incertidumbre} & \textbf{Error Absoluto} & \textbf{Error Relativo} \\ \hline
$[Z] i$ & 2310 & & & & \\ \hline
$[Z] o$ & 6800 & & & & \\ \hline
$[A]$ & 619.9 & 20.8333333333333 & 2.711892249 & 599.0666667 & 96.64\% \\ \hline
\end{tabular}
\caption{Modelo dinámico etapa impulsora}
\label{tab:med-modelo-dinamico-etapa-impulsora}
\end{table}

\begin{ilustracion}
    \centering
    \includegraphics[width=0.8\textwidth]{src/images/resultados/p3/ganancia-etapa-impulsora.png}
    \caption{Ganancia etapa impulsora}
    \label{ilus:ganancia-etapa-impulsora}
\end{ilustracion}

\begin{ilustracion}
    \centering
    \includegraphics[width=0.8\textwidth]{src/images/resultados/p3/max-excursion-etapa-impulsora.png}
    \caption{Máxima excursión etapa impulsora}
    \label{ilus:max-excursion-etapa-impulsora}
\end{ilustracion}

\FloatBarrier
\subsubsection{Puntos de operación amplificador multietapas acoplado}

\begin{table}[h!]
\centering
\begin{tabular}{|c|c|c|c|c|c|c|c|c|c|}
\hline
\textbf{Transistor} & \textbf{\(Vc[V]\)} & \textbf{\(\varDelta Vc[V]\)} & \textbf{\(Vb[V]\)} & \textbf{\(\varDelta Vb[V]\)} & \textbf{\(Ve[V]\)} & \textbf{\(\varDelta Ve[V]\)} & \textbf{\(Re[\Omega]\)} & \textbf{\(\varDelta Re[\Omega]\)} \\ \hline
Q1 & 7.2 & 0.4 & -0.02 & 0.01 & -0.8 & 0.04 & 4700 & 235 \\ \hline
Q2 & 7.6 & 0.4 & -0.06 & 0.01 & -0.64 & 0.04 & 4700 & 235 \\ \hline
Q3 & 8 & 1 & 8 & 1 & 9 & 1 & 6800 & 340 \\ \hline
Q4 & 10 & 1 & 2 & 0.2 & 3.6 & 0.4 & 5000 & 500 \\ \hline
Q5 & 10 & 1 & 3.6 & 0.4 & 2.6 & 0.2 & 20 & 1 \\ \hline
Q6 & -10 & 1 & 2 & 0.2 & -2 & 0.2 & 20 & 1 \\ \hline
\end{tabular}
\caption{Mediciones de voltaje DC transistores en amplificador multietapas acoplado}
\label{tab:med-mediciones-voltaje-dc-transistores-amplificador-multietapas-acoplado}
\end{table}

\begin{table}[h!]
\centering
\begin{tabular}{|c|c|c|c|c|c|c|}
\hline
\textbf{Parámetro} & \textbf{Transistor} & \textbf{Valor Teórico} & \textbf{Medición} & \textbf{Incertidumbre} & \textbf{Error Absoluto} & \textbf{Error Relativo} \\ \hline
$I_{c}$ & Q1 & 0.00062 & 0.000595745 & 0.000231084 & 0.00002426 & 3.91\% \\ \hline
$I_{c}$ & Q2 & 0.00062 & 0.000510638 & 0.000230574 & 0.00010936 & 17.64\% \\ \hline
$I_{c}$ & Q3 & -0.00237 & -0.002647059 & 0.000246516 & 0.00027706 & 11.69\% \\ \hline
$I_{c}$ & Q4 & 3.02E-04 & 0.00032 & 9.49947E-05 & 0.00001764 & 5.83\% \\ \hline
$I_{c}$ & Q5 & 3.50E-04 & 0.23 & 0.018227726 & 0.22965000 & 65614.29\% \\ \hline
$I_{c}$ & Q6 & 3.50E-04 & 0.23 & 0.018227726 & 0.22965000 & 65614.29\% \\ \hline
$V_{ce}$ & Q1 & 7.79 & 8 & 0.401995025 & 0.21000000 & 2.70\% \\ \hline
$V_{ce}$ & Q2 & 7.79 & 8.24 & 0.401995025 & 0.45000000 & 5.78\% \\ \hline
$V_{ce}$ & Q3 & 2.27 & 1 & 1.414213562 & 1.27000000 & 55.95\% \\ \hline
$V_{ce}$ & Q4 & 1.24 & 6.4 & 1.077032961 & 5.16000000 & 416.13\% \\ \hline
$V_{ce}$ & Q5 & 9.99 & 7.4 & 1.019803903 & 2.59000000 & 25.93\% \\ \hline
$V_{ce}$ & Q6 & -9.99 & -8 & 1.019803903 & 1.99000000 & 19.92\% \\ \hline
\end{tabular}
\caption{Puntos estáticos de operación transistor multietapas acoplado}
\label{tab:med-puntos-estaticos-operacion-transistor-multietapas-acoplado}
\end{table}

\FloatBarrier
\subsubsection{modelo dinámico amplificador multietapas modo diferencial}

\begin{table}[h!]
\centering
\begin{tabular}{|c|c|c|c|c|c|}
\hline
\textbf{\(Vg[V]\)} & \textbf{\(\varDelta Vg[V]\)} & \textbf{\(Vi[V]\)} & \textbf{\(\varDelta Vi[V]\)} & \textbf{\(Rp[\Omega]\)} & \textbf{\(\varDelta Rp[\Omega]\)} \\ \hline
0.52 & 0.01 & 0.17 & 0.01 & 42000 & 4200 \\ \hline
\end{tabular}
\caption{Mediciones impedancias de entrada circuito multietapas modo diferencial}
\label{tab:med-impedancias-entrada-circuito-multietapas-modo-diferencial}
\end{table}

\begin{table}[h!]
\centering
\begin{tabular}{|c|c|c|c|c|c|}
\hline
\textbf{\(Vo_{sc}[V]\)} & \textbf{\(\varDelta Vo_{sc}[V]\)} & \textbf{\(Vo_{cc}[V]\)} & \textbf{\(\varDelta Vo_{cc}[V]\)} & \textbf{\(Rp[\Omega]\)} & \textbf{\(\varDelta Rp[\Omega]\)} \\ \hline
0.52 & 0.04 & 0.32 & 0.02 & 100 & 5 \\ \hline
\end{tabular}
\caption{Mediciones impedancia de salida amplifiador multietapa}
\label{tab:med-impedancia-salida-amplifiador-multietapa}
\end{table}

\begin{table}[h!]
\centering
\begin{tabular}{|c|c|c|c|}
\hline
\textbf{\(Vi[V]\)} & \textbf{\(\varDelta Vi[V]\)} & \textbf{\(Vo[V]\)} & \textbf{\(\varDelta Vo[V]\)} \\ \hline
0.0036 & 0.0002 & 0.52 & 0.04 \\ \hline
\end{tabular}
\caption{Ganancia amplificador multietapas modo diferencial}
\label{tab:med-ganancia-amplificador-multietapas-modo-diferencial}
\end{table}


\begin{table}[h!]
\centering
\begin{tabular}{|c|c|c|c|c|c|}
\hline
\textbf{Parámetro} & \textbf{Valor} & \textbf{Medición} & \textbf{Incertidumbre} & \textbf{Error Absoluto} & \textbf{Error Relativo} \\ \hline
$[Z] d$ & 43,990 & 20400 & 2771.263618 & 23590 & 53.63\% \\ \hline
$[Z] o$ & 132 & 62.5 & 16.40625 & 69.5 & 52.65\% \\ \hline
$[A]$ & 379.22 & 144.444444444444 & 13.70592797 & 234.7755556 & 61.91\% \\ \hline
\end{tabular}
\caption{Modelo dinámico amplificador multietapas modo diferencial}
\label{tab:med-modelo-dinamico-amplificador-multietapas-modo-diferencial}
\end{table}

\begin{ilustracion}[ht]
    \centering
    \includegraphics[width=0.8\textwidth]{src/images/resultados/p3/ganancia-multietapas-mod-diff.png}
    \caption{Ganancia del modelo dinámico amplificador multietapas modo diferencial}
    \label{ilus:ganancia-multietapas-mod-diff}
\end{ilustracion}

\begin{ilustracion}[ht]
    \centering
    \includegraphics[width=0.8\textwidth]{src/images/resultados/p3/max-excursion-multietapas-mod-diff.png}
    \caption{Máxima excursión del modelo dinámico amplificador multietapas modo diferencial}
    \label{ilus:med-max-excursion-multietapas-mod-diff}
\end{ilustracion}

\FloatBarrier
\subsubsection{modelo dinámico amplificador multietapas modo común}

\begin{table}[h!]
\centering
\begin{tabular}{|c|c|c|c|c|c|}
\hline
\textbf{\(Vg[V]\)} & \textbf{\(\varDelta Vg[V]\)} & \textbf{\(Vi[V]\)} & \textbf{\(\varDelta Vi[V]\)} & \textbf{\(Rp[\Omega]\)} & \textbf{\(\varDelta Rp[\Omega]\)} \\ \hline
0.032 & 0.004 & 0.008 & 0.0004 & 20000 & 1000 \\ \hline
\end{tabular}
\caption{Impedancia de entrada amplificador multietapas modo común}
\label{tab:med-impedancia-entrada-amplificador-multietapas-modo-comun}
\end{table}


\begin{table}[h!]
\centering
\begin{tabular}{|c|c|c|c|}
\hline
\textbf{\(Vi[V]\)} & \textbf{\(\varDelta Vi[V]\)} & \textbf{\(Vo[V]\)} & \textbf{\(\varDelta Vo[V]\)} \\ \hline
0.024 & 0.002 & 0.38 & 0.02 \\ \hline
\end{tabular}
\caption{Ganancia amplificador multietapas modo común}
\label{tab:med-ganancia-amplificador-multietapas-modo-comun}
\end{table}

\begin{table}[h!]
\centering
\begin{tabular}{|c|c|c|c|c|c|}
\hline
\textbf{Parámetro} & \textbf{Valor} & \textbf{Medición} & \textbf{Incertidumbre} & \textbf{Error Absoluto} & \textbf{Error Relativo} \\ \hline
$[Z] d$ & 49000 & 41142.85714 & 5974.907966 & 7857.142857 & 16.03\% \\ \hline
$[Z] o$ & 132 & 62.5 & 602.0083575 & 69.5 & 52.65\% \\ \hline
$[A]$ & 40.05 & 15.83333333 & 1.560569795 & 24.21666667 & 60.47\% \\ \hline
\end{tabular}
\caption{Modelo dinámico amplificador multietapas modo común}
\label{tab:med-modelo-dinamico-amplificador-multietapas-modo-comun}
\end{table}

\begin{ilustracion}
    \centering
    \includegraphics[width=0.8\textwidth]{src/images/resultados/p3/ganancia-multietapas-mod-comun.png}
    \caption{Ganancia del modelo dinámico amplificador multietapas modo común}
    \label{ilus:ganancia-multietapas-mod-comun}
\end{ilustracion}

\begin{ilustracion}[ht]
    \centering
    \includegraphics[width=0.8\textwidth]{src/images/resultados/p3/max-excursion-multietapas-mod-comun.png}
    \caption{Máxima excursión del modelo dinámico amplificador multietapas modo común}
    \label{ilus:med-max-excursion-multietapas-mod-comun}
\end{ilustracion}


\FloatBarrier
\subsection{Práctica N° 4}

\subsubsection{Puntos estaticos de operación amplificador multietapas}

\begin{table}[h!]
\centering
\begin{tabular}{|c|c|c|c|c|c|c|c|c|c|}
\hline
\textbf{Transistor} & \textbf{\(Vc[V]\)} & \textbf{\(\varDelta Vc[V]\)} & \textbf{\(Vb[V]\)} & \textbf{\(\varDelta Vb[V]\)} & \textbf{\(Ve[V]\)} & \textbf{\(\varDelta Ve[V]\)} & \textbf{\(Re[\Omega]\)} & \textbf{\(\varDelta Re[\Omega]\)} \\ \hline
Q1 & 7.2 & 0.4 & -0.016 & 0.002 & -0.6 & 0.04 & 4700 & 235 \\ \hline
Q2 & 7.6 & 0.4 & 0.048 & 0.004 & -0.64 & 0.04 & 4700 & 235 \\ \hline
Q3 & 7.6 & 0.4 & 8 & 1 & 9 & 1 & 6800 & 340 \\ \hline
Q4 & 0.68 & 0.04 & 0 & 0.1 & -0.56 & 0.04 & 5000 & 500 \\ \hline
Q5 & 10 & 1 & 0.6 & 0.1 & 0.1 & 0.02 & 20 & 1 \\ \hline
Q6 & -10 & 1 & -0.5 & 0.1 & 0.2 & 0.02 & 20 & 1 \\ \hline
\end{tabular}
\caption{Mediciones de voltaje amplificador multietapas en respuesta en frecuencia}
\label{tab:med-voltaje-amplificador-multietapas-respuesta-frecuencia}
\end{table}


\begin{table}[h!]
\centering
\begin{tabular}{|c|c|c|c|c|c|c|}
\hline
\textbf{Parámetro} & \textbf{Transistor} & \textbf{Valor Teórico} & \textbf{Medición} & \textbf{Incertidumbre} & \textbf{Error Absoluto} & \textbf{Error Relativo} \\ \hline
$I_{c}$ & Q1 & 0.00062 & 0.000595745 & 0.000231084 & 0.00002426 & 3.91\% \\ \hline
$I_{c}$ & Q2 & 0.00062 & 0.000510638 & 0.000230574 & 0.00010936 & 17.64\% \\ \hline
$I_{c}$ & Q3 & -0.00237 & -0.002588235 & 0.000204533 & 0.00021824 & 9.21\% \\ \hline
$I_{c}$ & Q4 & 3.02E-04 & 0.000112 & 2.42784E-05 & 0.00019036 & 62.96\% \\ \hline
$I_{c}$ & Q5 & 3.50E-04 & 0.005 & 0.001436141 & 0.00465000 & 1328.57\% \\ \hline
$I_{c}$ & Q6 & 3.50E-04 & 0.005 & 0.001436141 & 0.00465000 & 1328.57\% \\ \hline
$V_{ce}$ & Q1 & 7.79 & 7.8 & 0.401995025 & 0.01000000 & 0.13\% \\ \hline
$V_{ce}$ & Q2 & 7.79 & 8.24 & 0.401995025 & 0.45000000 & 5.78\% \\ \hline
$V_{ce}$ & Q3 & 2.27 & 1.4 & 1.077032961 & 0.87000000 & 38.33\% \\ \hline
$V_{ce}$ & Q4 & 1.24 & 1.24 & 0.056568542 & 0.00000000 & 0.00\% \\ \hline
$V_{ce}$ & Q5 & 9.99 & 9.9 & 1.00019998 & 0.09000000 & 0.90\% \\ \hline
$V_{ce}$ & Q6 & -9.99 & -10.2 & 1.00019998 & 0.21000000 & 2.10\% \\ \hline
\end{tabular}
\caption{Puntos estáticos de operación amplificador multietapa para respuesta en frecuencia}
\label{tab:med-puntos-estaticos-operacion-amplificador-multietapa-respuesta-frecuencia}
\end{table}


\FloatBarrier
\subsubsection{Respuesta en frecuencia amplificador multietapas}

\begin{table}[h!]
\centering
\begin{tabular}{|c|c|c|c|c|c|c|}
\hline
\textbf{N} & \textbf{\(Vi[V]\)} & \textbf{\(\varDelta Vi[V]\)} & \textbf{\(Vo[V]\)} & \textbf{\(\varDelta Vo[V]\)} & \textbf{\(T\)} & \textbf{\(\varDelta T\)} \\ \hline
1 & 0.0032 & 0.0004 & 0.8 & 0.1 & 0.001 & 0.00004 \\ \hline
2 & 0.0032 & 0.0004 & 0.56 & 0.04 & 8.4E-05 & 0.002 \\ \hline
3 & 0.0032 & 0.0004 & 0.48 & 0.04 & 7.20046E-05 & 0.002 \\ \hline
4 & 0.0032 & 0.0004 & 0.56 & 0.04 & 0.014400922 & 0.0004 \\ \hline
5 & 0.0032 & 0.0004 & 0.36 & 0.04 & 0.0330033 & 0.001 \\ \hline
6 & 0.0032 & 0.0004 & 0.1 & 0.04 & 0.107991361 & 0.004 \\ \hline
7 & 0.0032 & 0.0004 & 0.6 & 0.04 & 0.011599582 & 0.0004 \\ \hline
8 & 0.0032 & 0.0004 & 0.76 & 0.04 & 0.005 & 0.0002 \\ \hline
9 & 0.0032 & 0.0004 & 0.68 & 0.04 & 0.00016 & 0.00001 \\ \hline
10 & 0.0032 & 0.0004 & 0.6 & 0.04 & 0.0001 & 0.000004 \\ \hline
11 & 0.0032 & 0.0004 & 0.48 & 0.04 & 7E-05 & 0.0000024 \\ \hline
\\ 
\\ \hline
\textbf{N} & \textbf{\(A\)} & \textbf{\(\varDelta A\)} & \textbf{\(A[dB]\)} & \textbf{\(\varDelta A[dB]\)} & \textbf{\(f[Hz]\)} & \textbf{\(\varDelta f[Hz]\)} \\ \hline
1 & 250 & 44.19417382 & 47.95880017 & 1.535462866 & 1000 & 40 \\ \hline
2 & 175 & 25.19455546 & 44.86076097 & 1.250497876 & 11904.76 & 283446.6213 \\ \hline
3 & 150 & 22.53469547 & 43.52182518 & 1.304892519 & 13888 & 385753.088 \\ \hline
4 & 175 & 25.19455546 & 44.86076097 & 1.250497876 & 69.44 & 1.92876544 \\ \hline
5 & 112.5 & 18.81499153 & 41.02305045 & 1.452666133 & 30.3 & 0.91809 \\ \hline
6 & 31.25 & 13.09613642 & 29.89700043 & 3.640051059 & 9.26 & 0.3429904 \\ \hline
7 & 187.5 & 26.5625 & 45.46002544 & 1.230501032 & 86.21 & 2.97286564 \\ \hline
8 & 237.5 & 32.2117627 & 47.51327228 & 1.178053962 & 200 & 8 \\ \hline
9 & 212.5 & 29.35670973 & 46.54717869 & 1.199948898 & 6250 & 390.625 \\ \hline
10 & 187.5 & 26.5625 & 45.46002544 & 1.230501032 & 10000 & 400 \\ \hline
11 & 150 & 22.53469547 & 43.52182518 & 1.304892519 & 14285.71 & 489.7956245 \\ \hline
\end{tabular}
\caption{Mediciones respuesta en frecuencia amplificador multietapas acoplado por condensadores}
\label{tab:med-respuesta-frecuencia-amplificador-multietapas-acoplado-condensadores}
\end{table}

\begin{table}[h!]
\centering
\begin{tabular}{|c|c|c|c|c|c|}
\hline
\textbf{Parámetro} & \textbf{Valor Teórico} & \textbf{Medición} & \textbf{Incertidumbre} & \textbf{Error Absoluto} & \textbf{Error Relativo} \\ \hline
$F_L$ [Hz] & 70.41 & 69.44 & 1.92876544 & 0.97000000 & 1.38\% \\ \hline
$H_H$ [Hz] & 10890 & 11904.76 & 283446.6213 & 1014.76000000 & 9.32\% \\ \hline
\end{tabular}
\caption{Medición de frecuencias de corte}
\label{tab:med-frecuencias-corte}
\end{table}


\begin{ilustracion}[ht]
    \centering
    \includegraphics[width=0.8\textwidth]{src/images/resultados/p4/respuesta en frecuencia practica 4 con condensadores.png}
    \caption{Respuesta en frecuencia amplificador multietapas acoplado por condensadores}
    \label{ilus:respuesta-frecuencia-amplificador-multietapas-acoplado-condensadores}
\end{ilustracion}

\begin{ilustracion}[ht]
    \centering
    \includegraphics[width=0.8\textwidth]{src/images/resultados/p4/superposicion respuesta en frecuencia con condensadores.png}
    \caption{Superposición de respuesta en frecuencia amplificador multietapas acoplado por condensadores con su simulación}
    \label{ilus:superposicion-respuesta-frecuencia-amplificador-multietapas-acoplado-condensadores}
\end{ilustracion}

\FloatBarrier
\subsubsection{Respuesta en frecuencia amplificador multietapas sin condensadores de acople}

\begin{table}[h!]
\centering
\begin{tabular}{|c|c|c|c|c|c|c|}
\hline
\textbf{N} & \textbf{\(Vi[V]\)} & \textbf{\(\varDelta Vi[V]\)} & \textbf{\(Vo[V]\)} & \textbf{\(\varDelta Vo[V]\)} & \textbf{\(T\)} & \textbf{\(\varDelta T\)} \\ \hline
1 & 1 & 0.1 & 0.026 & 0.002 & 0.001 & 0.00004 \\ \hline
2 & 1 & 0.1 & 0.0072 & 0.0004 & 0.006024096 & 0.0002 \\ \hline
3 & 1 & 0.1 & 0.01 & 0.001 & 0.003003003 & 0.0001 \\ \hline
4 & 1 & 0.1 & 0.015 & 0.001 & 0.002304147 & 0.0001 \\ \hline
5 & 1 & 0.1 & 0.024 & 0.002 & 0.001499993 & 0.0001 \\ \hline
6 & 1 & 0.1 & 0.034 & 0.002 & 0.00076 & 0.00004 \\ \hline
7 & 1 & 0.1 & 0.052 & 0.004 & 0.0005 & 0.00002 \\ \hline
8 & 1 & 0.1 & 0.09 & 0.01 & 0.00025 & 0.00001 \\ \hline
9 & 1 & 0.1 & 0.14 & 0.01 & 0.0001 & 0.000004 \\ \hline
10 & 1 & 0.1 & 0.15 & 0.01 & 5.99988E-05 & 0.000004 \\ \hline
11 & 1 & 0.1 & 0.15 & 0.01 & 0.00005 & 0.000002 \\ \hline
12 & 1 & 0.1 & 0.15 & 0.01 & 0.0001 & 0.00000014 \\ \hline
&&&&&&\\ \hline
\textbf{N} & \textbf{\(A\)} & \textbf{\(\varDelta A\)} & \textbf{\(A[dB]\)} & \textbf{\(\varDelta A[dB]\)} & \textbf{\(f[Hz]\)} & \textbf{\(\varDelta f[Hz]\)} \\ \hline
1 & 0.026 & 0.003280244 & -31.70053304 & 1.095839863 & 1000 & 40 \\ \hline
2 & 0.0072 & 0.00082365 & -42.85335007 & 0.99363008 & 166 & 5.5112 \\ \hline
3 & 0.01 & 0.001414214 & -40 & 1.228370293 & 333 & 11.0889 \\ \hline
4 & 0.015 & 0.001802776 & -36.47817482 & 1.043914015 & 434 & 18.8356 \\ \hline
5 & 0.024 & 0.0031241 & -32.39577517 & 1.130649446 & 666.67 & 44.44488889 \\ \hline
6 & 0.034 & 0.003944617 & -29.37042166 & 1.007720715 & 1315.79 & 69.25213296 \\ \hline
7 & 0.052 & 0.006560488 & -25.67993313 & 1.095839863 & 2000 & 80 \\ \hline
8 & 0.09 & 0.013453624 & -20.91514981 & 1.298407708 & 4000 & 160 \\ \hline
9 & 0.14 & 0.017204651 & -17.07743929 & 1.067412113 & 10000 & 400 \\ \hline
10 & 0.15 & 0.018027756 & -16.47817482 & 1.043914015 & 16667 & 1111.155556 \\ \hline
11 & 0.15 & 0.018027756 & -16.47817482 & 1.043914015 & 20000 & 800 \\ \hline
12 & 0.15 & 0.018027756 & -16.47817482 & 1.043914015 & 10000 & 14 \\ \hline
\end{tabular}
\caption{Mediciones respuesta en frecuencia amplificador multietapas sin condensadores de acople}
\label{tab:med-respuesta-frecuencia-amplificador-multietapas-sin-condensadores-acople}
\end{table}

\begin{ilustracion}[ht]
    \centering
    \includegraphics[width=0.8\textwidth]{src/images/resultados/p4/respuesta en frecuencia practica 4 sin condensadores de acople.png}
    \caption{Respuesta en frecuencia amplificador multietapas acoplado sin condensadores}
    \label{ilus:respuesta-frecuencia-amplificador-multietapas-acoplado-sin-condensadores}
\end{ilustracion}

\FloatBarrier

La figura \ref{fig:amplificador-base} muestra el modelo dinámico del amplificador base. Cuyos parámetros son los mostrados en la tabla \ref{tab:amplificador-base-dinamico}.

\begin{figure}[ht]
    \centering
    \includegraphics[width=0.6\textwidth]{src/images/p5/modelo-amplificador.png}
    \caption{Modelo dinámico del amplificador base}
    \label{fig:amplificador-base}
\end{figure}

\begin{table}[ht]
    \centering
    \begin{tabular}{|c|c|c|c|}
        \hline
        \textbf{Parámetro} & \textbf{Valor} \\ \hline
        $Z_d$ & $43.99$ [$k\Omega$] \\ \hline
        $Z_o$ & $12.13 $ [$\Omega$] \\ \hline
        $A_b$ & $300$ \\ \hline
        $f_L$ & $69.61$ [$Hz$] \\ \hline
        $f_H$ & $11.00$ [$kHz$] \\ \hline
    \end{tabular}
    \caption{Valores de los parámetros dinámicos del amplificador Realimentado}
    \label{tab:amplificador-base-dinamico}

\end{table}

Ahora calculamos los parámetros del amplificador realimentado negativamente.

para la impedancia de entrada $Z_i$ tenemos:

$$Z_i = R_s = 3.3 k\Omega$$

Y la impedancia de salida viene dada por la expresión:

$$Z_o = \frac{R_o}{A / (1 + \frac{R_f}{R_s})}$$

Por lo tanto el valor de $Z_o$ es:

$$Z_o = 0.137 \Omega$$

El valor de la ganancia de la realimentación negativa es: 

$$A_{fb} = - \frac{R_f}{R_s} = - \frac{11k\Omega}{3.3k\Omega} = -3.33$$

Y debido a que $A => \infty$ en el amplificador base, el valor de la ganancia de la realimentación negativa es:

$$A = -\frac{1}{\beta}$$

despejando $\beta$ de la expresión anterior, tenemos:

$$\beta = -\frac{1}{A} = -\frac{1}{3.33} = 0.333$$

Para encontrar las frecuencias de corte inferior utilizamos la expresión:

$$f_{Lf} = \frac{f_{Lb}}{1 + A_{b}}$$

entonces:

$$f_{Hf} = \frac{69.61 Hz}{1 + 49.54} = 1.37 Hz$$

Para hallar la frecuencia de corte superior utilizamos la expresión:

$$f_{Hf} = f_{Hb}\cdot (1 + A_{b})$$

Por lo tanto:

$$f_{Hf} = 11kHz\cdot (1 + 49.54) = 555.94KHz$$

Ahora, para el amplificador realimentado positivamente, procedemos a calcular la ganancia:

$$A_{fb} = 1 + \frac{R_f}{R_s} = 1 + \frac{11k\Omega}{3.3k\Omega} = 4.33$$

La impedancia de entrada con realimentación positiva es:

$$ Z_i = \frac{Z_d \cdot A_b}{1 + \frac{R_f}{R_s}} = \frac{43.99k\Omega \cdot 300}{1 + \frac{11k\Omega}{3.3k\Omega}} = 3.05M\Omega$$

Y la impedancia de salida con realimentación positiva es igual a la impedancia de salida de realimentación negativa:

$$Z_o = 0.137 \Omega$$

\section{Simulaciones}

La figura \ref{fig:amplificador-realimentado-negativo} muestra el circuito del amplificador realimentado negativamente construido en multisim.

\begin{figure}[ht]
    \centering
    \includegraphics[width=1.0\textwidth]{src/images/p5/Prelaboratorio 5 - Realimentación negativa - circuito.png}
    \caption{Circuito amplificador con realimentación negativa}
    \label{fig:amplificador-realimentado-negativo}
\end{figure}
\FloatBarrier

En la figura \ref{fig:ganancia-realimentacion-negativa} podemos observar una ganancia de aproximadamente $3.3$ que coincide con la ganancia de los cálculos.

\begin{figure}[ht]
    \centering
    \includegraphics[width=1\textwidth]{src/images/p5/Prelaboratorio 5 - Retroalimentación negativa - Ganancia.png}
    \caption{Ganancia de la realimentación negativa}
    \label{fig:ganancia-realimentacion-negativa}
\end{figure}

En la figura \ref{fig:respuesta-frecuencia-realimentacion-negativa} podemos observar un aumento en el ancho de banda con realimentación negativa. Podemos observar que las frecuencias de corte coinciden con los cálculados previamente.

\begin{figure}[ht]
    \centering
    \includegraphics[width=1.0\textwidth]{src/images/p5/Prelaboratorio 5 - retroalimentación negativa - respuesta en frecuencia.png}
    \caption{Respuesta en frecuencia del amplificador realimentado negativamente}
    \label{fig:respuesta-frecuencia-realimentacion-negativa}
\end{figure}
\FloatBarrier

% Realimentación positiva

La figura \ref{fig:circuito-realimentacion-positiva-sin-condensador} muestra la construcción del circuito del amplificador realimentado positivamente.

\begin{figure}[ht]
    \centering
    \includegraphics[width=\textwidth]{src/images/p5/Prelaboratorio 5 - Realimentacion positiva sin condensador - circuito.png}
    \caption{Circuito de realimentación positiva sin condensador}
    \label{fig:circuito-realimentacion-positiva-sin-condensador}
\end{figure}

La ganancia de este amplificador se puede observar en la figura del \ref{fig:ganancia-circuito-realimentacion-positiva-sin-condensador} y podemos observar que coincide con la ganancia calculada anteriormente de 4.33. Sin embargo podemos observar que despues de un tiempo la ganancia cambia y toma la forma mostrada en la figura \ref{fig:ganancia-circuito-realimentacion-positiva-sin-condensador-despues-de-unos-segundos}.

\begin{figure}[ht]
    \centering
    \includegraphics[width=\textwidth]{src/images/p5/Prelaboratorio 5 - Realimentacion positiva sin condensador - ganancia.png}
    \caption{Ganancia del circuito de realimentación positiva sin condensador}
    \label{fig:ganancia-circuito-realimentacion-positiva-sin-condensador}
\end{figure}
\FloatBarrier


\begin{figure}[ht]
    \centering
    \includegraphics[width=\textwidth]{src/images/p5/Prelaboratorio 5 - Retroalimentación positiva sin condensador despues de unos segundos.png}
    \caption{Ganancia del circuito de realimentación positiva sin condensador despues de unos segundos}
    \label{fig:ganancia-circuito-realimentacion-positiva-sin-condensador-despues-de-unos-segundos}
\end{figure}
\FloatBarrier

Podemos observar la respuesta en frecuencia del amplificador realimentado positivamente en la figura \ref{fig:respuesta-amplificador-realimentacion-positiva-sin-condensador}.

\begin{figure}[ht]
    \centering
    \includegraphics[width=\textwidth]{src/images/p5/Prelaboratorio 5 - realimentacion positiva sin condensador - respuesta en frecuencia.png}
    \caption{Respuesta en frecuencia del amplificador con Realimentación positiva sin condensador} 
    \label{fig:respuesta-amplificador-realimentacion-positiva-sin-condensador}
\end{figure}

Amplificador realimentado positiva y negativamente

La figura \ref{fig:circuito-amplificador-realimentacion-positiva-y-negativa} muestra el circuito del amplificador con realimentación positiva y negativa con condensador.

\begin{figure}[ht]
    \centering
    \includegraphics[width=\textwidth]{src/images/p5/prelaboratorio 5 - circuito realimentación positiva con condensador.png}
    \caption{Circuito con realimentación positiva y negativa} 
    \label{fig:circuito-amplificador-realimentacion-positiva-y-negativa}
\end{figure}
\FloatBarrier

La ganancia este amplificador se puede observar en la figura \ref{fig:ganancia-amplificador-realimentacion-positiva-y-negativa}.
\begin{figure}[ht]
    \centering
    \includegraphics[width=\textwidth]{src/images/p5/Prelaboratorio 5 - Ganancia realimentacion positiva con condensador.png}
    \caption{Ganancia con realimentación positiva y negativa} 
    \label{fig:ganancia-amplificador-realimentacion-positiva-y-negativa}
\end{figure}
\FloatBarrier

La respuesta en frecuencia del amplificador se puede observar en la figura \ref{fig:respuesta-amplificador-realimentacion-positiva-y-negativa}.

\begin{figure}[ht]
    \centering
    \includegraphics[width=\textwidth]{src/images/p5/Prelaboratorio 5 - Respuesta en frecuencia - realimentacion positiva con condensador.png}
    \caption{Circuito con realimentación positiva y negativa} 
    \label{fig:respuesta-amplificador-realimentacion-positiva-y-negativa}
\end{figure}


\section{Análisis de resultados}
\subsection{Análisis de las aplicaciones de las topologías clásicas}
De los datos obtenidos en la sección \ref{sec:resultados-generador-funciones} se obtienen los errores porcentuales mostrados en las siguientes tablas

\begin{table}[ht]
\centering
\begin{tabular}{|c|c|c|c|c|}
\hline
Descripción & $V_C$ (V) & $\Delta V_C$ (V) & Valor teórico (V) & Error \% \\ \hline
$V_C$ & 10 & 1 & 7.88 & 26.93 \\ \hline
$V_{sat+}$ & 10 & 1 & 8.4 & 19.05 \\ \hline
$V_C$ V[pp] & 17 & 1 & 15.76 & 7.87 \\ \hline
$V_{sat}$ V[pp] & 17 & 1 & 16.8 & 1.19 \\ \hline
\end{tabular}
\caption{Comparación Resultados experimentales con valores teóricos del voltaje $V_c$ }
\label{tab:comparacion-voltaje-Vc}
\end{table}

\begin{table}[ht]
\centering
\begin{tabular}{|c|c|c|c|c|}
\hline
Descripción & $V_t$ (V) & $\Delta V_t$ (V) & Valor teórico (V) & Error \% \\ \hline
Tensión & 7.6 & 0.4 & 7.0 & 8.57 \\ \hline
\end{tabular}
\caption{Resultados experimentales de la tensión $V_t$ y comparación con el valor teórico}
\label{tab:comparacion-voltaje-Vt}
\end{table}

\begin{table}[ht]
\centering
\begin{tabular}{|c|c|c|c|c|}
\hline
Descripción & $T$ ($\mu s$) & $\Delta T$ ($\mu s$) & Valor teórico ($\mu s$) & Error \% \\ \hline
Periodo T & 192.00 & 4 & 200.00 & 4.00 \\ \hline
Retraso SR & 32.00 & 4 & 27 & 18.52 \\ \hline
\end{tabular}
\caption{Resultados experimentales de tiempos y comparación con valores teóricos}
\label{tab:comparacion-tiempo-generador}
\end{table}

De la tabla \ref{tab:comparacion-voltaje-Vc} se observa que el error porcentual del voltaje pico $V_C$ con respecto a su valor teórico es de 26.90\%, mientras que con respecto a $V_{sat+}$ el error es de 19.05\%, estos errores son algo altos. Sin embargo al observar la ilustración \ref{ilus:generador-funciones} se tiene que la forma de onda está levantada con respecto al eje horizontal por lo cual se tomaron los valores pico a pico de la señal, obteniendo un error de 7.87\% con respecto a $V_C$ y 1.19\% con respecto a $V_{sat}$. Estos errores son mucho más aceptables que los obtenidos con los valores pico.

Al observar la tabla \ref{tab:comparacion-voltaje-Vt} se tiene que el error porcentual de la tensión $V_t$ con respecto a su valor teórico es de 8.57\%. Este error es aceptable ya que se encuentra dentro del rango de error esperado.

De la ilustración \ref{ilus:generador-funciones} se observa que la onda $V_t$ no es completamente triangular, esto es debido a que durante el tiempo de retraso del Slewrate, el condensador se sigue cargando y descargando por encima del pico esperado con un comportamiento exponencial, cambiando así la forma de la onda a algo parecido a una senoidal, es también por esta razón, y sumado al hecho de que $V_c \approx V_t \approx V_{sat}$, que la onda triangular también se satura para valores mínimos, mientras se encuentra en la transición debida al Slewrate.

De la tabla \ref{tab:comparacion-tiempo-generador} se observa que el error porcentual del periodo $T$ con respecto a su valor teórico es de 4.00\%, mientras que el error porcentual del retraso debido al Slewrate es de 18.52\%. El error del periodo es aceptable ya que se encuentra dentro del rango de error esperado, sin embargo el error del retraso es bastante alto, esto puede ser debido a la construcción del amplificador y las condiciones de funcionamiento como la temperatura, el error de 18.52\% también puede ser debido a que se tomó la medición con una escala del osciloscopio demasiado grande en comparación con el valor medido.
\FloatBarrier
\subsection{Análisis del amplificador operacional real}
De los datos obtenidos en la sección \ref{sec:resultados-generador-funciones} se obtienen los errores porcentuales mostrados en las siguientes tablas

\begin{table}[ht]
\centering
\begin{tabular}{|c|c|c|c|c|}
\hline
Descripción & $V_C$ (V) & $\Delta V_C$ (V) & Valor teórico (V) & Error \% \\ \hline
$V_C$ & 10 & 1 & 7.88 & 26.93 \\ \hline
$V_{sat+}$ & 10 & 1 & 8.4 & 19.05 \\ \hline
$V_C$ V[pp] & 17 & 1 & 15.76 & 7.87 \\ \hline
$V_{sat}$ V[pp] & 17 & 1 & 16.8 & 1.19 \\ \hline
\end{tabular}
\caption{Comparación Resultados experimentales con valores teóricos del voltaje $V_c$ }
\label{tab:comparacion-voltaje-Vc}
\end{table}

\begin{table}[ht]
\centering
\begin{tabular}{|c|c|c|c|c|}
\hline
Descripción & $V_t$ (V) & $\Delta V_t$ (V) & Valor teórico (V) & Error \% \\ \hline
Tensión & 7.6 & 0.4 & 7.0 & 8.57 \\ \hline
\end{tabular}
\caption{Resultados experimentales de la tensión $V_t$ y comparación con el valor teórico}
\label{tab:comparacion-voltaje-Vt}
\end{table}

\begin{table}[ht]
\centering
\begin{tabular}{|c|c|c|c|c|}
\hline
Descripción & $T$ ($\mu s$) & $\Delta T$ ($\mu s$) & Valor teórico ($\mu s$) & Error \% \\ \hline
Periodo T & 192.00 & 4 & 200.00 & 4.00 \\ \hline
Retraso SR & 32.00 & 4 & 27 & 18.52 \\ \hline
\end{tabular}
\caption{Resultados experimentales de tiempos y comparación con valores teóricos}
\label{tab:comparacion-tiempo-generador}
\end{table}

De la tabla \ref{tab:comparacion-voltaje-Vc} se observa que el error porcentual del voltaje pico $V_C$ con respecto a su valor teórico es de 26.90\%, mientras que con respecto a $V_{sat+}$ el error es de 19.05\%, estos errores son algo altos. Sin embargo al observar la ilustración \ref{ilus:generador-funciones} se tiene que la forma de onda está levantada con respecto al eje horizontal por lo cual se tomaron los valores pico a pico de la señal, obteniendo un error de 7.87\% con respecto a $V_C$ y 1.19\% con respecto a $V_{sat}$. Estos errores son mucho más aceptables que los obtenidos con los valores pico.

Al observar la tabla \ref{tab:comparacion-voltaje-Vt} se tiene que el error porcentual de la tensión $V_t$ con respecto a su valor teórico es de 8.57\%. Este error es aceptable ya que se encuentra dentro del rango de error esperado.

De la ilustración \ref{ilus:generador-funciones} se observa que la onda $V_t$ no es completamente triangular, esto es debido a que durante el tiempo de retraso del Slewrate, el condensador se sigue cargando y descargando por encima del pico esperado con un comportamiento exponencial, cambiando así la forma de la onda a algo parecido a una senoidal, es también por esta razón, y sumado al hecho de que $V_c \approx V_t \approx V_{sat}$, que la onda triangular también se satura para valores mínimos, mientras se encuentra en la transición debida al Slewrate.

De la tabla \ref{tab:comparacion-tiempo-generador} se observa que el error porcentual del periodo $T$ con respecto a su valor teórico es de 4.00\%, mientras que el error porcentual del retraso debido al Slewrate es de 18.52\%. El error del periodo es aceptable ya que se encuentra dentro del rango de error esperado, sin embargo el error del retraso es bastante alto, esto puede ser debido a la construcción del amplificador y las condiciones de funcionamiento como la temperatura, el error de 18.52\% también puede ser debido a que se tomó la medición con una escala del osciloscopio demasiado grande en comparación con el valor medido.
\FloatBarrier
\subsection{Análisis de los filtros activos}
De los datos obtenidos en la sección \ref{sec:resultados-generador-funciones} se obtienen los errores porcentuales mostrados en las siguientes tablas

\begin{table}[ht]
\centering
\begin{tabular}{|c|c|c|c|c|}
\hline
Descripción & $V_C$ (V) & $\Delta V_C$ (V) & Valor teórico (V) & Error \% \\ \hline
$V_C$ & 10 & 1 & 7.88 & 26.93 \\ \hline
$V_{sat+}$ & 10 & 1 & 8.4 & 19.05 \\ \hline
$V_C$ V[pp] & 17 & 1 & 15.76 & 7.87 \\ \hline
$V_{sat}$ V[pp] & 17 & 1 & 16.8 & 1.19 \\ \hline
\end{tabular}
\caption{Comparación Resultados experimentales con valores teóricos del voltaje $V_c$ }
\label{tab:comparacion-voltaje-Vc}
\end{table}

\begin{table}[ht]
\centering
\begin{tabular}{|c|c|c|c|c|}
\hline
Descripción & $V_t$ (V) & $\Delta V_t$ (V) & Valor teórico (V) & Error \% \\ \hline
Tensión & 7.6 & 0.4 & 7.0 & 8.57 \\ \hline
\end{tabular}
\caption{Resultados experimentales de la tensión $V_t$ y comparación con el valor teórico}
\label{tab:comparacion-voltaje-Vt}
\end{table}

\begin{table}[ht]
\centering
\begin{tabular}{|c|c|c|c|c|}
\hline
Descripción & $T$ ($\mu s$) & $\Delta T$ ($\mu s$) & Valor teórico ($\mu s$) & Error \% \\ \hline
Periodo T & 192.00 & 4 & 200.00 & 4.00 \\ \hline
Retraso SR & 32.00 & 4 & 27 & 18.52 \\ \hline
\end{tabular}
\caption{Resultados experimentales de tiempos y comparación con valores teóricos}
\label{tab:comparacion-tiempo-generador}
\end{table}

De la tabla \ref{tab:comparacion-voltaje-Vc} se observa que el error porcentual del voltaje pico $V_C$ con respecto a su valor teórico es de 26.90\%, mientras que con respecto a $V_{sat+}$ el error es de 19.05\%, estos errores son algo altos. Sin embargo al observar la ilustración \ref{ilus:generador-funciones} se tiene que la forma de onda está levantada con respecto al eje horizontal por lo cual se tomaron los valores pico a pico de la señal, obteniendo un error de 7.87\% con respecto a $V_C$ y 1.19\% con respecto a $V_{sat}$. Estos errores son mucho más aceptables que los obtenidos con los valores pico.

Al observar la tabla \ref{tab:comparacion-voltaje-Vt} se tiene que el error porcentual de la tensión $V_t$ con respecto a su valor teórico es de 8.57\%. Este error es aceptable ya que se encuentra dentro del rango de error esperado.

De la ilustración \ref{ilus:generador-funciones} se observa que la onda $V_t$ no es completamente triangular, esto es debido a que durante el tiempo de retraso del Slewrate, el condensador se sigue cargando y descargando por encima del pico esperado con un comportamiento exponencial, cambiando así la forma de la onda a algo parecido a una senoidal, es también por esta razón, y sumado al hecho de que $V_c \approx V_t \approx V_{sat}$, que la onda triangular también se satura para valores mínimos, mientras se encuentra en la transición debida al Slewrate.

De la tabla \ref{tab:comparacion-tiempo-generador} se observa que el error porcentual del periodo $T$ con respecto a su valor teórico es de 4.00\%, mientras que el error porcentual del retraso debido al Slewrate es de 18.52\%. El error del periodo es aceptable ya que se encuentra dentro del rango de error esperado, sin embargo el error del retraso es bastante alto, esto puede ser debido a la construcción del amplificador y las condiciones de funcionamiento como la temperatura, el error de 18.52\% también puede ser debido a que se tomó la medición con una escala del osciloscopio demasiado grande en comparación con el valor medido.
\FloatBarrier
\subsection{Análisis de las fuentes lineales y reguladores monolíticos}
De los datos obtenidos en la sección \ref{sec:resultados-generador-funciones} se obtienen los errores porcentuales mostrados en las siguientes tablas

\begin{table}[ht]
\centering
\begin{tabular}{|c|c|c|c|c|}
\hline
Descripción & $V_C$ (V) & $\Delta V_C$ (V) & Valor teórico (V) & Error \% \\ \hline
$V_C$ & 10 & 1 & 7.88 & 26.93 \\ \hline
$V_{sat+}$ & 10 & 1 & 8.4 & 19.05 \\ \hline
$V_C$ V[pp] & 17 & 1 & 15.76 & 7.87 \\ \hline
$V_{sat}$ V[pp] & 17 & 1 & 16.8 & 1.19 \\ \hline
\end{tabular}
\caption{Comparación Resultados experimentales con valores teóricos del voltaje $V_c$ }
\label{tab:comparacion-voltaje-Vc}
\end{table}

\begin{table}[ht]
\centering
\begin{tabular}{|c|c|c|c|c|}
\hline
Descripción & $V_t$ (V) & $\Delta V_t$ (V) & Valor teórico (V) & Error \% \\ \hline
Tensión & 7.6 & 0.4 & 7.0 & 8.57 \\ \hline
\end{tabular}
\caption{Resultados experimentales de la tensión $V_t$ y comparación con el valor teórico}
\label{tab:comparacion-voltaje-Vt}
\end{table}

\begin{table}[ht]
\centering
\begin{tabular}{|c|c|c|c|c|}
\hline
Descripción & $T$ ($\mu s$) & $\Delta T$ ($\mu s$) & Valor teórico ($\mu s$) & Error \% \\ \hline
Periodo T & 192.00 & 4 & 200.00 & 4.00 \\ \hline
Retraso SR & 32.00 & 4 & 27 & 18.52 \\ \hline
\end{tabular}
\caption{Resultados experimentales de tiempos y comparación con valores teóricos}
\label{tab:comparacion-tiempo-generador}
\end{table}

De la tabla \ref{tab:comparacion-voltaje-Vc} se observa que el error porcentual del voltaje pico $V_C$ con respecto a su valor teórico es de 26.90\%, mientras que con respecto a $V_{sat+}$ el error es de 19.05\%, estos errores son algo altos. Sin embargo al observar la ilustración \ref{ilus:generador-funciones} se tiene que la forma de onda está levantada con respecto al eje horizontal por lo cual se tomaron los valores pico a pico de la señal, obteniendo un error de 7.87\% con respecto a $V_C$ y 1.19\% con respecto a $V_{sat}$. Estos errores son mucho más aceptables que los obtenidos con los valores pico.

Al observar la tabla \ref{tab:comparacion-voltaje-Vt} se tiene que el error porcentual de la tensión $V_t$ con respecto a su valor teórico es de 8.57\%. Este error es aceptable ya que se encuentra dentro del rango de error esperado.

De la ilustración \ref{ilus:generador-funciones} se observa que la onda $V_t$ no es completamente triangular, esto es debido a que durante el tiempo de retraso del Slewrate, el condensador se sigue cargando y descargando por encima del pico esperado con un comportamiento exponencial, cambiando así la forma de la onda a algo parecido a una senoidal, es también por esta razón, y sumado al hecho de que $V_c \approx V_t \approx V_{sat}$, que la onda triangular también se satura para valores mínimos, mientras se encuentra en la transición debida al Slewrate.

De la tabla \ref{tab:comparacion-tiempo-generador} se observa que el error porcentual del periodo $T$ con respecto a su valor teórico es de 4.00\%, mientras que el error porcentual del retraso debido al Slewrate es de 18.52\%. El error del periodo es aceptable ya que se encuentra dentro del rango de error esperado, sin embargo el error del retraso es bastante alto, esto puede ser debido a la construcción del amplificador y las condiciones de funcionamiento como la temperatura, el error de 18.52\% también puede ser debido a que se tomó la medición con una escala del osciloscopio demasiado grande en comparación con el valor medido.
\FloatBarrier
\section{Conclusiones}

A lo largo de este trabajo de laboratorio, se estudiaron diferentes aspectos de los amplificadores operacionales y sus aplicaciones, llegando a las siguientes conclusiones:

\begin{itemize}
    \item Las topologías clásicas (inversor y no inversor) mostraron un comportamiento muy cercano al ideal, con errores del 0\% en sus ganancias. Esto demuestra la fiabilidad de estas configuraciones básicas cuando están correctamente implementadas.
    
    \item El amplificador operacional real $\mu A741$ mostró limitaciones importantes en comparación con el modelo ideal:
    \begin{itemize}
        \item La tensión de offset medida fue significativamente mayor que la especificada (error del 700\% respecto al valor típico), posiblemente debido a las diferentes condiciones de operación.
        \item Las corrientes de bias mostraron una gran desviación (99.09\% de error), evidenciando la sensibilidad de estos parámetros a las condiciones de operación.
        \item El producto ganancia-ancho de banda (GBWP) se mantuvo relativamente constante con errores menores al 13\%, validando esta característica fundamental del dispositivo.
    \end{itemize}
    
    \item Los filtros activos implementados demostraron ser efectivos en el procesamiento de señales:
    \begin{itemize}
        \item El filtro Sallen-Key mostró una excelente precisión en ganancia (0\% error) aunque con un error del 12.23\% en frecuencia.
        \item El filtro de realimentación múltiple presentó mayores desviaciones (10\% en ganancia, 27.10\% en frecuencia), probablemente debido a las tolerancias de los componentes.
        \item No fue posible medir el filtro de variables de estado debido a un error en el circuito implementado.
        \item No se pudo medir el factor de amortiguamiento, por lo cual es necesario realizar más mediciones en las zonas de interes al realizar el barrido.
        \item Por un lado el filtro de realimentación múltiple fue el más dificil de diseñar debido a la fuerte dependencia de sus parámetros de ganancia, frecuencia de corte y factor de amortiguamiento, esto se vio reflejado en los resultados, donde cambios en los componentes utilizados tuvieron un impacto significativo en los parámetros del filtro.
    \end{itemize}
    
    \item En cuanto a las fuentes lineales y reguladores:
    \begin{itemize}
        \item El regulador de voltaje de salida fija demostró excelente precisión con 0\% de error.
        \item La fuente de corriente ajustable mostró variaciones significativas en su precisión (errores entre 0.74\% y 36.48\%), evidenciando la dificultad de ajuste preciso.
        \item Se observó que el voltaje de rizado aumenta a medida que disminuye la carga, lo cual es un factor importante a considerar en el diseño de cualquier circuito que use reguladores.
    \end{itemize}
\end{itemize}

Estas observaciones demuestran la importancia de considerar las no idealidades y limitaciones prácticas al trabajar con circuitos analógicos reales, así como la necesidad de seleccionar cuidadosamente los componentes y condiciones de operación para obtener los resultados deseados.

\FloatBarrier
\section{Anexos}

\begin{ilustracion}[ht]
    \centering
    \includegraphics[width=1.0\textwidth]{src/images/p1/p1-hoja-de-datos.jpg}
    \caption{Hoja de datos práctica N° 1}
    \label{ilus:hoja-de-datos-p1}
\end{ilustracion}

\begin{ilustracion}[ht]
    \centering
    \includegraphics[width=1.0\textwidth]{src/images/p2/p2-hoja-de-datos.jpg}
    \caption{Hoja de datos práctica N° 2}
    \label{ilus:hoja-de-datos-p2}
\end{ilustracion}

\begin{ilustracion}[ht]
    \centering
    \includegraphics[width=1.0\textwidth]{src/images/p3/p3-hoja-de-datos.jpg}
    \caption{Hoja de datos práctica N° 3-1}
    \label{ilus:hoja-de-datos-p3-1}
\end{ilustracion}

\begin{ilustracion}[ht]
    \centering
    \includegraphics[width=1.0\textwidth]{src/images/p3/p3-hoja-de-datos-2.jpg}
    \caption{Hoja de datos práctica N° 3-2}
    \label{ilus:hoja-de-datos-p3-2}
\end{ilustracion}

\begin{ilustracion}[ht]
    \centering
    \includegraphics[width=1.0\textwidth]{src/images/p4/p4-hoja-de-datos-1.jpg}
    \caption{Hoja de datos práctica N° 4-1}
    \label{ilus:hoja-de-datos-p4-1}
\end{ilustracion}

\begin{ilustracion}[ht]
    \centering
    \includegraphics[width=1.0\textwidth]{src/images/p4/p4-hoja-de-datos-2.jpg}
    \caption{Hoja de datos práctica N° 4-2}
    \label{ilus:hoja-de-datos-p4-2}
\end{ilustracion}

\begin{ilustracion}[ht]
    \centering
    \includegraphics[width=1.0\textwidth, angle=90]{src/images/p5/p5-hoja-de-datos-1.jpg}
    \caption{Hoja de datos práctica N° 5-1}
    \label{ilus:hoja-de-datos-p5-1}
\end{ilustracion}

\begin{ilustracion}[ht]
    \centering
    \includegraphics[width=1.0\textwidth,angle=90]{src/images/p5/p5-hoja-de-datos-2.jpg}
    \caption{Hoja de datos práctica N° 5-2}
    \label{ilus:hoja-de-datos-p5-2}
\end{ilustracion}

\begin{ilustracion}[ht]
    \centering
    \includegraphics[width=1.0\textwidth, angle=90]{src/images/p5/p5-hoja-de-datos-3.jpg}
    \caption{Hoja de datos práctica N° 5-3}
    \label{ilus:hoja-de-datos-p5-3}
\end{ilustracion}

\includepdf[page=-, width=0.9\textwidth]{src/assets/BC237.pdf}
\captionof{ilustracion}{Hoja de datos del transistor BC237}

\includepdf[page=-, width=0.9\textwidth]{src/assets/bc307.pdf}
\captionof{ilustracion}{Hoja de datos del transistor BC307}
\end{document}