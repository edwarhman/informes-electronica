\section{Conclusiones}

A lo largo de este trabajo de laboratorio, se estudiaron diferentes aspectos de los amplificadores operacionales y sus aplicaciones, llegando a las siguientes conclusiones:

\begin{itemize}
    \item Las topologías clásicas (inversor y no inversor) mostraron un comportamiento muy cercano al ideal, con errores del 0\% en sus ganancias. Esto demuestra la fiabilidad de estas configuraciones básicas cuando están correctamente implementadas.
    
    \item El amplificador operacional real $\mu A741$ mostró limitaciones importantes en comparación con el modelo ideal:
    \begin{itemize}
        \item La tensión de offset medida fue significativamente mayor que la especificada (error del 700\% respecto al valor típico), posiblemente debido a las diferentes condiciones de operación.
        \item Las corrientes de bias mostraron una gran desviación (99.09\% de error), evidenciando la sensibilidad de estos parámetros a las condiciones de operación.
        \item El producto ganancia-ancho de banda (GBWP) se mantuvo relativamente constante con errores menores al 13\%, validando esta característica fundamental del dispositivo.
    \end{itemize}
    
    \item Los filtros activos implementados demostraron ser efectivos en el procesamiento de señales:
    \begin{itemize}
        \item El filtro Sallen-Key mostró una excelente precisión en ganancia (0\% error) aunque con un error del 12.23\% en frecuencia.
        \item El filtro de realimentación múltiple presentó mayores desviaciones (10\% en ganancia, 27.10\% en frecuencia), probablemente debido a las tolerancias de los componentes.
        \item No fue posible medir el filtro de variables de estado debido a un error en el circuito implementado.
        \item No se pudo medir el factor de amortiguamiento, por lo cual es necesario realizar más mediciones en las zonas de interes al realizar el barrido.
        \item Por un lado el filtro de realimentación múltiple fue el más dificil de diseñar debido a la fuerte dependencia de sus parámetros de ganancia, frecuencia de corte y factor de amortiguamiento, esto se vio reflejado en los resultados, donde cambios en los componentes utilizados tuvieron un impacto significativo en los parámetros del filtro.
    \end{itemize}
    
    \item En cuanto a las fuentes lineales y reguladores:
    \begin{itemize}
        \item El regulador de voltaje de salida fija demostró excelente precisión con 0\% de error.
        \item La fuente de corriente ajustable mostró variaciones significativas en su precisión (errores entre 0.74\% y 36.48\%), evidenciando la dificultad de ajuste preciso.
        \item Se observó que el voltaje de rizado aumenta a medida que disminuye la carga, lo cual es un factor importante a considerar en el diseño de cualquier circuito que use reguladores.
    \end{itemize}
\end{itemize}

Estas observaciones demuestran la importancia de considerar las no idealidades y limitaciones prácticas al trabajar con circuitos analógicos reales, así como la necesidad de seleccionar cuidadosamente los componentes y condiciones de operación para obtener los resultados deseados.