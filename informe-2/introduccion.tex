Los amplificadores operacionales, comúnmente conocidos como op-amps, son dispositivos electrónicos ampliamente utilizados en circuitos analógicos debido a su versatilidad y eficiencia. Estos componentes son amplificadores de voltaje de alta ganancia, diseñados para realizar una variedad de operaciones matemáticas y de procesamiento de señales, como amplificación, filtrado, integración, diferenciación y sumación, entre otras.

Los amplificadores operacionales son componentes esenciales en la electrónica moderna, gracias a su flexibilidad y capacidad para realizar una amplia variedad de tareas. Su comprensión y manejo son fundamentales para el diseño y análisis de circuitos analógicos avanzados.

En este informe, se estudiarán las características y aplicaciones de los amplificadores operacionales, como pueden ser los filtros activos, así como los reguladores de voltaje y fuentes de corriente, que son dispositivos esenciales en la electrónica analógica.



