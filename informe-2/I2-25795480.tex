\documentclass{article}
\usepackage{xr}
\usepackage{graphicx} % Para incluir imágenes
\usepackage[spanish]{babel}
\usepackage{makeidx}
\usepackage{geometry}
\usepackage{amsmath}
\usepackage{graphicx}
\usepackage{caption}
\usepackage{pdfpages}
\usepackage{hyperref}
\usepackage{placeins}
\usepackage{adjustbox}
\usepackage{import}


\hypersetup{
    colorlinks,
    citecolor=black,
    filecolor=black,
    linkcolor=black,
    urlcolor=black
}

\geometry{a4paper, total={170mm,257mm}, left=20mm, top=20mm}
\renewcommand{\familydefault}{\sfdefault}


\graphicspath{
    {./parte-1/src/images/}
}


\begin{document}
\import{./}{portada.tex}
\tableofcontents
\newpage
\section{Introducción}
\import{./}{introduccion.tex}

\section{Resumen}
\import{./}{resumen.tex}

\section{Objetivos}
\import{parte-1/src/}{objetivos.tex}

\section {Marco teórico}
\import{./parte-1/src/}{marco-teorico.tex}

\section{Metodología}
\import{./parte-1/src/}{trabajo-de-preparacion.tex}

\end{document}