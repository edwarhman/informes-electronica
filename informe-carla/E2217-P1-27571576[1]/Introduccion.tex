\section{Introducción}

Los amplificadores discretos, implementados con componentes individuales, constituyen un elemento fundamental en electrónica para comprender los principios básicos de amplificación de señales. Este estudio experimental caracteriza el comportamiento de un amplificador discreto, analizando tanto el funcionamiento independiente como acoplado de sus tres etapas principales: la etapa diferencial (encargada del procesamiento inicial de la señal), la etapa impulsora (que condiciona la señal para su amplificación final) y la etapa de potencia (responsable de entregar la energía a la carga).

A través de mediciones precisas de ganancia, impedancias y respuesta en frecuencia, la investigación establece una conexión entre los fundamentos teóricos y su manifestación práctica, con especial énfasis en los efectos de la realimentación negativa (para estabilidad) y positiva (para incremento de ganancia).

Los resultados obtenidos no solo validan los modelos teóricos, sino que además revelan las relaciones críticas entre las diferentes etapas del amplificador. Estas conclusiones proporcionan directrices valiosas para el diseño y optimización de circuitos amplificadores, remarcando la importancia de considerar tanto el comportamiento individual de cada componente como su interacción dentro del sistema completo.