\section{Resumen}

En este laboratorio se estudió el amplificador discreto, compuesto por transistores, resistencias y condensadores, analizando sus tres etapas clave: diferencial (encargada de amplificar la diferencia entre señales de entrada), impulsora (que provee la corriente necesaria para excitar la etapa final) y de potencia (que entrega la señal amplificada a la carga). El propósito fue comparar los valores teóricos y experimentales de los puntos de operación, ganancia e impedancias (en modos diferencial y común), así como evaluar su respuesta en frecuencia (ancho de banda y ganancia) y el efecto de la realimentación negativa (estabilización de la ganancia) y positiva (aumento de ganancia con riesgo de inestabilidad). Los resultados demostraron la importancia del diseño individual de cada etapa y su interacción en el sistema, validando los principios teóricos de un amplificador discreto y realimentación. 
