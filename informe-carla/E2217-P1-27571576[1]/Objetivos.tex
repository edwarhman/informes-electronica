\section{Objetivos}

    \subsection{Objetivos Generales}

        \begin{itemize}

            \item Analizar el comportamiento de un Amplificador de Potencia.
            \item Analizar el comportamiento de un Amplificador Diferencial.
            \item Analizar un Amplificador de múltiples etapas, acoplado capacitivamente.
            \item Analizar el comportamiento en función de la frecuencia de un amplificador multietapas. 
            \item Analizar el efecto de la realimentación en el comportamiento de un amplificador.
            
        \end{itemize}

    \subsection{Objetivos Específicos}

        \begin{itemize}

            %-----Parte01-----
            \item Caracterizar, en dinámico, etapas amplificadoras de potencia B y AB. 
            \item Calcular y medir la polarizacion de los Amplificadores de potencia B y AB.
            \item Determinar las impedancias de entrada y salida de las distintas etapas.
            \item Reconocer las distorsiones generadas por las etapas de simetría complementaria.
            %-----Parte02-----
            \item Modelar circuitalmente etapas amplificadoras diferenciales, en dinámico y en términos de cada modo, común y diferencial.
            \item Determinar la polarización de un Amplificador diferencial y verificarla experimentalmente.
            \item Distinguir entre una señal de entrada modo común y modo diferencial.
            %-----Parte03-----
            \item Analizar un amplificador de múltiples etapas, acoplado capacitivamente de acuerdo a: Ganancia del amplificador, Impedancia de entrada y salida. Tensión de alimentación.
            \item Construir el amplificador con los transistores existentes comercialmente y con las especificaciones y curvas suministradas por el fabricante.
            \item Caracterizar el amplificador multietapas con los valores obtenidos experimental y teóricamente.
            %-----parte04-----
            \item Definir el modelo de banda ancha para amplificadores.
            \item Analizar los efectos de las reactancias en la respuesta en frecuencia de un amplificador y discriminar aquellas que actúan en la región de bajas frecuencias de
            la que actúan en altas frecuencias.
            %-----parte05-----
            \item Reconocer que la realimentación negativa reduce la ganancia y a cambio ofrece:  

            \begin{itemize}
            
                \item Mayor independencia de la ganancia del valor de la ganancia del amplificador base.  
                \item Disminución de la impedancia de salida.  
                \item Aumento de la impedancia de entrada.  
                \item Aumento del ancho de banda.  
                \item Mejoramiento de la linealidad del amplificador.  

            \end{itemize}
            
        \end{itemize}