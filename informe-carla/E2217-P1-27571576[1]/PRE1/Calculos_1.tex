\begin{enumerate}
    \item Identifique la etapa de Potencia (EP) en el amplificador base (anexo), si es necesario consulte a su instructor.

    \begin{figure}[H]
        \centering
        \includegraphics[width=0.5\linewidth]{IMAGENES - PRE1/EP.PNG}
        \caption{Etapa de Potencia del amplificador base}
        \label{fig:enter-label}
    \end{figure} 

    \item Determinar los puntos de reposo de todos los transistores en la EP.

    Para determinar los puntos de operación de los transistores en la etapa de potencia, se realizó un análisis en DC, sin considerar los condensadores C5 y C7. 

    Se comenzó por estudiar al transistor Q4. Este transistor para el caso de la etapa de potencia cumple la función de generar el voltaje necesario para que los transistores Q5 y Q6 conduzcan, asumiendo que la corriente que pasa por la base del transistor Q5 es muy pequeña, se puede decir por LCK:

     \begin{equation}
        I_{R17} = I_{RV1} + I_{CQ4}\label{ecuacion1}
    \end{equation}

    Donde:

    \begin{itemize}
        \item \( I_{CQ4} \): Corriente de colector en el transistor Q4.
        \item \( I_{RV1} \): Corriente en el potenciómetro RV1.
        \item \( I_{R17} \): Corriente en la resistencia R17.
    \end{itemize}

    Si se asume que la corriente en la base de Q4 se puede despreciar por ser muy pequeña, se tiene que la corriente que pasa por el potenciómetro es:

    \begin{equation}
        I_{RV1} = \frac{V_{BE}}{X(RV1)}\label{ecuacion2}
    \end{equation}

    Donde:

    \begin{itemize}
        \item $I_{RV1}$: Corriente en el potenciómetro.
        \item $V_{BEQ4}$: Tensión base-emisor del transistor Q4.
        \item $RV1$: Potenciómetro.
    \end{itemize}
    
    Por LVK se tiene que:

    \begin{equation}
        V_{CEQ4} = I_{RV1}(RV1)\label{ecuacion3}
    \end{equation}

    Sustituyendo la Ec. \ref{ecuacion2}  en la Ec. \ref{ecuacion3} se tiene que:  

    \begin{equation}
        V_{CE} = \frac{V_{BE}}{X(RV1)}(RV1)
    \end{equation}

    Simplicando:

    \begin{equation}
        V_{CEQ4} = \frac{V_{BE}}{X}\label{ecuacion4}
    \end{equation}

    Como X solo puede adquirir valores entre 0 y 1 para que los transistores Q5 y Q6 conduzcan es necesario que el voltaje Colector – Emisor sea 2 veces el voltaje Base – Emisor por lo que se puede decir que X debería ser 0,5.

    Para calcular la corriente que pasa por el transistor Q4 se despreciaron las corrientes de base de Q5 y Q6 y usando LVK se tiene:

    \begin{equation}
        20 - (R_{17} + R_{12})I_{R17} - 2V_{BE} = 0  
    \end{equation}

    Despejando la corriente que pasa por R17:

    \begin{equation}
        I_{R17} = \frac{20 - 2V_{BE}}{R_{17} + R_{12}}\label{ecuacion5}
    \end{equation}

    Reemplazando  la Ec.\ref{ecuacion5} y la Ec. \ref{ecuacion2} en la Ec. \ref{ecuacion1}  se despeja la corriente de colector de Q4:

    \begin{equation}
        V_{CQ4} = \frac{20 - 2V_{BE}}{R_{17} + R_{12}} - \frac{V_{BE}}{X}\label{ecuacion6}
    \end{equation}

    Usando:
    \begin{itemize}
        \item $X = 0,5$
        \item $V_{BE} = 0,62V.$
        \item $RV1 = 10K$$\Omega$$.$
        \item $R_{12} = R_{17} = 22K$$\Omega$$.$ 
    \end{itemize}
    
    Se tiene que:
    
    $$ I_{CQ4} = 302.36\mu\text{A} $$
    
    Para calcular el punto de operaciones de Q5 y Q6 se usó la LVK en la malla que tiene los 3 transistores sin tomar en cuenta las corrientes de base por ser muy pequeñas:
    
        \begin{equation}
            V_{BEQ5} + 10I_{EQ5} + 10I_{EQ6} - V_{EBQ6} + V_{ECQ4} = 0 \label{ecuacion7}
        \end{equation}
    
    Donde se tomó en consideración que: $ I_{EQ5} \approx I_{EQ6}; V_{CEQ4} = 2V_{BEQ4} $

    Despejando $I_{E}$ 
    
    \begin{equation}
        I_E = \frac{V_{CEQ4} - V_{BEQ5} - V_{BEQ6}}{R_{13} + R_{14}}
    \end{equation}

    Sabiendo que \( V_{BEQ5} = 0{,}62\,V \) y \( V_{BEQ6} = 0{,}614\,V \), se obtiene:

    \[
    \therefore I_E = 0{,}3\,mA
    \]
    
    Por lo tanto, la corriente de colector para los transistores Q5 y Q6 se obtendrá de la siguiente relación: 

    \begin{equation}
        I_C = I_E \left( \frac{\beta}{1 + \beta} \right)
        \label{eq:corriente_colector}
    \end{equation}

    Para Q5 se tiene un \(\beta_{Q5} = 210\).\par
    Para Q6 se tiene un \(\beta_{Q6} = 240\).
    

    \[I_{CQ5} = 298{,}57 \, \mu\mathrm{A}\]
    
    \[I_{CQ6} = 298{,}75 \, \mu\mathrm{A}\]
    
    Se asume que \(V_{CEQ5} \approx -V_{CEQ6} = V_{CE}\), aplicando LVK tenemos:
    
    \[2V_{CE} + I_C (R_{13} + R_{14}) - 20 = 0\]
    
    Despejando \(V_{CE}\):
    
    \[V_{CE} = \frac{20 - I_C (R_{13} + R_{14})}{2}\]
    
    \[\therefore V_{CE} = 10\,\mathrm{V}\]
    
    \begin{table}[H]
    \centering
    \caption{Puntos estáticos de operación del amplificador de potencia}
    \label{tab:operacion}
    \begin{tabular}{|c|c|c|}
    \hline
    \textbf{Transistor} &       \(\bm{I_C}\)          & \(\bm{V_{CE}}\)         \\ \hline
    Q4                  & \(302{,}36\,\mu\mathrm{A}\) & \(1{,}24\,\mathrm{V}\)  \\ \hline
    Q5                  & \(298{,}57\,\mu\mathrm{A}\) & \(9{,}99\,\mathrm{V}\)  \\ \hline
    Q6                  & \(298{,}75\,\mu\mathrm{A}\) & \(-9{,}99\,\mathrm{V}\) \\ \hline
    \end{tabular}
    \end{table}
        
    \item Determine el modelo dinámico a pequeña señal y a frecuencias medias de la EP, ganancia e impedancias de entrada y salida.

    \begin{figure}[H]
        \centering
        \includegraphics[width=0.5\linewidth]{IMAGENES - PRE1/Modelo hibrido.png}
        \caption{Modelo híbrido}
        \label{fig:enter-label}
    \end{figure}

    Las ecuaciones a usar para obtener los parámetros serán los siguientes:

    \begin{equation}
        g_{m} = \frac{I_{C}}{V_{T}} \label{ec_gm}    
    \end{equation}
    
    \begin{equation}
        r_{\pi} = \frac{\beta}{g_{m}} \label{ec_rpi}   
    \end{equation}
    
    Donde $V_T$ es el voltaje térmico que a temperatura ambiente se tomó de 25,86mV aproximadamente, $\beta$ será la ganancia de corriente DC del transistor y $ R_{o} = \infty$.

    \begin{equation}
        h_{fe} = \beta \qquad
    \end{equation}

    \begin{equation}
        h_{ie} = r_\pi
    \end{equation}

    Asi se obtuvo que:
    
    \begin{table}[H]
    \centering
    \caption{Parámetros híbridos de los transistores en la etapa de potencia}
    \label{tab:hibridos}
    \begin{tabular}{|c|c|c|c|}
    \hline
    \textbf{Transistor} & \(\bm{h_{fe}}\) &     \(\bm{g_m}\)         & \(\bm{r_\pi}\)               \\ \hline
            Q4          &       210       & \(11{,}69\,\mathrm{mS}\) & \(17{,}96\,\mathrm{k\Omega}\) \\ \hline
            Q5          &       210       & \(11{,}54\,\mathrm{mS}\) & \(18{,}19\,\mathrm{k\Omega}\) \\ \hline
            Q6          &       240       & \(11{,}55\,\mathrm{mS}\) & \(20{,}77\,\mathrm{k\Omega}\) \\ \hline
    \end{tabular}
    \end{table}

    \underline{\textbf{Ganancia}}

    \begin{equation}
        \frac{V_{out}}{V_{in}} = \frac{(R_L + R_{13})(1 + g_{m5}r_{\pi5})}
                                {r_{\pi5} + (R_L + R_{13})(1 + g_{m5}r_{\pi5})} 
    \end{equation}
       
    Al sustituir los valores se tiene que:
    
    $$\frac{V_{out}}{V_{in}} = 0.96 \approx 1 $$
    
    \underline{\textbf{Impedancia de entrada}}
    
    \begin{equation}
        Z_{in} = R_{17}|| R_{12}||[r_{\pi} + (1 + g_mr_{\pi})(R_L + R_{13})]   
    \end{equation}
    
    $$ Z_{in} = 10,75 k\Omega \approx 10 k\Omega$$
    
    \underline{\textbf{Impedancia de salida}}
    
    \begin{equation}
        Z_o = R_{13} + \frac{r_{\pi} + R_{17}||R_{12}}{(1 + g_mr_{\pi})}
    \end{equation}
    
    $$ Z_o \approx R_{13} \approx 10\Omega $$

    \item Realice la simulación de la etapa con el fin de verificar tanto los puntos de operación, como el modelo circuital a pequeña señal. Explique cualquier diferencia respecto a sus cálculos, si la hay.

    \begin{ilustracion}[H]
        \centering
        \includegraphics[width=0.7\linewidth]{IMAGENES - PRE1/S1_Puntos_de_Operacion.PNG}
        \caption{Caption}
        \label{fig:enter-label}
    \end{ilustracion}

    \begin{ilustracion}[H]
        \centering
        \includegraphics[width=0.7\linewidth]{IMAGENES - PRE1/S2_A.PNG}
        \caption{Ganancia en la etapa de potencia}
        \label{fig:enter-label}
    \end{ilustracion}

    \begin{ilustracion}[H]
        \centering
        \includegraphics[width=0.7\linewidth]{IMAGENES - PRE1/S3_Zin.PNG}
        \caption{Impedancia de entrada en la etapa de potencia}
        \label{fig:enter-label}
    \end{ilustracion}

    \begin{ilustracion}[H]
        \centering
        \includegraphics[width=0.7\linewidth]{IMAGENES - PRE1/S4_Zo_EP.PNG}
        \caption{Impedancia de salida en la etapa de potencia}
        \label{fig:enter-label}
    \end{ilustracion}
    
\end{enumerate}