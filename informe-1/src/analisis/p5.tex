\subsection{Análisis Realimentación}


\begin{itemize}
    \item observando las mediciones del modelo dinámico se tiene que para la impedancia de entrada encontramos un error del 57.14\% esto puede ser debido a que la resistencia patrón utilizada es muy diferente al valor real, por otro lado podemos ver que la medición de la impedancia de salida tiene un error de 100\% pero hay que tener en cuenta que el valor teórico es de $0.17 \Omega$ y que la resistencia patrón utilizada fue de $1 \Omega$, ya que al utilizar una de un valor más bajo no se podía tomar la medición correctamente. Al observar la incertidumbre de la medición vemos que se acerca bastante al valor teórico calculado.
    \item En la ilustración \ref{ilus:superposicion-respuesta-frecuencia-amplificador-multietapas-acoplado-condensadores} al superponer la respuesta en frecuencia medida con la simulación una correspondencia en la curva de la respuesta, tienen aproximadamente el mismo valor máximo y el mismo ancho de banda.
    \item observando las frecuencia de cortes podemos ver que la frecuencia de corte superior tiene un error de 27.01\% sin embargo teniendo en cuenta que el valor práctico dio 1 Hz y el teórico era 1.37 Hz parece una aproximación aceptable. Por otro lado el error en la frecuencia de corte superior es de tan solo 5.81\%.
    \item En la ilustración \ref{ilus:amplificador-realimentado-negativamente-superposicion} al comparar la respuesta en frecuencia práctica con la simulación podemos observar que la forma de la curva se asemejan, la magnitud es aproximadamente igual (13dB) y que el ancho de banda corresponde. Las discrepancias más notables son debido a que no se tomaron suficientes puntos en el muestreo en la parte de baja frecuencia. Por otro lado al compararla con la respuesta en frecuencia del amplificador sin realimentar podemos observar que el ancho de banda aumentó bastante. 
\end{itemize}