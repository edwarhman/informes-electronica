\subsection{Análisis de la etapa de potencia}


\begin{itemize}
    \item Al medir los puntos de operación en la etapa de potencia podemos observar que el máximo error en la medición de voltaje $V_{ce}$ es de tan solo $4.84\%$ por lo que nuestra predeterminación para este valor es correcto. Por otro lado podemos observar que el error en la medición $I_c$ está entre 60\% y 529\%, estos errores son grandísimos, esto se puede deber a que dichas corrientes de colector son muy pequeñas.
    \item Para el modelo dinámico del amplificador podemos observar que la impedancia de entrada $Z_i$ tiene un error de solo $7.15\%$ siendo un valor aceptable, igualmente la medición de ganancia tiene un error de $3.85\%$ indicando que la predeterminación es correcta, por su parte la medición de impedancia de salida $Z_o$ tiene un error de $91.16\%$ lo cual es muy alto. El valor de la medición es de 11, lo cual se acerca más al valor de la resistencia $r_{13} = 100\Omega$.
    \item Al modificar la etapa para que se comporte como un amplificador clase C podemos observar que los voltajes $V_{ce}$ medidos corresponden con los teóricos, mientras que las corrientes $I_e$ medidas siguen teniendo un error altísimos. No se observan cambios notables entre el punto de operación del amplificador clase C y el AB. 
    \item En la figura \ref{fig:efecto-crossover-amplificador-clase-c} podemos observar el efecto crossover del amplificador clase C el cual consiste en una pequeña distorsión en la zona de cruce de la onde. No se persive ningún cambio en la ganancia con respecto al otro modo. .
\end{itemize}