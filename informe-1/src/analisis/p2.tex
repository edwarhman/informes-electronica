\subsection{Análisis de la etapa diferencial}


\begin{itemize}
    \item Observando el punto estático de operación del cuadro \ref{tab:med-puntos-reposo-etapa-diferencial} se observa que el error en la medición $V_{ce}$ es de 0.13\% y el de $I_c$ es de 3.91\%. Por lo tanto la predeterminación realizada para este valor es correcta.
    \item Para el modelo dinámico del amplificador en la etapa diferencial del cuadro \ref{tab:med-modelo-dinamico-etapa-diferencial-modo-diferencial} podemos observar que la impedancia de entrada $Z_i$ tiene un error de solo $6.47\%$ siendo un valor aceptable, igualmente la medición de ganancia tiene un error de $8.11\%$ indicando que la predeterminación es correcta, por su parte la medición de impedancia de salida $Z_o$ tiene un error de $0.00\%$ el valor práctico es igual al valor teórico.
    \item Para el modo común el cuadro \ref{tab:med-modelo-dinamico-etapa-diferencial-modo-comun} muestra que la impedancia de entrada $Z_i$ tiene error de $16.80\%$ lo cual es un poco alto, esto es debido a que la resistencia patrón utilizada de $48k\Omega$ dista bastante del valor real $24.50k\Omega$. Por otro lado la medición de impedancia de salida es el mismo que en el modelo dinámico del modo diferencial, con un error de $0.00\%$ y el error de la ganancia es de tan solo $3.23\%$. Por lo cual queda que los valores prácticos corresponden con los valores teóricos.
    \item Al observar las señales de la ilustracións \ref{ilus:max-excursion-mod-diff} y \ref{ilus:max-excursion-mod-comun} podemos observar que la señal de entrada para la máxima excursión en el modo diferencial es de $1V$ mientras que la máxima excursión en modo común es de 6V pico.

\end{itemize}