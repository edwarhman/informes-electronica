\subsection{Análisis respuesta en frecuencia}


\begin{itemize}
    \item Observando el punto estático de operación del cuadro \ref{tab:med-puntos-estaticos-operacion-amplificador-multietapa-respuesta-frecuencia} se observa que los corresponden a los teóricos exepto para los valores de corriente $I_c$ en la etapa de potencia, este error podría deberse a que no se usa la expresión corecta para hallar la medición indirecta correcta.
    \item En la ilustración \ref{ilus:superposicion-respuesta-frecuencia-amplificador-multietapas-acoplado-condensadores} al superponer la respuesta en frecuencia medida con la simulación una correspondencia en la curva de la respuesta, tienen aproximadamente el mismo valor máximo y el mismo ancho de banda.
    \item En la ilustración \ref{ilus:respuesta-frecuencia-amplificador-multietapas-acoplado-sin-condensadores} podemos observar que la respuesta en frecuencia al quitar los condensadores de acople cambia totalmente, teniendo una magnitud máxima de aproximadamente -20dB y cambiando totalmente el ancho de banda. Esto puede ser debido a una mala conexión del cable que sustituye al condensador de acople entre la etapa de potencia y la etapa de impulsora. Ya que dicho cable debería estar conectado al colector de $Q_3$ y al terminar la práctica se observó que estaba conectado al emisor.

\end{itemize}