
\section{Conclusiones}

Para la etapa de potencia se pudo constatar que las predeterminaciones de voltaje $V_{ce}$, impedancia de entrada y salida y la ganancia fueron correctas. Mientras que se observó que la corriente $I_c$ discrepaba bastante con los cálculos previos. En la etapa diferencial fue la que dió mejores resultados, siendo los valores medidos muy similares a los valores teóricos para ambos modos de operación (común y diferencial). Sin embargo en la etapa impulsora se encontró una gran diferencia entre la ganancia medida y la ganancia esperada, y tampoco se pudieron encontrar las impedancias de entrada y salida.

Para el amplificador multietapas, se pudo constatar que los puntos de operación de las distintas etapas no cambian demasiado al acoplarlas juntas, ya sea con condensadores o con cortos, mientras que no se pudieron alcanzar los valores esperados de ganancia, quedandose algo por debajo del valor teórico. Por otro lado existía una gran diferencia en las impedancias de entrada teoricas y prácticas.

La respuesta en frecuencia del amplificador multietapas también dio muy buenos resultados, siendo la curva medida bastante similar a la curva teórica. Las frecuencias de corte medidas se asemejan a las esperadas. Sin embargo, para el amplificador sin condensadores de acople, la curva difiere totalmente de la curva esperada, que debió ser muy similar a la respuesta en frecuencia con condensadores.

En la realimentación pudimos constatar el efecto en el ancho de banda de la respuesta en frecuencia, incrementandolo considerablemente, También pudimos confirmar que la impedancia de salida disminuye con respecto a la impedancia de entrada del amplificador sin realimentar, Todas estas son características del amplificador realimentado. 