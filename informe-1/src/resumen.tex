\section{Resumen}

En el presente informe, se expondrán los resultados teóricos y prácticos de un amplificador construido con transistores y multiples elementos pasivos. Este dispositivo consta de una etapa diferencial, una impulsora y una de amplificación. Se analizarán cada una de estas etapas de manera individual así cómo de manera  conjunta para así entender su comportamiento y el aporte realizado por cada una de estas etapas cuando operan simultáneamente y están conectadas entre sí. Se realizarán estudios de los modelos estáticos y dinámicos del amplificador así cómo su respuesta en frecuencia. Por último se analizará el efecto que tiene la realimentación en el amplificador de estudio, observando las características y ventajas que tiene este tipo de conexiones. 