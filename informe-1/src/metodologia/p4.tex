\subsection{Respuesta en frecuencia}

En primer lugar se identifican los capacitores de baja frecuencia, los cuales son C1, C2, C3, C5, C6 y C7. para obtener las frecuencias de corte usaremos:

$$ \omega_{ci} = \frac{1}{C_i R_{eqCi}}$$

Para $C_1$ tenemos:

$$ \omega_{C1} = \frac{1}{C_1 * R_{eqC1}} =  \frac{1}{42K \times 10^{-6}} = 23.34 rad / s$$ 

Para $C_2$ tenemos:

$$ \omega_{C2} = \frac{1}{C_2 * R_{eqC2}} =  \frac{1}{10K \times 10^{-6}} = 140.34 rad / s$$ 

Para $C_3$ tenemos:

$$ \omega_{C3} = \frac{1}{C_3 * R_{eqC3}} =  \frac{1}{10K \times 10^{-6}} = 23.34 rad / s$$ 

Para $C_5$ tenemos:

$$ \omega_{C5} = \frac{1}{C_5 * R_{eqC5}} =  \frac{1}{10K \times 10^{-6}} = 0.84 rad / s$$ 

Para $C_6$ tenemos:

$$ \omega_{C6} = \frac{1}{C_6 * R_{eqC6}} =  \frac{1}{10K \times 10^{-6}} = 442.6 rad / s$$ 

Para $C_7$ tenemos:

$$ \omega_{C7} = \frac{1}{C_7 * R_{eqC7}} =  \frac{1}{10K \times 10^{-6}} = 65.34 rad / s$$

sabiendo que: 

$$ f = \frac{\omega}{2\pi} = \frac{1}{T} $$

\begin{table}[ht]
    \centering
    \begin{tabular}{|c|c|c|}
    \hline
    Capacitor & Velocidad angular & Frecuencia\\
    \hline
    C1 & 23.34 rad / s & 3.401 Hz\\
    \hline
    C2 & 140.34 rad / s & 21.002 Hz\\
    \hline
    C3 & 23.34 rad / s & 3.561 Hz\\
    \hline
    C5 & 0.84 rad / s & 0.14 Hz\\
    \hline
    C6 & 442.6 rad / s & 70.414 Hz\\
    \hline
    C7 & 65.34 rad / s & 9.013 Hz\\
    \hline
    \end{tabular}
    \caption{Frecuencia de corte de los capacitores de baja frecuencia}
    \label{tab:frecuencia-de-corte-capacitores-baja-frecuencia}
\end{table}


La frecuencia de corte inferior es la mayor frecuencia entre todos los valores obtenidos, por lo tanto

$$ f_{L} = 70.414 Hz$$