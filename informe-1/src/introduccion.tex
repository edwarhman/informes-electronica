\section{Introducción}

En el ámbito de la electrónica, los amplificadores operacionales son componentes fundamentales que se utilizan en una amplia variedad de aplicaciones, desde sistemas de audio hasta equipos de medición y control. Un amplificador operacional es un dispositivo de alta ganancia con dos entradas y una salida, que puede amplificar señales de voltaje muy pequeñas.

El objetivo de este informe de laboratorio es analizar el comportamiento y las características de un amplificador operacional en sus etapas diferencial, impulsora y de potencia y del circuito multietapas en diferentes conilustraciónciones. Se realizarán mediciones de los puntos estáticos, modelos dinámicos, ganancia respuesta en frecuencia y utilizando circuitos con realimentación negativa y positiva.  
